\documentclass{zkdl-tests-template}

\title{\huge\sffamily\bfseries Lecture \#5 Exercises}
\author{\Large\sffamily Distributed Lab}
\date{\sffamily August 20, 2024}
\titlepic{\includegraphics[width=0.15\textwidth]{../lectures/images/logo.png}}

\begin{document}

\pagestyle{fancy}

\maketitle

\textbf{Exercise 1.} Denis decided to commit to the race results, but forgot to use the blinding factor.  
    He used a hash-based commitment and SHA256 hash function. 
    Set of race participant numbers: $(0x1, 0x3, 0x8, 0x15)$. Can you break the hiding property of the commitment? Select the participant number 
    Denis made a commitment to, if 
    
    $\mathcal{C} = 0xbeead77994cf573341ec17b58bbf7eb34d2711c993c1d976b128b3188dc1829a$.
\begin{enumerate}[(A)]
    \item $0x1$.
    \item $0x3$.
    \item $0x8$.
    \item $0x15$.
\end{enumerate}

\textbf{Exercise 2.} Denis made a setup (points $G$ and $U$) for a Pedersen commitment scheme and commited values $(3, 7)$ to Dmytro
by sending him $\mathcal{C} = [3]G + [7]U$. Dmytro did not verify the setup. It turns out that Denis knows the discrete logarithm of $U$ = $[6]G$. 
He wants to change the committed message
to $15$. Which values $(m, r)$ should he send to Dmytro at the opening stage?
\begin{enumerate}[(A)]
    \item $(15, 5)$
    \item $(15, 7)$
    \item $(15, 4)$
    \item $(3, 7)$
    \item $(3, 5)$
\end{enumerate}

\textbf{Exercise 3.} We define a dummy hash function $H(a, b) = (a \cdot 3 + b \cdot 7) \pmod{41}$. You have a Merkle tree built with depth $4$ 
using hash function $H$ with root equal $37$. Which inclusion proof is valid for element $3$?
Position defines how leaves should be hashed:

    - if $left \rightarrow h_i = Hash(h_{i-1}, branch[i])$

    - if $right \rightarrow  h_i = Hash(branch[i], h_{i-1})$

\begin{enumerate}[(A)]
    \item branch: $[4, 16, 13]$, position: $[left, right, left]$
    \item branch: $[1, 40, 3]$, position: $[left, left, left]$
    \item branch: $[5, 12, 13]$, position: $[right, right, left]$
    \item branch: $[4, 17, 13]$, position: $[left, right, left]$
\end{enumerate}

\textbf{Exercise 4.} Given a polynomial $p(x) = x^3 - 10x^2 + 31x - 30$, you want to prove that $p(2) = 0$. 
    Calculate a \textit{quotient} polynomial.

\begin{enumerate}[(A)]
    \item $q(x) = 2x^2 + 4x - 6$
    \item $q(x) = x^3 - 10x^2 + 30x - 28$
    \item $q(x) = x^2 - 8x + 15$
    \item $q(x) = x^2 + 5x + 18$
\end{enumerate}

\end{document}
