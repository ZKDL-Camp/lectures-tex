\documentclass{zkdl-tests-template}

\title{\huge\sffamily\bfseries Lecture \#1 Exercises}
\author{\Large\sffamily Distributed Lab}
\date{\sffamily July 18, 2024}
\titlepic{\includegraphics[width=0.2\textwidth]{../lectures/images/logo.png}}

\begin{document}

\pagestyle{fancy}

\maketitle

\textbf{Exercise 1.} Which of the following statements is \textbf{false}?
\begin{enumerate}
    \item $(\forall a, b \in \mathbb{Q}, a \neq b) \; (\exists q \in \mathbb{R}): \{a < q < b\}$.
    \item $(\forall \varepsilon > 0) \, (\exists n_{\varepsilon} \in \mathbb{N}) \, (\forall n \geq n_{\varepsilon}): \{1/n < \varepsilon\}$.
    \item $(\forall k \in \mathbb{Z}) \; (\exists n \in \mathbb{N}): \{n < k\}$.
    \item $(\forall x \in \mathbb{Z} \setminus \{-1\}) \; (\exists! y \in \mathbb{Q}): \{(x+1)y = 2\}$.
\end{enumerate}

\textbf{Exercise 2.} Denote $X := \{(x,y) \in \mathbb{Q}^2: xy = 1\}$. Oleksandr claims the following:
\begin{enumerate}
    \item $X \cap \mathbb{N}^2 = \{(1,1)\}$. 
    \item $|X \cap \mathbb{Z}^2| = 2|X \cap \mathbb{N}^2|$.
    \item $X$ is a group under the operation $(x_1,y_1) \oplus (x_2,y_2) = (x_1x_2, y_1y_2)$.
\end{enumerate}

Which statements are \textbf{true}?
\begin{enumerate}[a)]
    \item Only 1.
    \item Only 1 and 2.
    \item Only 1 and 3.
    \item Only 2 and 3.
    \item All statements are correct.
\end{enumerate}

\textbf{Exercise 3.} Does a tuple $(\mathbb{Z},\oplus)$ with operation $a \oplus b = a + b - 1$ define a group?
\begin{enumerate}[a)]
    \item Yes, and this group is abelian.
    \item Yes, but this group is not abelian.
    \item No, since the associativity property does not hold.
    \item No, since there is no identity element in this group.
    \item No, since there is no inverse element in this group.
\end{enumerate}

\pagebreak
\textbf{Exercise 4.} Consider the Cartesian plane $\mathbb{R}^2$, where two coordinates are real numbers. For two points $A,B$ define the operation $\oplus$
as follows: $A \oplus B$ is the midpoint on segment $AB$. Does $(\mathbb{R}^2, \oplus)$ define a group? 
\begin{enumerate}[a)]
    \item Yes, and this group is abelian.
    \item Yes, but this group is not abelian.
    \item No, since the associativity property does not hold and there is no identity element in this group.
    \item No, since the associativity property does not hold, but we might define an identity element nonetheless.
\end{enumerate}

\textbf{Exercise 5.} Find the inverse of $4$ in $\mathbb{F}_{11}$.
\begin{enumerate}[a)]
    \item 8
    \item 5
    \item 3
    \item 7
\end{enumerate}

\textbf{Exercise 6.} Suppose for three polynomials $p,q,r \in \mathbb{F}[x]$ we have $\deg p = 3, \deg q = 4, \deg r = 5$. Which of the following is true for $n := \deg \{(p-q)r\}$?
\begin{enumerate}[a)]
    \item $n = 9$.
    \item $n$ might be less than $9$.
    \item $n = 20$.
    \item $n$ is less than $\deg \{qr\}$. 
\end{enumerate}

\textbf{Exercise 7.} Define the polynomial over $\mathbb{F}_5$: $f(x) := 4x^2 + 7$. Which of the following is the root of $f(x)$?
\begin{enumerate}[a)]
    \item $2$
    \item $3$
    \item $4$
    \item This polynomial has no roots over $\mathbb{F}_5$.
\end{enumerate}

\textbf{Exercise 8.} Quadratic polynomial $p(x) = ax^2+bx+c \in \mathbb{R}[x]$ has zeros at $1$ and $2$ and $p(0) = 2$. Find the value of $a+b+c$.
\begin{enumerate}[a)]
    \item $0$
    \item $-1$
    \item $1$
    \item Not enough information to determine.
\end{enumerate}

\textbf{Exercise 9.} Which of the following is a \textbf{valid} endomorphism $f: X \to X$? 
\begin{enumerate}[a)]
    \item $X = [0,1]$, $f: x \mapsto x^2$.
    \item $X = [0,1]$, $f: x \mapsto x + 1$.
    \item $X = \mathbb{R}_{>0}$, $f: x \mapsto (x-1)^3$.
    \item $X = \mathbb{Q}_{>0}$, $f: x \mapsto \sqrt{x}$.
\end{enumerate}

\pagebreak
\textbf{Exercise 10*.} Denote by $\text{GL}(2,\mathbb{R})$ a set of $2\times 2$ invertable matrices with real entries. Define two functions $\varphi: \text{GL}(2,\mathbb{R}) \to \mathbb{R}$:
\begin{equation}
    \varphi_1 \left(\begin{bmatrix}
        a & b \\ c & d
    \end{bmatrix}\right) = ad - bc, \; \varphi_2 \left(\begin{bmatrix}
        a & b \\ c & d
    \end{bmatrix}\right) = a + d
\end{equation}

Den claims the following:
\begin{enumerate}
    \item $\varphi_1$ is a group homomorphism between multiplicative groups $(\text{GL}(2,\mathbb{R}), \times)$ and $(\mathbb{R}, \times)$.
    \item $\varphi_2$ is a group homomorphism between additive groups $(\text{GL}(2, \mathbb{R}), +)$ and $(\mathbb{R}, +)$.
\end{enumerate}

Which of the following is \textbf{true}?

\begin{enumerate}[a)]
    \item Only statement 1 is correct.
    \item Only statement 2 is correct.
    \item Both statements 1 and 2 are correct.
    \item None of the statements is correct.
\end{enumerate}

\end{document}
