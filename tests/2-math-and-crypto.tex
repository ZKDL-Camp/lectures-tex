\documentclass{zkdl-tests-template}

\title{\huge\sffamily\bfseries Lecture \#2 Exercises}
\author{\Large\sffamily Distributed Lab}
\date{\sffamily July 25, 2024}
\titlepic{\includegraphics[width=0.15\textwidth]{../lectures/images/logo.png}}

\begin{document}

\pagestyle{fancy}

\maketitle

\textbf{Exercise 1.} Suppose that for the given cipher with a security parameter $\lambda$, the adversary $\mathcal{A}$ can deduce the least significant bit of the plaintext from the ciphertext. Recall that the advantage 
of a bit-guessing game is defined as $\text{SS}\mathsf{Adv}[\mathcal{A}] = \left|\Pr[b=\hat{b}] - \frac{1}{2}\right|$, where $b$ is the randomly chosen bit of a challenger, while 
$\hat{b}$ is the adversary's guess. What is the maximal advantage of $\mathcal{A}$ in this case?

\textbf{Hint:} The adversary can choose which messages to send to challenger to further distinguish the plaintexts.

\begin{enumerate}[a)]
    \item $1$
    \item $\frac{1}{2}$
    \item $\frac{1}{4}$
    \item $0$
    \item Negligible value ($\text{negl}(\lambda)$).
\end{enumerate}

\textbf{Exercise 2.} Consider the cipher $\mathcal{E} = (E,D)$ with encryption function $E: \mathcal{K} \times \mathcal{M} \to \mathcal{C}$ over the message space $\mathcal{M}$, ciphertext space $\mathcal{C}$, and key space $\mathcal{K}$. We want to define the security
that, based on the cipher, the adversary $\mathcal{A}$ cannot restore the message (\textit{security against message recovery}). For that reason, we define the following game:
\begin{enumerate}
    \item Challenger chooses random $m \xleftarrow{R} \mathcal{M}, k \xleftarrow{R} \mathcal{K}$.
    \item Challenger computes the ciphertext $c \gets E(k,m)$ and sends to $\mathcal{A}$.
    \item Adversary outputs $\hat{m}$, and wins if $\hat{m} = m$.
\end{enumerate}

We say that the cipher $\mathcal{E}$ is secure against message recovery if the \textbf{message recovery advantage}, denoted as $\text{MR}\textsf{adv}[\mathcal{A}, \mathcal{E}]$ is negligible. Which of the following statements is a valid interpretation of the message recovery advantage?
\begin{enumerate}[a)]
    \item $\text{MR}\textsf{adv}[\mathcal{A},\mathcal{E}] := \left|\text{Pr}[m=\hat{m}] - \frac{1}{2}\right|$
    \item $\text{MR}\textsf{adv}[\mathcal{A},\mathcal{E}] := \left|\text{Pr}[m=\hat{m}] - 1\right|$.
    \item $\text{MR}\textsf{adv}[\mathcal{A},\mathcal{E}] := \text{Pr}[m=\hat{m}]$
    \item $\text{MR}\textsf{adv}[\mathcal{A},\mathcal{E}] := \left|\text{Pr}[m=\hat{m}] - \frac{1}{|\mathcal{M}|}\right|$
\end{enumerate}

\textbf{Exercise 3.} Suppose that $f$ and $g$ are negligible functions. Which of the following functions is not neccessarily negligible?
\begin{enumerate}[a)]
    \item $f + g$
    \item $f \times g$
    \item $f - g$
    \item $f/g$
    \item $h(\lambda) := \begin{cases}
        1/f(\lambda) & \text{if } 0 < \lambda < 100000 \\
        g(\lambda) & \text{if } \lambda \geq 100000
    \end{cases}$
\end{enumerate}

\textbf{Exercise 4.} Suppose that $f \in \mathbb{F}_p[x]$ is a $d$-degree polynomial with $d$ \textbf{distinct} roots in $\mathbb{F}_p$. What is the probability that, when evaluating $f$ at $n$ random points, the polynomial will be zero at all of them?

\begin{enumerate}[a)]
    \item Exactly $(d/p)^n$.
    \item Strictly less that $(d/p)^n$.
    \item Exactly $nd/p$.
    \item Exactly $d/np$.
\end{enumerate}

\textbf{Exercise 5-6.} To demonstrate the idea of Reed-Solomon codes, consider the toy construction. Suppose that our message is a tuple of two elements $a,b \in \mathbb{F}_{13}$. Consider function $f: \mathbb{F}_{13} \to \mathbb{F}_{13}$, defined as $f(x) = ax+b$, and define the encoding of the message $(a,b)$ as $(a,b) \mapsto (f(0),f(1),f(2),f(3))$. 

\textbf{Question 5.} Suppose that you received the encoded message $(3,5,6,9)$. Which number from the encoded message is corrupted?
\begin{enumerate}[a)]
    \item First element ($3$).
    \item Second element ($5$).
    \item Third element ($6$).
    \item Fourth element ($9$).
    \item The message is not corrupted.
\end{enumerate}

\textbf{Question 6.} Consider the previous question. Suppose that the original message was $(a,b)$. Find the value of $a \times b$ (in $\mathbb{F}_{13}$).
\begin{enumerate}[a)]
    \item $4$
    \item $6$
    \item $12$
    \item $2$
    \item $1$
\end{enumerate}

\end{document}
