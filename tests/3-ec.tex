\documentclass{zkdl-tests-template}

\title{\huge\sffamily\bfseries Lecture \#3 Exercises}
\author{\Large\sffamily Distributed Lab}
\date{\sffamily August 1, 2024}
\titlepic{\includegraphics[width=0.15\textwidth]{../lectures/images/common/logo.png}}

\begin{document}

\pagestyle{fancy}

\maketitle

\subsubsection*{Warmup (Oleksandr in search of perfect field extension)}

\textbf{Exercise 1.} Oleksandr decided to build $\mathbb{F}_{49}$ as $\mathbb{F}_7[i]/(i^2+1)$. Compute $(3+i)(4+i)$. 

\begin{enumerate}[a)]
    \item $6+i$.
    \item $6$.
    \item $4+i$.
    \item $4$.
    \item $2+4i$.
\end{enumerate}

\textbf{Exercise 2.} Oleksandr came up with yet another extension $\mathbb{F}_{p^2} = \mathbb{F}_p[i]/(i^2+2)$. He asked interns to calculate $2/i$. Based on five answers given below, help Oleksandr to find the correct one.
\begin{enumerate}[a)]
    \item $1$.
    \item $p-2$.
    \item $(p-3)i$.
    \item $(p-1)i$.
    \item $p-1$.
\end{enumerate}

\textbf{Exercise 3*.} After endless tries, Oleksandr has finally found the perfect field extension: $\mathbb{F}_{p^2} := \mathbb{F}_p[v] / (v^2+v+1)$. However, Oleksandr became very frustrated since not for any $p$ this would be a valid field extension. For which of the following values $p$ such construction would \textbf{not} be a valid field extension? Use the fact that equation $\omega^3=1$ over $\mathbb{F}_p$ has non-trivial solutions (meaning, two others except for $\omega=1$) if $p \equiv 1 \pmod{3}$. You can assume that listed numbers are primes.
\begin{enumerate}[a)]
    \item $8431$.
    \item $9173$.
    \item $9419$.
    \item $6947$.
\end{enumerate}

\pagebreak

\begin{tcolorbox}[colback=blue!5!white,fonttitle=\bfseries,colframe=blue!80!white,title=Exercises 4-9. Tower of Extensions]
    \textit{You are given the passage explaining the topic of tower of extensions. The text has gaps that you need to fill in with the correct statement among the provided choices.}
    \vspace{10px}
    
    This question demonstrates the concept of the so-called \textbf{tower of extensions}. Suppose we want to build an extension field $\mathbb{F}_{p^4}$. Of course, we can find some irreducible polynomial $p(X)$ of degree $4$ over $\mathbb{F}_p$ and build $\mathbb{F}_{p^4}$ as $\mathbb{F}_p[X]/(p(X))$. However, this method is very inconvenient since implementing the full $4$-degree polynomial arithmetic is inconvenient. Moreover, if we were to implement arithmetic over, say, $\mathbb{F}_{p^{24}}$, that would make the matters worse. For this reason, we will build $\mathbb{F}_{p^4}$ as $\mathbb{F}_{p^2}[j]/(q(j))$ where $q(j)$ is an irreducible polynomial of degree $2$ over $\mathbb{F}_{p^2}$, which itself is represented as $\mathbb{F}_p[i]/(r(i))$ for some suitable irreducible quadratic polynomial $r(i)$. This way, we can first implement $\mathbb{F}_{p^2}$, then $\mathbb{F}_{p^4}$, relying on the implementation of $\mathbb{F}_{p^2}$ and so on.
    \vspace{5px}

    For illustration purposes, let us pick $p:=5$. As noted above, we want to build $\mathbb{F}_{5^2}$ first. A valid way to represent $\mathbb{F}_{5^2}$ would be to set $\mathbb{F}_{5^2} := \boxed{\textcolor{blue}{\textbf{4}}}$. Given this representation, the zero of a linear polynomial $f(x) = ix - (i+3)$, defined over $\mathbb{F}_{5^2}$, is $\boxed{\textcolor{blue}{\textbf{5}}}$.
    \vspace{5px}

    Now, assume that we represent $\mathbb{F}_{5^4}$ as $\mathbb{F}_{5^2}[j]/(j^2-\xi)$ for $\xi = i+1$. Given such representation, the value of $j^4$ is $\boxed{\textcolor{blue}{\textbf{6}}}$. Finally, given $c_0+c_1j \in \mathbb{F}_{5^4}$ we call $c_0 \in \mathbb{F}_{5^2}$ a \textbf{real part}, while $c_1 \in \mathbb{F}_{5^2}$ an \textbf{imaginary part}. For example, the imaginary part of number $j^3+2i^2\xi$ is $\boxed{\textcolor{blue}{\textbf{7}}}$, while the real part of $(a_0+a_1j)b_1j$ is $\boxed{\textcolor{blue}{\textbf{8}}}$. Similarly to complex numbers, it motivates us to define the number's \textbf{conjugate}: for $z = c_0+c_1j$, define the conjugate as $\overline{z} := c_0-c_1j$. The expression $z\overline{z}$ is then $\boxed{\textcolor{blue}{\textbf{9}}}$.
\end{tcolorbox}

\begin{center}
    \begin{tabular}{p{5cm}p{5cm}p{5cm}}
        % Exercise 3
        \textbf{Exercise 4.}
        \begin{enumerate}[a)]
            \item $\mathbb{F}_5[i]/(i^2+1)$
            \item $\mathbb{F}_5[i]/(i^2+2)$
            \item $\mathbb{F}_5[i]/(i^2+4)$
            \item $\mathbb{F}_5[i]/(i^2+2i+1)$
            \item $\mathbb{F}_5[i]/(i^2+4i+4)$
        \end{enumerate} &   
        \textbf{Exercise 5.}
        \begin{enumerate}[a)]
            \item $1+i$
            \item $1+2i$
            \item $1+4i$
            \item $2+3i$
            \item $3+i$
        \end{enumerate} &
        \textbf{Exercise 6.}
        \begin{enumerate}[a)]
            \item $4+2i$
            \item $4i$
            \item $1$
            \item $1+2i$
            \item $2+4i$
        \end{enumerate}
        \\
        \textbf{Exercise 7.}
        \begin{enumerate}[a)]
            \item equal to zero.
            \item equal to one.
            \item equal to the real part.
            \item $2(1+i)$
            \item $-4$
        \end{enumerate} & 
        % Set A but not B
        \textbf{Exercise 8.}
        \begin{enumerate}[a)]
            \item $a_1b_1$
            \item $a_1b_1\xi$
            \item $a_0b_1$
            \item $a_0b_1\xi$
            \item $a_0a_1$
        \end{enumerate} & 
        \textbf{Exercise 9.}
        \begin{enumerate}[a)]
            \item $c_0^2+c_1^2$
            \item $c_0^2-c_1^2\xi$
            \item $c_0^2+c_1^2\xi^2$
            \item $(c_0^2+c_1^2\xi)j$
            \item $(c_0^2-c_1^2)j$
        \end{enumerate} 
    \end{tabular}
\end{center}

\pagebreak
\subsubsection*{Elliptic Curves}
\textbf{Exercise 10.} Suppose that elliptic curve is defined as $E/\mathbb{F}_{7}: y^2=x^3+b$. Suppose $(2,3)$ lies on the curve. What is the value of $b$?

\textbf{Exercise 11.} Sum of which of the following pairs of points on the elliptic curve $E/\mathbb{F}_{11}$ is equal to the point at infinity $\mathcal{O}$ for any valid curve equation?
\begin{enumerate}[a)]
    \item $P=(2,3),Q=(2,8)$.
    \item $P=(9,2),Q=(2,8)$.
    \item $P=(9,9),Q=(5,7)$.
    \item $P=\mathcal{O},Q=(2,3)$.
    \item $P=[10]G,Q=G$ where $G$ is a generator.
\end{enumerate}

\textbf{Exercise 12.} Consider an elliptic curve $E$ over $\mathbb{F}_{167^2}$. Denote by $r$ the order of the group of points on $E$ (that is, $r=|E|$). Which of the following \textbf{can} be the value of $r$?
\begin{enumerate}[a)]
    \item $167^2 - 5$
    \item $167^2-1000$
    \item $167^2+5 \cdot 167$
    \item $170^2$
    \item $160^2$
\end{enumerate}

\textbf{Exercise 13.} Suppose that for some elliptic curve $E$ the order is $|E| = qr$ where both $q$ and $r$ are prime numbers. Among listed, what is the most optimal complexity of algorithm to solve the discrete logarithm problem on $E$?
\begin{enumerate}[a)]
    \item $O(qr)$
    \item $O(\sqrt{qr})$
    \item $O(\sqrt{\max\{q,r\}})$
    \item $O(\sqrt{\min\{q,r\}})$
    \item $O(\max\{q,r\})$
\end{enumerate}

\end{document}
