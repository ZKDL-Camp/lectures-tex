\documentclass{zkdl-tests-template}

\title{\huge\sffamily\bfseries Lecture \#4 Exercises}
\author{\Large\sffamily Distributed Lab}
\date{\sffamily August 8, 2024}
\titlepic{\includegraphics[width=0.15\textwidth]{../lectures/images/logo.png}}

\begin{document}

\pagestyle{fancy}

\maketitle

\textbf{Exercise 1.} What is \textbf{not} a valid equivalence relation $\sim$ over a set $\mathcal{X}$?
\begin{enumerate}[(A)]
    \item $a \sim b$ iff $a+b < 0$, $\mathcal{X} = \mathbb{Q}$.
    \item $a\sim b$ iff $a=b$, $\mathcal{X} = \mathbb{R}$.
    \item $a\sim b$ iff $a \equiv b \pmod{5}$, $\mathcal{X} = \mathbb{Z}$.
    \item $a\sim b$ iff the length of $a$ = the length of $b$, $\mathcal{X} = \mathbb{R}^2$.
    \item $(a_1,a_2,a_3)\sim (b_1,b_2,b_3)$ iff $a_3=b_3$, $\mathcal{X} = \mathbb{R}^3$.
\end{enumerate}

\textbf{Exercise 2.} Suppose that over $\mathbb{R}$ we define the following equivalence relation: $a \sim b$ iff $a-b \in \mathbb{Z}$ ($a,b \in \mathbb{R}$). What is the equivalence class of $1.4$ (that is, $[1.4]_{\sim}$)?
\begin{enumerate}[(A)]
    \item A set of all real numbers.
    \item A set of all integers.
    \item A set of reals $x \in \mathbb{R}$ with the fractional part of $x$ equal to $0.4$.
    \item A set of reals $x \in \mathbb{R}$ with the integer part of $x$ equal to $1$.
    \item A set of reals $x \in \mathbb{R}$ with the fractional part of $x$ equal to $0.6$.
\end{enumerate}

\textbf{Exercise 3.} Which of the following pairs of points in homogeneous projective space $\mathbb{P}^2(\mathbb{R})$ are \textbf{not} equivalent?
\begin{enumerate}[(A)]
    \item $(1:2:3)$ and $(2:4:6)$.
    \item $(2:3:1)$ and $(6:9:3)$.
    \item $(5:5:5)$ and $(2:2:2)$.
    \item $(4:3:2)$ and $(16:8:4)$.
\end{enumerate}

\textbf{Exercise 4.} The main reason for using projective coordinates in elliptic curve cryptography is:
\begin{enumerate}[(A)]
    \item To reduce the number of point additions in algorithms involving elliptic curves.
    \item To make the curve more secure against attacks.
    \item To make the curve more efficient in terms of memory usage.
    \item To reduce the number of field multiplications when performing scalar multiplication.
    \item To avoid making too many field inversions in complicated algorithms involving elliptic curves.
\end{enumerate}

\textbf{Exercise 5.} Suppose $k=19$ is a scalar and we are calculating $[k]P$ using the double-and-add algorithm. How many elliptic curve point addition operations will be performed?
\begin{enumerate}[(A)]
    \item $0$.
    \item $1$.
    \item $2$.
    \item $3$.
    \item $4$.
\end{enumerate}

\textbf{Exercise 6.} What is the minimal number of inversions needed to calculate the value of expression (over $\mathbb{F}_p$)
\begin{equation*}
    \frac{a-b}{(a+b)^4} + \frac{c}{a+b} + \frac{d}{a^2+c^2},
\end{equation*}
for the given scalars $a,b,c,d \in \mathbb{F}_p$?
\begin{enumerate}[(A)]
    \item $1$.
    \item $2$.
    \item $3$.
    \item $4$.
    \item $5$.
\end{enumerate}

\textbf{Exercise 7.} Given pairing $e: \mathbb{G}_1 \times \mathbb{G}_2 \to \mathbb{G}_T$ with $G_1$ --- generator of $\mathbb{G}_1$ and $G_2 \in \mathbb{G}_2$ --- generator of $\mathbb{G}_2$, which of the following is \textbf{not} equal to $e([3]G_1, [5]G_2)$?
\begin{enumerate}[(A)]
    \item $e([5]G_1, [3]G_2)$.
    \item $e([4]G_1, [4]G_2)$.
    \item $e([15]G_1, G_2)$.
    \item $e([3]G_1,G_2)e(G_1,[12]G_2)$.
    \item $e(G_1, G_2)^{15}$.
\end{enumerate} 

\textbf{Exercise 8.} \textit{Unit Circle Proof.} Suppose Alice wants to convince Bob that she knows a point on the unit circle $x^2+y^2=1$. Suppose pairing is given by $e: \mathbb{G}_1 \times \mathbb{G}_2 \to \mathbb{G}_T$ and Alice computes $P \gets [x]G_1 \in \mathbb{G}_1, Q \gets [y]G_2 \in \mathbb{G}_2$. She then proceeds to sending $(P,Q)$ to Bob. Which of the following checks should Bob perform to verify that Alice indeed knows a point on the unit circle?
\begin{enumerate}[(A)]
    \item Check if $e(P,Q)e(Q,P)=1$.
    \item Check if $e([2]P,[2]Q) = e(G_1,G_2)$.
    \item Check if $e([2]P,Q)e(Q,[2]P) = 1$.
    \item Check if $e(P,P)+e(Q,Q) = 1$.
    \item Check if $e(P,P)e(Q,Q)=e(G_1,G_2)$.
\end{enumerate}

\end{document}
