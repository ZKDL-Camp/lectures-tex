\documentclass[../lecture-notes-148x210.tex]{subfiles}

\begin{document}
\subsection*{Introduction to Number Theory}

\textbf{Exercise 1. Answer: E.} Statement ``any prime number $p$ equals to 
the product of its divisors'' is true since the only divisors of $p$
are $1$ and $p$ by definition.

\textbf{Exercise 2. Answer: D.} Statement $\lcm(n+1,n-1)=n^2-1$ is false. For 
instance, take $n := 5$. Then, $\lcm(6,4) = 12 \neq 5^2-1 = 24$. 

\textbf{Exercise 3. Answer: C.} Answer is $9$. Indeed, $9 \cdot 5 = 45 \equiv 1 \pmod{11}$.

\textbf{Exercise 4. Answer: C.} Answer is $15$. Indeed, $\gcd(15, 21) = 3 \neq 1$, so 
$15 \not\in \mathbb{Z}_{21}^{\times}$ by definition.

\textbf{Exercise 5. Answer: A.} Notice that $n^2+1=n(n+1)-(n-1)$. Thus, if 
$(n+1)\mid(n^2+1)$ we should have $(n+1)\mid (n-1)$ which is possible 
only if $n-1=0$. Thus, $n=1$ is the only such number.

\textbf{Exercise 6. Answer: C.} Let us find roots of $f$. Notice that:
\begin{equation*}
    f(x) = 2x^2 - (3p+q)x + (p^2 + pq)
\end{equation*}

By equating $f(x)=0$ we get two solutions: $r_1 = p$ and $r_2=\frac{p+q}{2}$. One 
can easily check that the only possible $p$ and $q$ for $r_2$ to be prime is $(7,3)$.
Thus, the sum is $10$.

\textbf{Exercise 7. Answer: B.} We have the only $\alpha_1=100$ which 
by the definition of $\tau$ function gives $\tau(p^{\alpha_1})=\alpha_1+1=101$.

\textbf{Exercise 8. Answer: D.} Suppose $n=\prod_{j=1}^r p_j^{\alpha_j}$. Then,
$\tau(n) = \prod_{j=1}^r (\alpha_j+1) = 7$. The only possible way to represent $7$
as a product of integers is to have the only non-zero $\alpha$ such that $\alpha+1=7$.
Thus, $\alpha=6$, so $n=p^6$ for aribtrary prime $p$. Therefore, $\tau(n^2)=\tau(p^{12})=13$, 
following the reasoning from the previous exercise.

\textbf{Exercise 9. Answer: B.} Suppose $m=\prod_{j=1}^rp_j^{\alpha_j}$, so 
again $\tau(m) = \prod_{j=1}^r (\alpha_j+1)$. For this expression to be odd, we 
need every term $\alpha_j+1$ to be odd. This is possible only if every $\alpha_j$ is even.
Therefore, $m$ must be a square of a number. The smallest square greater than $n^2$ is 
of course $(n+1)^2$.

\subsection*{Introduction to Abstract Algebra}

\textbf{Exercise 1. Answer: E.} Notice that if $xy=1$ then $y=\frac{1}{x}$. If 
$x$ is an integer then $y$ is integer only if $x=\pm 1$. Therefore,
\begin{equation*}
    X \cap \mathbb{Z}^2 = \{(1,1),(-1,-1)\}, \quad X \cap \mathbb{N}^2 = \{(1,1)\}.
\end{equation*}

Therefore, first two statements are correct. When it comes to the third
statement, it is also true: indeed, the identity element is $(1,1)$ while the
inverse element for $(x_1,x_2)$ is $(1/x_1,1/x_2)$ (such operation is closed since
$(1/x_1)(1/x_2) = 1/(x_1x_2) = 1$). Other properties are also satisfied.

\textbf{Exercise 2. Answer: A.} Identity element in this group is $1$. Indeed,
$a \oplus 1 = 1 \oplus a = a+1-1 = a$. Associativity follows immediately from 
the sum operation associativity. Inverse element for $a$ is $2-a$ since:
\begin{equation*}
    a \oplus (2-a) = a + 2 - a - 1 = 1
\end{equation*}

All operations are closed since the sum and subtraction of two integers is an
integer.

\textbf{Exercise 3. Answer: C.} Suppose the identity element $E \in
\mathbb{R}^2$ exists. Then, $A\oplus E = E$ by definition. However, the 
midpoint of $AE$ is $E$ only if $A=E$. Since identity element depends 
on the chosen group element $A$, it does not exist.

\textbf{Exercise 4. Answer: E.} Indeed, take $A = \begin{bmatrix}
    2 & 0 \\ 0 & 2
\end{bmatrix}$. Suppose inverse $B$ exists. Then, $AB=E_{2 \times 2}$ where 
$E_{2 \times 2}$ is an identity matrix. Therefore, $\det(AB) = \det A \cdot \det B$.
Since $\det A = 4$ we have $\det B = \frac{1}{4}$. However, the determinant 
of a matrix, composed of integers, cannot be a fraction. Therefore, the inverse
does not exist for $A$.

\textbf{Exercise 5. Answer: A.} Let us check all the properties carefully:
\begin{itemize}
    \item \textbf{Closure:} Take $f,g \in \mathcal{F}$. The function $f \times
    g$ has a domain $\Omega$ and range $\mathbb{G}$ since $f(\omega)\times
    g(\omega)$ always outputs $\mathbb{G}$ for any $\omega \in \Omega$ due to
    the closure of $\mathbb{G}$.
    \item \textbf{Identity Element:} Define identity element $i \in \mathcal{F}$
    as a function $i(\omega) = e$ for any $\omega \in \Omega$ where $e$ is an
    identity element in $\mathbb{G}$. Then:
    \begin{equation*}
        (f \star i)(\omega) = f(\omega) \times i(\omega) = f(\omega) \times e = f(\omega) \implies f \star i = f
    \end{equation*}
    \item \textbf{Inverse:} Define inverse element for $f \in \mathcal{F}$ as 
    $g$ defined as $g := (f(\omega))^{-1}$ for every $\omega \in \Omega$. Then:
    \begin{equation*}
        (f \star g)(\omega) = f(\omega) \times (f(\omega))^{-1} = e \; \forall \omega \in \Omega \implies f \star g = i
    \end{equation*}
    \item \textbf{Associativity:} Holds due to the associativity of $\times$.
    \item \textbf{Commutativity:} Holds due to the commutativity of $\times$.
\end{itemize}

\textbf{Exercise 6. Answer: A.} First, $f: x \mapsto x^2$ is obviously closed over
$[0,1]$. Let us check why it is an automorphism. First, check whether it is a homomorphism:
\begin{equation*}
    f(x \cdot y) = (x \cdot y)^2 = x^2 \cdot y^2 = f(x) \cdot f(y)
\end{equation*}

Now, let us show that $f$ is a bijection. Indeed, $f$ is injective since
$f(x)=f(y)$ implies $x^2=y^2$ which is possible only if $x=y$ when both $x$ and
$y$ are positive. $f$ is surjective since for any $y \in [0,1]$ there is 
$x=\sqrt{y}$ such that $f(x)=y$. Therefore, $f$ is an automorphism.

\textbf{Exercise 7. Answer: C.} In fact, $\varphi_1=\det$ while
$\varphi_2=\text{trace}$. It is well-known that $\det(AB)=\det A \det B$ and
$\text{trace}(A+B)=\text{trace}(A)+\text{trace}(B)$. Therefore, both 
functions are homomorphisms.

\textbf{Exercise 8. Answer: A.} Let us check the homomorphism property. Take 
$a+bi,c+di \in \mathbb{C} \setminus \{0\}$. Then:
\begin{equation*}
    \psi((a+bi)(c+di)) = \psi((ac-bd)+(ad+bc)i) = \begin{bmatrix}
        ac-bd & -(ad+bc) \\ ad+bc & ac-bd
    \end{bmatrix}
\end{equation*}

On the other hand:
\begin{equation*}
    \psi(a+bi)\psi(c+di) = \begin{bmatrix}
        a & -b \\ b & a
    \end{bmatrix}\cdot \begin{bmatrix}
        c & -d \\ d & c
    \end{bmatrix} = \begin{bmatrix}
        ac-bd & -(ad+bc) \\ ad+bc & ac-bd
    \end{bmatrix}
\end{equation*}

Therefore, $\psi((a+bi)(c+di))=\psi(a+bi)\psi(c+di)$, so $\psi$ is a homomorphism.
Let us show it is a bijection. Suppose $\psi(a+bi)=\psi(c+di)$. Then:
\begin{equation*}
    \begin{bmatrix}
        a & -b \\ b & a
    \end{bmatrix} = \begin{bmatrix}
        c & -d \\ d & c
    \end{bmatrix} \implies a=c, b=d
\end{equation*}

Thus, $\psi$ is injective. To show the surjectivity notice that for any matrix
$\begin{bmatrix} a & -b \\ b & a \end{bmatrix}$ there is $a+bi$ such that
$\psi(a+bi)=\begin{bmatrix} a & -b \\ b & a \end{bmatrix}$. Therefore, $\psi$ is
a bijection. Therefore, $\psi$ is an isomorphism.

\textbf{Exercise 9.} Let $\mathbb{G}$ be a group of order $9$. If $\mathbb{G}$
contains the element of order $9$, then it is cyclic and thus abelian, so we are
done. Suppose this is not the case and we have the element $x \in \mathbb{G}$ of
order other than $9$. Due to Lagrange's theorem, the order of $g$ must divide
the order of $\mathbb{G}$, so it is either $1$, $3$ or $9$. Since $1$
corresponds to the identity element, suppose $\mathbb{H}_X = \langle x\rangle$
has order $3$. Let $y \in \mathbb{G} \setminus \mathbb{H}_X$ be some other
non-identity element of $\mathbb{G}$ and $\mathbb{H}_Y := \langle y \rangle$.
Since $\mathbb{H}_X \cap \mathbb{H}_Y$ is a subgroup of $\mathbb{H}_X$ and
$|\mathbb{H}_X|=3$ by Lagrange's Theorem either $\mathbb{H}_X \cap
\mathbb{H}_Y=\{e\}$ or equals $\mathbb{H}_X$. Since $y \not\in \mathbb{H}_X$, we
conclude that $\mathbb{H}_X \cap \mathbb{H}_Y=\{e\}$. Therefore,
\begin{equation*}
    |\mathbb{H}_X\mathbb{H}_Y| = \frac{|\mathbb{H}_X|\cdot|\mathbb{H}_Y|}{|\mathbb{H}_X \cap \mathbb{H}_Y|} = 9
\end{equation*}

Therefore, $\mathbb{G} = \mathbb{H}_X\mathbb{H}_Y =
\{e,x,x^2,y,y^2,xy,xy^2,x^2y,x^2y^2\}$ since $x^3=y^3=e$. Now, consider $yx \in
\mathbb{G}$. It must be one of elements listed in $\mathbb{G}$:
\begin{itemize}
    \item $yx \neq e$, otherwise $y=x^2$ but $\mathbb{H}_X \cap \mathbb{H}_Y = \{e\}$.
    \item $yx \neq x$, otherwise $y=e$.
    \item $yx \neq x^2$, otherwise $x=y$.
    \item $yx \neq y$, otherwise $x=e$.
    \item $yx \neq y^2$, otherwise $x=y$.
    \item $yx \neq xy^2$, otherwise $yxy=x$ (multiplied both sides by $y$), but $(yx)^3=yx(yxy)x=yx^3=y=e$, so $y=e$.
    \item $yx \neq x^2y$, otherwise $xyx=y$ and thus $x=e$ (see explanation above).
    \item $yx \neq x^2y^2$, otherwise $(yx)^2=(yx)(yx)=(x^2y^2)(yx) = e$. Yet, $(yx)^3=e$, so $yx=e$ or $y=x^2$, contradicting $\mathbb{H}_X \cap \mathbb{H}_Y = \{e\}$.
\end{itemize} 

Therefore, $yx=xy$, so $\mathbb{G}$ is an abelian group.

\subsection*{Polynomials}

\textbf{Exercise 1. Answer: A.} Note that $\deg(p-q)=4$ since $p$ and $q$ are of
different degrees. Therefore, $\deg((p-q)r)=\deg(p-q)+\deg(r) = 4+5=9$.

\textbf{Exercise 2. Answer: D.} It suffices to plug each of $\{0,1,2,3,4\}$
in $4x^2+7$ and verify that the result is never $0$ over $\mathbb{F}_5$.

\textbf{Exercise 3. Answer: A.} Note that $p(x)=a(x-1)(x-2)$. Since $p(0)=2$, 
we have $a=1$. Therefore, $p(x)=x^2-3x+2$ and thus $b=-3$, $c=2$. The sum 
$a+b+c$ is then $0$.

\textbf{Exercise 4. Answer: C.} Roots of $Q$ are $\pm i$. If $Q(x) \mid P(x)$,
we must have $P(\pm i) = 0$. Thus we have $P(i)=i^{2n}+i^n-2=(-1)^n+i^n-2$. 
Consider four cases:
\begin{itemize}
    \item $n=4k$. Then, $P(i)=(-1)^{4k}+i^{4k}-2=0$.
    \item $n=4k+1$. Then, $P(i)=(-1)^{4k+1}+i^{4k+1}-2=-3+i \neq 0$.
    \item $n=4k+2$. Then, $P(i)=(-1)^{4k+2}+i^{4k+2}-2=-2 \neq 0$.
    \item $n=4k+3$. Then, $P(i)=(-1)^{4k+3}+i^{4k+3}-2=-3-i \neq 0$.
\end{itemize}

Therefore, $n$ must be divisible by $4$. It is easy to check that in such 
case $P(-i)$ equals $0$ as well.

\textbf{Exercise 5. Answer: A.} Since polynomial has $d$ distinct roots,
the probability of choosing a root is \emph{exactly} $d/p$. The probability 
that Alice fools Bob after $n$ rounds is then $(d/p)^n$.

\textbf{Exercise 6. Answer: B.} According to the Schwartz-Zippel Lemma, we have
$\text{Pr}[f(\mathbf{x})=0 \,|\, \mathbf{x} \xleftarrow{R} \mathbb{S}] \leq \deg
f/|\mathbb{S}|$, so we need to find the size $|\mathbb{S}|$. Our claim is that 
the number of solutions to $x_1+\dots+x_n=1$ over $\mathbb{F}_p$ is exactly 
$p^{n-1}$. Indeed, for any $p^{n-1}$ possible values of $(x_2,\dots,x_n)$ there 
is a unique $x_1$ such that $x_1+\dots+x_n=1$, which is $1-\sum_{i=2}^nx_i$. 
Therefore, $\text{Pr}[f(\mathbf{x})=0] \leq \deg f/p^{n-1}$.

\subsection*{Field Extensions}

\textbf{Exercise 1. Answer: D.} Since $i^2=-1$, we have
$(3+i)(4+i)=12+7i+i^2=11+7i$. Reducing this modulo $7$ gives $4$.

\textbf{Exercise 2. Answer: D.} Note that $2/i=2i/i^2=2i/(-2)=-i=(p-1)i$. 

\textbf{Exercise 3. Answer: A.} Such field extension is valid as long as $v^2+v+1$
has no roots over $\mathbb{F}_p$. Note that $v^2+v+1=0$ has no solutions if and
only if $v^3-1$ has no trivial solutions: indeed, $v^3-1=(v-1)(v^2+v+1)$. From
the problem statement we know this is the case when $p \equiv 1 \pmod{3}$. From 
the provided choice, this is the case for $p=8431$.

\textbf{Exercise 4. Answer: B.} Field extension $\mathbb{F}_5[i]/(r(i))$
is valid only when $r(x)$ is irreducible over $\mathbb{F}_5$. This is the 
case when $r(x) = x^2+2$.

\textbf{Exercise 5. Answer: A.} The zero of the provided polynomial is:
\begin{equation*}
    x_0 = \frac{i+3}{i} = \frac{i^2+3i}{i^2} = 1 + \frac{3i}{-2} = 1 + -3i \cdot 3 = 1-9i=1+i
\end{equation*}

\textbf{Exercise 6. Answer: A.} Notice that $j^4=(j^2)^2=\xi^2$. Since $\xi=1+i$
by definition, we have $j^4=(1+i)^2=-1+2i=4+2i$.

\textbf{Exercise 7. Answer: C.} Again, using identity $j^2=\xi$, we arrive at 
$j^3+2i^2\xi = j\xi-4\xi = j\xi+\xi$. Therefore, both the real and imaginary 
parts are same and equal to $\xi$.

\textbf{Exercise 8. Answer: B.} Let us expand the expression: $(a_0+a_1j)b_1j=
a_0b_1j+a_1b_1j^2=a_1b_1\xi+a_0b_1j$. So the real part is thus $a_1b_1\xi$.

\textbf{Exercise 9. Answer: B.} $z\overline{z}=(c_0+c_1j)(c_0-c_1j) = c_0^2-c_1^2j^2=c_0^2-\xi c_1^2$.

\subsection*{Elliptic Curves and Pairing}

\textbf{Exercise 1. Answer: 1.} If $(2,3) \in E/\mathbb{F}_7$, then $3^2=2^3+b$ 
and thus $b=9-8=1$. 

\textbf{Exercise 2. Answer: A.} If $P\oplus Q=\mathcal{O}$, then $P$ and $Q$ must have 
the same $x$ coordinates and opposite $y$ coordinates. Points $P=(2,3)$ and 
$Q=(2,8)$ satisfy this condition.

\textbf{Exercise 3. Answer: A.} According to Hasse's theorem, we have
\begin{equation*}
    r = p^2 + 1 - t, \quad |t| \leq 2p
\end{equation*}

In our case, $p=167$, so $r=167^2 + 1 - t$ for $|t| \leq 334$. The only 
suitable option among provided is $167^2 - 5$.

\textbf{Exercise 4. Answer: C.} Upper bound for factors of $|E|=qr$ is
$\max\{q,r\}$, so the complexity is $\mathcal{O}(\sqrt{\max\{q,r\}})$.

\textbf{Exercise 5. Answer: A.} Note that reflexivity is not satisfied. Indeed,
$a+a<0$ does not hold for $a \in \mathbb{Q}_{>0}$.

\textbf{Exercise 6. Answer: C.} By definition, $[1.4]_{\sim} = \{x \in \mathbb{R}: x-1.4 \in \mathbb{Z}\}$. That means 
that every $x \in [1.4]_{\sim}$ is of form $x=1.4+n$ where $n \in \mathbb{Z}$. This can be 
further simplified to $x=0.4 + n'$ with $n' \in \mathbb{Z}$. Thus, the fractional part 
must be $0.4$.

\textbf{Exercise 7. Answer: D.} For homogeneous projective coordinates we must
have $X_2/X_1=Y_2/Y_1=Z_2/Z_1$. In our case, the fraction $X_2/X_1=16/4=4$ while
$Y_2/Y_1=8/3$, so two points are not equivalent.

\textbf{Exercise 8. Answer: E.} Projective coordinates are effective since 
addition of points over the projective space does not require inversion. The 
only inversion needed is during the transforming point back to the affine space.

\textbf{Exercise 9. Answer: D.} The number of additions equals the number of 
non-zero bits in the binary representation of $k$. Since $19=10011_2$, the number
of additions is $3$.

\textbf{Exercise 10. Answer: A.} One way to reduce the number of inversions 
is to calculate the $t \gets (a+b)^{-1}$ first, which gives $(a+b)^{-4}=t^4$
for free. All left to do is to calculate $(a^2+c^2)^{-1}$, making the 
total cost $2$ inversions. However, we can use only one inversion: notice 
that
\begin{equation*}
    \frac{a-b}{(a+b)^4} + \frac{c}{a+b} + \frac{d}{a^2+c^2} = \frac{(a-b)(a^2+c^2) + c(a+b)^3(a^2+c^2) + d(a+b)^4}{(a+b)^4(a^2+c^2)}
\end{equation*}

Therefore, we need only one inversion: first we calculate $z \gets (a+b)^4(a^2+c^2)$ and then calculate $z^{-1}$. Then we multiply 
by the numerator, costing zero inversions. So the optimal solution is $1$ inversion. Note that the second approach 
requires larger number of field multiplications, so it might be not optimal in case inversion is cheap.

\textbf{Exercise 11. Answer: B.} Notice that $e([4]G_1,[4]G_2)=e(G_1,G_2)^{4 \cdot 4} = e(G_1,G_2)^{16} \neq e([3]G_1,[5]G_2)=e(G_1,G_2)^{15}$.

\textbf{Exercise 12. Answer: E.} Note that
$e(P,P)e(Q,Q)=e(G,G)^{x^2}e(G,G)^{y^2}=e(G,G)^{x^2+y^2}$. Therefore, checking
whether $e(P,P)e(Q,Q)=e(G,G)$ is equivalent to checking whether $x^2+y^2=1$.

\subsection*{Commitment Schemes}

\textbf{Exercise 1. Answer: C.} We have $H(x)=C$. By the definition of $H$,
we have $13x+17=39$ over $\mathbb{F}_{41}$. Therefore, $13x=22$,
so $x=22 \cdot 13^{-1} = 8$. 

\textbf{Exercise 2. Answer: A.} If $U=[6]G$ then the Pedersen commitment for element
$(m,r)$ can be written as $\mathsf{com}(m,r) = [m]G+[r]U=[m]G+[6r]G=[m+6r]G$.
Therefore, to fool Dmytro, if suffices to take such $(m_2,r_2)$ that
$m_2+6r_2=m_1+6r_1$. Substituting numbers we get $15+6r_2=3+6\cdot 7=45$.
Therefore, $6r_2=30$ so finally $r_2=5$.

\textbf{Exercise 3. Answer: D.} It suffices to manually check every option. For
option (D) we indeed have $H(H(17, H(3,4)), 13) = 37$.  

\textbf{Exercise 4. Answer: C.} The quotient polynomial $q(x)$ must satisfy
$q(x)\cdot (x-2)=p(x)$. It is easy to check that for $q(x)=x^2-8x+15$ we indeed
have $(x^2-8x+15)(x-2) = x^3-10x^2+31x-30$.

\subsection*{Security Analysis}

\textbf{Exercise 1. Answer: B.} Adversary $\mathcal{A}$ can safely send two
different messages $m_0$ and $m_1$ to the challenger with $\mathsf{lsb}(m_b)=b$
for $b \in \{0,1\}$. Then, having received $c$, $\mathcal{A}$ can output $b'
\gets \mathcal{O}_{\text{lsb}}(c)$ where $\mathcal{O}_{\text{lsb}}(c)$ is an
oracle that returns the least significant bit of $m$. Thus,
$\text{Pr}[b=\hat{b}] = 1$ and thus $\text{SSAdv}[\mathcal{A}]=1/2$.

\textbf{Exercise 2. Answer: D.} The probability should be subtracted from 
the probability of random guess. The random guess corresponds to the probability 
$1/|\mathcal{M}|$, thus the answer.

\textbf{Exercise 3. Answer: D.} Take $f=g$. Then, $f/g=1 \neq
\text{negl}(\lambda)$.

\subsection*{Introduction to Zero-Knowledge Proofs}

\textbf{Exercise 1. Answer: A.}

\textbf{Exercise 2. Answer: B.} The probability of accepting the false statement after 
$n$ rounds is $(1/8)^n=2^{-3n}$. To guarantee $120$ bits of security, we 
need $3n=120$ or $n=40$ rounds.

\textbf{Exercise 3. Answer: C.} To check a great explanation, see the answer
from
\href{https://crypto.stackexchange.com/questions/9528/zero-knowledge-proof-using-quadratic-residue-why-two-options}{this
forum}.

\textbf{Exercise 4. Answer: C.}

\textbf{Exercise 5. Answer: B.}

\textbf{Exercise 6. Answer: D.}

\subsection*{Sigma Protocols}

\textbf{Exercise 1.} The protocol is sound and complete, but it is not zero-knowledge.

\textbf{Exercise 2.} The protocol is not complete. Yet, it is sound and zero-knowledge.

\textbf{Exercise 3.} While the protocol is sound and complete, it is not zero-knowledge: 
based on $(a,r)$, the verifier can find $\alpha$ as $a-r$.

\textbf{Exercise 4.} The protocol is complete and zero-knowledge, but it is not
sound. Indeed, the prover does not prove that $r$ is chosen randomly, so he can
send $(g^z/h, z)$ to the verifier, which would make the verifier accept the
statement, yet the prover might not know the discrete logarithm of $h$.

\textbf{Exercise 5.} This is a Schnorr's protocol. It is sound, complete, and
zero-knowledge.

\textbf{Exercise 6. Answer: B.} Note that
$w=g^{\alpha\beta}=(g^{\alpha})^{\beta}=u^{\beta}$. 

\textbf{Exercise 7. Answer: C.} $v=g^{\beta}$ and $w=u^{\beta}$.

\textbf{Exercise 8. Answer: C.} $v_r=g^{\beta_r}$ and $w_r=u^{\beta_r}$.

\textbf{Exercise 9. Answer: A.} We need to hash both the initial statement
$(u,v,w)$ and the commitment $(v_r,w_r)$, so the hash function should input five 
group elements.

\textbf{Exercise 10. Answer: A.} $c \gets H((u,v,w),(v_r,w_r))$.

\end{document}
