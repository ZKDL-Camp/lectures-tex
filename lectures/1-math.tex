\documentclass[../lecture-notes.tex]{subfiles}

\begin{document}
\subsection{Introduction to Abstract Algebra}

\subsubsection{Group Definition}

Throughout the lectures, probably the most important topic is the \textit{group theory}. 

As you can recall from the high school math, typically real-world processes are described using real numbers, denoted by $\mathbb{R}$. For example, to describe the position or the velocity of an object, you would rather use real numbers. 

When it comes to working with computers though, real numbers become very inconvenient to work with. For instance, different programming languages might output different values for quite a straightforward operation \textsf{2.01 + 2.00}. This becomes a huge problem when dealing with cryptography, which must check \textit{precisely} whether two quantities are equal. For example, if the person's card number is $N$ and the developed system pays from a different, but very similar card with number $N + k$ for $k \ll N$, then this system can be safely thrown out of the window.

This motivates us to work with integers (denoted by $\mathbb{Z}$), instead. This solves the problem with card numbers, but for cryptography this object is still not really suitable since it is hard to build a secure and reliable protocol exploiting pure integers. 

This motivates us to use a different primitive for dealing with cryptographic systems. Similarly to programmers working with interfaces (or traits, if you are the \textit{Rust} developer), mathematicians also use the so-called \textit{groups} to represent objects obeying a certain set of rules. The beauty is that we do not concretize \textit{how} operations in this set are performed, but rather state the fact that we can somehow combine elements with the pre-defined properties. We can then discover properties of such objects and whenever we apply the concrete ``implementation'' (spoiler, group of points on elliptic curve), these properties would still hold.

\begin{remark}
    Further discussion with abstract objects should be regarded as ``interfaces'' which do not concretize the ``implementation'' of an object. It merely shows the nature of an object without going into the details.
\end{remark}

Now, let us get dirty and define what the \textbf{group} is.

\begin{definition}
    \textbf{Group}, further denoted by $(\mathbb{G},\oplus)$, is a set with a binary operation $\oplus$, obeying the following rules:
    \begin{enumerate}
        \item \textbf{Closure:} Binary operations always outputs an element from $\mathbb{G}$, that is $\forall a,b \in \mathbb{G}: a \oplus b \in \mathbb{G}$.
        \item \textbf{Associativity:} $\forall a,b,c \in \mathbb{G}: (a \oplus b)\oplus c = a \oplus (b \oplus c)$.
        \item \textbf{Identity element:} There exists a so-called identity element $e \in \mathbb{G}$ such that $\forall a \in \mathbb{G}: e \oplus a = a \oplus e = a$.
        \item \textbf{Inverse element:} $\forall a \in \mathbb{G} \; \exists a^{-1} \in \mathbb{G}: a\oplus a^{-1} = a^{-1} \oplus a = e$.
    \end{enumerate}
\end{definition}

Quite confusing at first glance, right? The best way to grasp this concept is to consider a couple of examples.

\begin{example}
    A group of integers with the regular addition $(\mathbb{Z},+)$ (also called the \textit{additive} group of integers) is a group. Indeed, an identity element is $e_{\mathbb{Z}}=0$, associativity obviously holds, and an inverse for each element $a \in \mathbb{Z}$ is $a^{-1} := -a \in \mathbb{Z}$. 
\end{example}

\begin{remark}
    Sometimes we use the term \textit{additive group} when we mean that the binary operations in the set is addition $+$, while \textit{multiplicative group} means that we are multiplying two numbers via $\cdot$.
\end{remark}

\begin{example}
    The multiplicative group of positive real numbers $(\mathbb{R}_{> 0}, \cdot)$ is a group for similar reasons. An identity element is $e_{\mathbb{R}_{>0}} = 1$, while the inverse for $a \in \mathbb{R}_{>0}$ is defined as $a^{-1} = \frac{1}{a}$.
\end{example}

\begin{example}
    The additive group of natural numbers $(\mathbb{N}, +)$ is not a group. Although operation of addition is closed, there is no identity element nor inverse element for, say, $2$ or $10$.
\end{example}

One might ask a reasonable question: suppose you pick $a,b \in \mathbb{G}$. Is $a \oplus b$ the same as $b \oplus a$? Unfortunately, generally, this is not true. Indeed, consider $(\mathbb{Z}, -)$ -- this is clearly a group. However,  $a-b \neq b-a$ for obvious reasons. 

\subsubsection{Subgroup}


\subsection{Basic Number Theory}

Here be number theory

\subsection{Other Primitives}

Here be vector spaces etc.

\end{document}