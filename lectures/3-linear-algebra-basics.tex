\documentclass[../lecture-notes.tex]{subfiles}

\begin{document}

\subsection{Linear Algebra Basics}

Although linear algebra is out of the main scope of this course, familiarity
with its basic definitions and concepts is still essential (especially when
considering Linear PCPs further from \Cref{section:r1cs}). In this section, we
provide a brief review of the key principles that are used throughout the
course. You can skip this section without any doubts if you are already
familiare with that, and come back to it whenever you need to refresh your
memory.

\subsubsection{Vector Space}

Similarly to group theory working with \textit{groups}, the linear algebra also has a special designated primitive --- \textbf{vector space}. If previously we were working
with the (finite) field $\mathbb{F}$, now we will work with the vector space $V$ over this field. In many practical applications,
vector space is formed by \textbf{vectors} consisting of a finite fixed collection of elements from the field $\mathbb{F}$. For example,
the vector space might be simply $\mathbb{F}^n$: the set of all $n$-tuples $(x_1,x_2,\dots,x_n)$ of elements from $\mathbb{F}$. Yet, let us give a bit more general definition.

\begin{definition}
    A \textbf{vector space} $V$ over the field $\mathbb{F}$ is an abelian group for addition $+$ together with a scalar multiplication operation $\cdot$ from $\mathbb{F} \times V$ to $V$, sending $(\lambda,x) \mapsto \lambda x$ and such that for any $\mathbf{v},\mathbf{u} \in V$ and $\lambda,\mu \in \mathbb{F}$ we have:
    \begin{itemize}
        \item $\lambda(\mathbf{u}+\mathbf{v}) = \lambda \mathbf{u} + \lambda \mathbf{v}$
        \item $(\lambda + \mu)\mathbf{v} = \lambda \mathbf{v} + \mu \mathbf{v}$
        \item $(\lambda \mu)\mathbf{v} = \lambda(\mu \mathbf{v})$
        \item $1\mathbf{v} = \mathbf{v}$
    \end{itemize}

    Any element $\mathbf{v} \in V$ is called a \textbf{vector}, and any element $\lambda \in \mathbb{F}$ is called a \textbf{scalar}. We also mark vector elements in boldface.
\end{definition}

\begin{example}
    For example, $V = \mathbb{F}^n$ with operations defined as:
    \begin{equation*}
        \begin{aligned}
            \lambda \cdot (x_1,x_2,\dots,x_n) &= (\lambda x_1, \lambda x_2, \dots, \lambda x_n) \\
            (x_1,x_2,\dots,x_n) + (y_1,y_2,\dots,y_n) &= (x_1+y_1, x_2+y_2, \dots, x_n+y_n)
        \end{aligned}
    \end{equation*}
    is a vector space. Similarly, the following three sets $V_1,V_2,V_3$ with operations defined above are also valid vector spaces:
    \begin{equation*}
        \begin{aligned}
            &V_1 = \{(x_1,x_2,\dots,x_n) \in \mathbb{F}^n: x_1 = 0\} \\
            &V_2 = \{(x_1,x_2,\dots,x_n) \in \mathbb{F}^n: x_1-x_3 = 0\} \\
            &V_3 = \{(x_1,x_2,\dots,x_n) \in \mathbb{F}^n: x_1+x_2+\dots+x_n = 0\}
        \end{aligned}
    \end{equation*}

    \textbf{Reasoning.} Let us see why, for example, $V_3$ is a valid vector space. Since all operations 
    are inherited from $\mathbb{F}^n$, we only need to check that $V_3$ is closed under addition and scalar multiplication.
    Suppose $\mathbf{v}_1=(x_1,\dots,x_n),\mathbf{v}_2=(y_1,\dots,y_n) \in V_3$. Then:
    \begin{equation*}
        \mathbf{v}_1 + \mathbf{v_2} = (x_1+x_2,\dots,x_n+y_n)
    \end{equation*}
    Note that $\sum_{i=1}^n x_i+y_i = \left(\sum_{i=1}^n x_i\right) + \left(\sum_{i=1}^ny_i\right) = 0 + 0 = 0$, so the sum is indeed closed.
\end{example}

We will also need to define one very important notion: the linear dependence and independence of vectors.

\begin{definition}
    A set of vectors $\{\mathbf{v}_1, \mathbf{v}_2, \dots, \mathbf{v}_k\}$ in a vector space $V$ over a field $\mathbb{F}$ is said to be \textbf{linearly independent} if the only scalars $\lambda_1, \lambda_2, \dots, \lambda_k \in \mathbb{F}$ that satisfy the equation
    \begin{equation*}
        \lambda_1 \mathbf{v}_1 + \lambda_2 \mathbf{v}_2 + \cdots + \lambda_k \mathbf{v}_k = \mathbf{0}
    \end{equation*}
    are $\lambda_1 = \lambda_2 = \cdots = \lambda_k = 0$. If there exist scalars, not all zero, that satisfy this equation, then the vectors are said to be \textbf{linearly dependent}.
\end{definition}

\begin{example}
    Consider the vectors $\mathbf{v}_1 = (1, 0, 0)$, $\mathbf{v}_2 = (0, 1, 0)$, and $\mathbf{v}_3 = (0, 0, 1)$ in $\mathbb{R}^3$. These vectors are linearly independent because the only solution to
    \begin{equation*}
        \lambda_1 \mathbf{v}_1 + \lambda_2 \mathbf{v}_2 + \lambda_3 \mathbf{v}_3 = \mathbf{0}
    \end{equation*}
    is $\lambda_1 = \lambda_2 = \lambda_3 = 0$.
\end{example}

\begin{example}
    Consider the vectors $\mathbf{u}_1 = (1, 2, 3)$, $\mathbf{u}_2 = (2, 4, 6)$ in $\mathbb{R}^3$. These vectors are linearly dependent because $\mathbf{u}_2 = 2\mathbf{u}_1$, so the equation
    \begin{equation*}
        \lambda_1 \mathbf{u}_1 + \lambda_2 \mathbf{u}_2 = \mathbf{0}
    \end{equation*}
    has non-trivial solutions, such as $\lambda_1 = 2$ and $\lambda_2 = -1$.
\end{example}

\subsubsection{Matrix}

Besides vectors, frequently we are working with \textbf{matrices}. In the most
basic sense, the matrix is a rectangular array of numbers, symbols, or
expressions, arranged in rows and columns. For example, the matrix $A$ with $m$
rows and $n$ columns, consisting of elements from the finite field $\mathbb{F}$
is denoted as $A \in \mathbb{F}^{m \times n}$. Additionally, we use notation $A
= \{a_{i,j}\}_{i,j=1}^{m \times n}$ to denote the square matrix $A$ of size $m
\times n$ with elements $a_{i,j}$. Now, let us define operations on matrices.

\begin{definition}
    Let $A,B$ be two matrices over the field $\mathbb{F}$. The following operations are defined:
    \begin{itemize}
        \item \textbf{Matrix addition/subtraction}: $A \pm B = \{a_{i,j} \pm b_{i,j}\}_{i,j=1}^{m \times n}$. The matrices $A$ and $B$ must have the same size $m \times n$.
        \item \textbf{Scalar multiplication}: $\lambda A = \{\lambda a_{i,j}\}_{1\leq i,j \leq n}$ for any $\lambda \in \mathbb{F}$.
        \item \textbf{Matrix multiplication}: $C = AB$ is a matrix $C \in \mathbb{F}^{m \times p}$ with elements $c_{i,j} = \sum_{\ell=1}^{n} a_{i,\ell}b_{\ell,j}$. The number of columns in $A$ must be equal to the number of rows in $B$, that is $A \in \mathbb{F}^{m \times n}$ and $B \in \mathbb{F}^{n \times p}$.
    \end{itemize}
\end{definition}

\begin{example}
    Suppose $\mathbb{F} = \mathbb{R}$. Then, consider 
    \begin{equation*}
        A = \begin{bmatrix}
            1 & 1 & 2 \\
            2 & 2 & 1
        \end{bmatrix} \in \mathbb{R}^{2 \times 3}, \quad B = \begin{bmatrix}
            2 & 1 \\
            1 & 3 \\
            1 & 1
        \end{bmatrix} \in \mathbb{R}^{3 \times 2}
    \end{equation*}

    We cannot add $A$ and $B$ since they have different sizes. However, we can multiply them:
    \begin{equation*}
        AB = \begin{bmatrix}
            5 & 6 \\ 7 & 9
        \end{bmatrix}, \quad BA = \begin{bmatrix}
            4 & 4 & 5 \\
            7 & 7 & 5 \\
            3 & 3 & 3
        \end{bmatrix}
    \end{equation*}

    To see why, for example, the upper left element of $AB$ is $5$, we can calculate it as $\sum_{\ell=1}^3 a_{1,\ell}b_{\ell,1} = 1 \times 2 + 1 \times 1 + 2 \times 1 = 5$.
\end{example}

\begin{remark}
    Now, we add a very important remark. It just so happens that when working with vectors, we usually assume that they are \textbf{column vectors}. This means that the vector $v = (v_1,v_2,\dots,v_n)$ is represented as a matrix:
    \begin{equation*}
        \mathbf{v} = \begin{bmatrix} v_1 \\ v_2 \\ \vdots \\ v_n \end{bmatrix}
    \end{equation*} 
    This is a common convention in linear algebra, and we will use it in the following sections.
\end{remark}

Sometimes we may derive some rows or columns of the matrix from the others. There is a very important notion related to this.

\begin{definition}
    The \textbf{rank} of a matrix $A \in \mathbb{F}^{m \times n}$ is the maximum number of linearly independent rows or columns in $A$. This is also known as the \textbf{row rank} or \textbf{column rank} of the matrix.
\end{definition}

\begin{remark}
    The row rank and column rank of a matrix are always equal.
\end{remark}

\begin{example}
    Consider the matrix $A = \begin{bmatrix} 1 & 2 \\ 3 & 4 \end{bmatrix}$. To
    determine the rank of $A$, we can perform certain linear rows (or columns)
    permutations to show that the rows (or columns) are linearly independent
    vectors. For example, we can perform the following operations:
    \begin{equation*}
        \begin{bmatrix} 
            1 & 2 \\ 
            3 & 4 
        \end{bmatrix} 
        \xrightarrow{R_2 \gets R_2 - 3R_1} 
        \begin{bmatrix} 
            1 & 2 \\ 
            0 & -2 
        \end{bmatrix} 
        \xrightarrow{R_2 \gets - \frac{1}{2}R_2} 
        \begin{bmatrix} 
            1 & 2 \\ 
            0 & 1 
        \end{bmatrix} 
        \xrightarrow{R_1 \gets R_1 - 2R_2} 
        \begin{bmatrix}
            1 & 0 \\
            0 & 1 
        \end{bmatrix}
    \end{equation*}
    Since we can clearly see that the rows (and columns) are linearly independent, the rank of $A$ is $2$.
\end{example}

\begin{example}
    Consider the matrix $B = \begin{bmatrix} 1 & 2 \\ 2 & 4 \end{bmatrix}$. Let's perform the following operations:
    \begin{equation*}
        \begin{bmatrix} 
            1 & 2 \\ 
            2 & 4 
        \end{bmatrix} 
        \xrightarrow{R_2 \gets R_2 - 2R_1} 
        \begin{bmatrix} 
            1 & 2 \\ 
            0 & 0 
        \end{bmatrix}
    \end{equation*}
    Here we can see that the second row is a linear combination of the first row, so the rank of $B$ is $1$.
\end{example}

One important operation we will be frequently working with is the \textbf{transpose} of the matrix. The transpose of a matrix is an operator that flips a matrix over its diagonal, that is, it switches the row and column indices of the matrix by producing another matrix denoted as $A^{\top}$.

\begin{definition}[Transposition]
    Given a matrix $A \in \mathbb{F}^{m \times n}$, the \textbf{transpose} of $A$ is a matrix $A^{\top} \in \mathbb{F}^{n \times m}$ with elements $A^{\top}_{ij} = A_{ji}$.
\end{definition}

\begin{example}
    For example, consider the square matrix $A = \begin{bmatrix} 1 & 2 \\ 3 & 4 \end{bmatrix}$. Then, the transpose of $A$ is $A^{\top} = \begin{bmatrix} 1 & 3 \\ 2 & 4 \end{bmatrix}$. However, we can transpose any matrix, for example, the matrix $B = \begin{bmatrix} 1 & 2 & 3 \\ 4 & 5 & 6 \end{bmatrix}$ has the transpose $B^{\top} = \begin{bmatrix} 1 & 4 \\ 2 & 5 \\ 3 & 6 \end{bmatrix}$. Finally, what is probably very important to us, the column vector $\mathbf{v} = \begin{bmatrix} 1 \\ 2 \\ 3 \end{bmatrix}$ has the transpose $\mathbf{v}^{\top} = [1,2,3]$.
\end{example}

Finally, is just happens that we can construct matrix from the vectors. Therefore, let us introduce the corresponding notation.
\begin{definition}[Composing Matrix from vectors]
    Suppose we are given $n$ vectors $\mathbf{v}_1,\mathbf{v}_2,\dots,\mathbf{v}_n \in \mathbb{F}^m$. Then, we might define matrix $A$ as a matrix with columns $\mathbf{v}_1,\mathbf{v}_2,\dots,\mathbf{v}_n$ as follows:
    \begin{equation*}
        A = \begin{bmatrix}
            \mathbf{v}_1 & \mathbf{v}_2 & \dots & \mathbf{v}_n
        \end{bmatrix} = \begin{bmatrix}
            v_{1,1} & v_{2,1} & \dots & v_{n,1} \\
            v_{1,2} & v_{2,2} & \dots & v_{n,2} \\
            \vdots & \vdots & \ddots & \vdots \\
            v_{1,m} & v_{2,m} & \dots & v_{n,m}
        \end{bmatrix}
    \end{equation*}

    Alternatively, vectors might be represented as rows, and the matrix $A$ might be defined as a matrix with rows $\mathbf{v}_1,\mathbf{v}_2,\dots,\mathbf{v}_n$:
    \begin{equation*}
        A = \begin{bmatrix}
            \mathbf{v}_1^{\top} \\ \mathbf{v}_2^{\top} \\ \vdots \\ \mathbf{v}_n^{\top}
        \end{bmatrix} = \begin{bmatrix}
            v_{1,1} & v_{1,2} & \dots & v_{1,m} \\
            v_{2,1} & v_{2,2} & \dots & v_{2,m} \\
            \vdots & \vdots & \ddots & \vdots \\
            v_{n,1} & v_{n,2} & \dots & v_{n,m}
        \end{bmatrix}
    \end{equation*}
\end{definition}

\begin{example}
    For example, consider the vectors $\mathbf{v}_1 = (1,2,3)$ and $\mathbf{v}_2 = (4,5,6)$. Then, the matrix $A$ with columns $\mathbf{v}_1$ and $\mathbf{v}_2$ is:
    \begin{equation*}
        A = \begin{bmatrix}
            \mathbf{v}_1 & \mathbf{v}_2
        \end{bmatrix} = \begin{bmatrix}
            1 & 4 \\
            2 & 5 \\
            3 & 6
        \end{bmatrix}
    \end{equation*}

    Similarly, the matrix $B$ with rows $\mathbf{v}_1$ and $\mathbf{v}_2$ is:
    \begin{equation*}
        B = \begin{bmatrix}
            \mathbf{v}_1^{\top} \\ \mathbf{v}_2^{\top}
        \end{bmatrix} = \begin{bmatrix}
            1 & 2 & 3 \\
            4 & 5 & 6
        \end{bmatrix}
    \end{equation*}
\end{example}

\subsubsection{Inner Product}

\begin{definition}
    Consider the vector space $\mathbb{F}^n$. The \textbf{inner product} is a function $\langle \cdot, \cdot \rangle: \mathbb{F}^n \times \mathbb{F}^n \to \mathbb{F}$ satisfying the following conditions for all $\mathbf{u},\mathbf{v},\mathbf{w} \in \mathbb{F}^n$:
    \begin{itemize}
        \item $\langle \mathbf{u} + \mathbf{v}, \mathbf{w} \rangle = \langle \mathbf{u}, \mathbf{w} \rangle + \langle \mathbf{v}, \mathbf{w} \rangle$.
        \item $\langle \mathbf{u}, \mathbf{v} + \mathbf{w} \rangle = \langle \mathbf{u}, \mathbf{v} \rangle + \langle \mathbf{u}, \mathbf{w} \rangle$.
        \item $\langle \mathbf{u}, \mathbf{v} \rangle = 0$ for all $\mathbf{u} \in \mathbb{F}^n$ iff $\mathbf{v} = \mathbf{0}$.
        \item $\langle \mathbf{u}, \mathbf{v} \rangle = 0$ for all $\mathbf{v} \in \mathbb{F}^n$ iff $\mathbf{u} = \mathbf{0}$.
    \end{itemize}
\end{definition}

Plenty of functions can be built that satisfy the inner product definition, we will use the one that
is usually called \textbf{dot product}.
\begin{definition}
    Consider the vector space $\mathbb{F}^n$. The \textbf{dot product} on $\mathbb{F}^n$ is a function $\langle \cdot, \cdot \rangle: \mathbb{V} \times \mathbb{V} \rightarrow \mathbb{F}$, 
    defined for every $\mathbf{u}, \mathbf{v} \in \mathbb{F}^n$ as follows:
    \begin{equation*}
        \langle \mathbf{u}, \mathbf{v} \rangle := \mathbf{u}^{\top}\mathbf{v} = \sum_{i=1}^{n} u_i v_i
    \end{equation*}

    Alternatively, the dot product can also be denoted using the dot notation as $\mathbf{u} \cdot \mathbf{v}$. That is why it is called the ``dot`` product.
\end{definition}

\begin{example}
    Let $\mathbf{u}, \mathbf{v}$ are vectors over the real number $\mathbb{R}$, where
    \begin{equation*}
        \mathbf{u} = (1, 2, 3), \quad \mathbf{v} = (2, 4, 3)
    \end{equation*}
    Then: 
    \begin{equation*}
        \langle \mathbf{u}, \mathbf{v}\rangle = \sum_{i=1}^{3}u_iv_i = 2 \cdot 1 + 2 \cdot 4 + 3 \cdot 3 = 2 + 8 + 9 = 19
    \end{equation*}
\end{example}

\subsubsection{Hadamard Product}

Yet another product we are going to use is the \textbf{Hadamard product}. Let us see how it works.
\begin{definition}
    Suppose $A,B \in \mathbb{F}^{m \times n}$. The \textbf{Hadamard product} $A \odot B$ gives a matrix $C$ such that $C_{i,j} = A_{i,j}B_{i,j}$. Essentially, we multiply elements elementwise.
\end{definition}

\begin{example}
    Consider $A = \begin{bmatrix}
        1 & 1 & 2 \\
        3 & 0 & 3
    \end{bmatrix}, B = \begin{bmatrix}
        3 & 2 & 1 \\ 0 & 2 & 1
    \end{bmatrix}$. Then, the Hadamard product is:
    \begin{equation*}
        A \odot B = \begin{bmatrix}
            1 \cdot 3 & 1 \cdot 2 & 2 \cdot 1 \\
            3 \cdot 0 & 0 \cdot 2 & 3 \cdot 1
        \end{bmatrix} = \begin{bmatrix}
            3 & 2 & 2 \\
            0 & 0 & 3
        \end{bmatrix}
    \end{equation*}
\end{example}

\subsubsection{Outer Product}

The final product we want to introduce is the \textbf{outer product} and some of its properties.

\begin{definition}
    Given two vectors $\mathbf{u} \in \mathbb{F}^n$, $\mathbf{v} \in \mathbb{F}^m$ the \textbf{outer product} is a
    matrix whose entries are all products of an element in the first vector with an element 
    in the second vector:
    \begin{equation*}
        \mathbf{u} \otimes \mathbf{v} := \mathbf{u}\mathbf{v}^{\top} = \begin{bmatrix}
            u_1 v_1 & u_1 v_2 & \cdots & u_1 v_n \\
            u_2 v_1 & u_2 v_2 & \cdots & u_2 v_n \\
            \vdots & \vdots & \ddots & \vdots \\
            u_m v_1 & u_m v_2 & \cdots & u_m v_n
        \end{bmatrix}
    \end{equation*}
\end{definition}

\begin{lemma}[Properties of outer product]
    For any scalar $c \in \mathbb{F}$ and $(\mathbf{u}, \mathbf{v}, \mathbf{w}) \in \mathbb{F}^n \times \mathbb{F}^m \times \mathbb{F}^p$:
    \begin{itemize}
        \item Transpose: $(\mathbf{u} \otimes \mathbf{v}) = (\mathbf{v} \otimes \mathbf{u})^{\textsf{T}}$
        \item Distributivity: $\mathbf{u} \otimes (\mathbf{v} + \mathbf{w}) = \mathbf{u} \otimes \mathbf{v} + \mathbf{u} \otimes \mathbf{w}$
        \item Scalar Multiplication: $c(\mathbf{v} \otimes \mathbf{u}) = (c\mathbf{v}) \otimes \mathbf{u} = \mathbf{v} \otimes (c\mathbf{u})$
        \item Rank: the outer product $\mathbf{u} \otimes \mathbf{v}$ is a rank-1 matrix if $\mathbf{u}$ and $\mathbf{v}$ are non-zero
        vectors
    \end{itemize}
\end{lemma}

\begin{example}
    Let $\mathbf{u}, \mathbf{v}$ are vectors over the real number $\mathbb{R}$, where
    \begin{equation*}
        \mathbf{u} = (1, 2, 3), \quad \mathbf{v} = (2, 4, 3)
    \end{equation*}
    Then: 
    \begin{equation*}
        \mathbf{u} \otimes \mathbf{v} = \mathbf{u}\mathbf{v}^{\top} = \begin{bmatrix}
        1 \\ 2 \\ 3
        \end{bmatrix}\begin{bmatrix}
        2 & 4 & 3
        \end{bmatrix} = \begin{bmatrix}
            1 \cdot 2 & 1 \cdot 4 & 1 \cdot 3 \\
            2 \cdot 2 & 2 \cdot 4 & 2 \cdot 3 \\
            3 \cdot 2 & 3 \cdot 4 & 3 \cdot 3
        \end{bmatrix} = \begin{bmatrix}
            2 & 4 & 3 \\
            4 & 8 & 6 \\
            6 & 12 & 9
        \end{bmatrix}
    \end{equation*}
    Additionally, as we can see the rows number 2 and 3 in the result matrix can be represented
    as a linear combination of the first row, specifically by multiplying it by 2 and 3, 
    respectively. The same property applies to the columns. This demonstrates the property of the
    outer product, that the resulting matrix has a rank of 1.
\end{example}

\end{document}

