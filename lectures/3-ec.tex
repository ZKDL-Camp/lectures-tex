\documentclass[../lecture-notes.tex]{subfiles}

\begin{document}

\subsection{Finite Field Extensions}

\subsubsection{General Definition}

Previously, our discussion resolved around the finite field $\mathbb{F}_p$ for a prime $p$. However, many protocols
need more than just a prime field. For example, elliptic curve pairings and certain STARK constructions require
extending $\mathbb{F}_p$ to, in a sense, the analogous of complex numbers. 

From school and, possibly, university, you might remember how complex numbers $\mathbb{C}$ are constructed. You take two real numbers, say, $x,y \in \mathbb{R}$, introduce a new
symbol $i$ satisfying $i^2 = -1$, and define the complex number as $z = x + iy$. In certain cases, one might encounter a bit more rigorous and abstract definition of complex numbers as
the set of pairs $(x,y) \in \mathbb{R}^2$ where addition in naturally defined as $(x_1,y_1)+(x_2,y_2)=(x_1+x_2,y_1+y_2)$, and the multiplication is:
\begin{equation}
    (x_1,y_1) \cdot (x_2,y_2) = (x_1x_2-y_1y_2,x_1y_2+x_2y_1)\footnote{Notice that $(x_1+iy_1)(x_2+iy_2) = x_1x_2+iy_2x_1+iy_1x_2+i^2y_1y_2 = (x_1x_2-y_1y_2) + (x_1y_2+x_2y_1)i$.}.
\end{equation}

In spite of what interpretation you have seen, the complex number is just a tuple of two real numbers that satisfy a bit different rules of multiplication (since addition is typically defined in the same way). What is even more important to us, is that $\mathbb{C}$ is our first example of the so-called \textbf{field extension} of $\mathbb{R}$.

Formally, definition of the field extension is very straightforward:
\begin{definition}
    Let $\mathbb{F}$ be a field and $\mathbb{K}$ be another field. We say that $\mathbb{K}$ is an \textbf{extension} of $\mathbb{F}$ if $\mathbb{F} \subset \mathbb{K}$ and we denote it as $\mathbb{K}/\mathbb{F}$.
\end{definition}

Despite just a simplicity of the definition, the field extensions are a very powerful tool in mathematics. But first, let us consider a few examples of field extensions.

\begin{example}
    Denote by $\mathbb{Q}(\sqrt{2}) = \{x + y\sqrt{2}: x,y \in \mathbb{Q}\}$. This is a field extension of $\mathbb{Q}$. It is obvious that $\mathbb{Q} \subset \mathbb{Q}(\sqrt{2})$, but why is $\mathbb{Q}(\sqrt{2})$ a field? Addition and multiplication operations are obviously closed:
    \begin{equation}
        \begin{aligned}
            (x_1+y_1\sqrt{2}) + (x_2+y_2\sqrt{2}) = (x_1+x_2) + (y_1+y_2)\sqrt{2},\\
            (x_1+y_1\sqrt{2}) \cdot (x_2+y_2\sqrt{2}) = (x_1x_2+2y_1y_2) + (x_1y_2+x_2y_1)\sqrt{2}.
        \end{aligned}
    \end{equation}
    But what about the inverse element? Well, here is the trick:
    \begin{equation}
        \frac{1}{x+y\sqrt{2}} = \frac{x-y\sqrt{2}}{(x+y\sqrt{2})(x-y\sqrt{2})} = \frac{x-y\sqrt{2}}{x^2-2y^2} = \frac{x}{x^2-2y^2} - \frac{y}{x^2-2y^2}\sqrt{2} \in \mathbb{Q}(\sqrt{2}).
    \end{equation}
\end{example}

\begin{example}
    Consider $\mathbb{Q}(\sqrt{2}, i) = \{a+bi: a,b \in \mathbb{Q}(\sqrt{2})\}$ where $i^2=-1$. This is a field extension of $\mathbb{Q}(\sqrt{2})$ and, consequently, of $\mathbb{Q}$. The representation of the element is:
    \begin{equation}
        (a+b\sqrt{2}) + (c+d\sqrt{2})i = a + b\sqrt{2} + ci + d\sqrt{2}i
    \end{equation}

    Showing that this is a field is a bit more tedious, but still straightforward. Suppose we take $\alpha+i\beta \in \mathbb{Q}(\sqrt{2}, i)$ with $\alpha,\beta \in \mathbb{Q}(\sqrt{2})$. Then:
    \begin{equation}
        \frac{1}{\alpha+\beta i} = \frac{\alpha-\beta i}{\alpha^2+\beta^2} = \frac{\alpha}{\alpha^2+\beta^2} - \frac{\beta}{\alpha^2+\beta^2}i
    \end{equation}

    Since $\mathbb{Q}(\sqrt{2})$ is a field, both $\frac{\alpha}{\alpha^2+\beta^2}$ and $\frac{\beta}{\alpha^2+\beta^2}$ are in $\mathbb{Q}(\sqrt{2})$, and, consequently, $\mathbb{Q}(\sqrt{2}, i)$ is a field as well.
\end{example}

\begin{remark}
    Notice that basically, $\mathbb{Q}(\sqrt{2}, i)$ is just a linear combination of $\{1,\sqrt{2},i,\sqrt{2}i\}$. This has a very important implication: $\mathbb{Q}(\sqrt{2}, i)$ is a four-dimensional vector space over $\mathbb{Q}$, where elements $\{1,\sqrt{2},i,\sqrt{2}i\}$ naturally form \textbf{basis}. We are not going to use it implicitly, but this observation might make further discussion a bit more intuitive.
\end{remark}

\subsubsection{Polynomial Quotient Ring}

Now, we present a more general way to construct field extensions. Notice that when constructing $\mathbb{C}$, we used the magical element $i$ that satisfies $i^2=-1$. But here is another way how to think of it.

Consider the set of polynomials $\mathbb{R}[x]$, then I pick $p(x):=x^2+1 \in \mathbb{R}[x]$ and ask you to find roots of $p(x)$. Of course, you would claim ``hey, this equation has no solutions over $\mathbb{R}$'' and that is totally true. That is why mathematicians introduced a new element $i$ that we formally called the root of $x^2+1$. Note however, that $i$ is not a number in the traditional sense, but rather a fictional symbol that we artifically introduced to satisfy the equation.

Now, could we have picked another polynomial, say, $q(x) = x^2+4$? Sure! As long as its roots cannot be found in $\mathbb{R}$, we are good to go.

\begin{example}
    Suppose $\beta$ is the root of $q(x):=x^2+4$. Then we could have defined complex numbers as a set of $x+y\beta$ for $x,y \in \mathbb{R}$. In this case, multiplication, for example, would be defined a bit differently than in the case of $\mathbb{C}$:
    \begin{equation}
        (x_1+y_1\beta) \cdot (x_2+y_2\beta) = (x_1x_2+4y_1y_2) + (x_1y_2+x_2y_1)\beta.
    \end{equation}
\end{example}

We shifted to the polynomial consideration for a reason: now, instead of considering the complex number $\mathbb{C}$ as ``some'' tuple of real numbers $(c_0,c_1)$, now let us view it as a polynomial\footnote{Here, we use $X$ to represent the polynomial variable to avoid confusion with the notation $x+iy$.} $c_0+c_1X$ modulo polynomial $X^2+1$. Indeed, take, for example, $p_1(X) := 1+2X$ and $p_2(X) := 2+3X$. Addition is performed as we are used to:
\begin{equation}
    p_1 + p_2 = (1+2X) + (2+3X) = 3+5X,
\end{equation}
but multiplication is a bit different:
\begin{equation}
    p_1p_2 = (1+2X) \cdot (2+3X) = 2+3X+4X+6X^2 = 6X^2+7X+2.
\end{equation}

Well, and what next? Recall that we are doing arithmetic modulo $X^2+1$ and for that reason, we divide the polynomial by $X^2+1$:
\begin{equation}
    6X^2+7X+2 = 6(X^2+1) + 7X - 4 \implies (6X^2+7X+2) \,\text{mod}\, (X^2+1) = 7X-4,
\end{equation}
meaning that $p_1p_2 = 7X-4$. Oh wow, hold on! Let us come back to our regular complex number representation and multiply $(1+2i)(2+3i)$. We get $2+3i+4i+6i^2=-4+7i$. That is exactly the same result if we change $X$ to $i$ above! In fact, what we have observed is the fact that our polynomial quotient ring $\mathbb{R}[X]/(X^2+1)$ is isomorphic to $\mathbb{C}$.

\begin{theorem}
    Let $\mathbb{F}$ be a field and $\mu(x)$ --- irreducible polynomial over $\mathbb{F}$. Consider a set of polynomials over $\mathbb{F}[x]$ modulo $\mu(x)$, formally denoted as $\mathbb{F}[x]/(\mu(x))$. Then, $\mathbb{F}[x]/(\mu(x))$ is a field.
\end{theorem}

\begin{example}
    As we considered above, let $\mathbb{F}=\mathbb{R}$, $\mu(x)=x^2+1$, then $\mathbb{R}[X]/(X^2+1)$ (a set of polynomials modulo $X^2+1$) is a field.
\end{example}
\begin{example}
    Suppose $\mathbb{F}=\mathbb{Q}$ and $\mu(x) := x^2-2$. Then, $\mathbb{Q}[X]/(X^2-2)$ is a field isomorphic to $\mathbb{Q}(\sqrt{2})$, considered above.
\end{example}
\begin{example}
    Suppose $\mathbb{F}=\mathbb{Q}$ and $\mu(x) := (x^2+1)(x^2-2)=x^4-x^2-2$. Then, $\mathbb{Q}[X]/(x^4-x^2-2)$ is a field isomorphic to $\mathbb{Q}(\sqrt{2},i)$.
\end{example}

This theorem (aka definition) corresponds to viewing complex numbers as a polynomial quotient ring $\mathbb{R}[X]/(X^2+1)$. But, we can give a theorem (aka definition) for our classical representation via magical root $i$ of $x^2+1$.

\begin{theorem}
    Let $\mathbb{F}$ be a field and $\mu(x) \in \mathbb{F}[x]$ is an irreducible polynomial of degree $n$ and let $\mathbb{K} := \mathbb{F}[x]/(\mu(x))$. 
\end{theorem}

\end{document}