\documentclass[../lecture-notes.tex]{subfiles}

\begin{document}

\subsection{Exercises 1}

\textbf{Exercise 1.} Which of the following statements is \textbf{false}?
\begin{enumerate}
    \item $(\forall a, b \in \mathbb{Q}, a \neq b) \; (\exists q \in \mathbb{R}): \{a < q < b\}$.
    \item $(\forall \varepsilon > 0) \, (\exists n_{\varepsilon} \in \mathbb{N}) \, (\forall n \geq n_{\varepsilon}): \{1/n < \varepsilon\}$.
    \item $(\forall k \in \mathbb{Z}) \; (\exists n \in \mathbb{N}): \{n < k\}$.
    \item $(\forall x \in \mathbb{Z} \setminus \{-1\}) \; (\exists! y \in \mathbb{Q}): \{(x+1)y = 2\}$.
\end{enumerate}

\textbf{Exercise 2.} Denote $X := \{(x,y) \in \mathbb{Q}^2: xy = 1\}$. Oleksandr claims the following:
\begin{enumerate}
    \item $X \cap \mathbb{N}^2 = \{(1,1)\}$. 
    \item $|X \cap \mathbb{Z}^2| = 2|X \cap \mathbb{N}^2|$.
    \item $X$ is a group under the operation $(x_1,y_1) \oplus (x_2,y_2) = (x_1x_2, y_1y_2)$.
\end{enumerate}

Which statements are \textbf{true}?
\begin{multicols}{3}
    \begin{enumerate}[a)]
        \item Only 1.
        \item Only 1 and 2.
        \item Only 1 and 3.
        \item Only 2 and 3.
        \item All statements are correct.
    \end{enumerate}
\end{multicols}

\textbf{Exercise 3.} Does a tuple $(\mathbb{Z},\oplus)$ with operation $a \oplus b = a + b - 1$ define a group?
\begin{enumerate}[a)]
    \item Yes, and this group is abelian.
    \item Yes, but this group is not abelian.
    \item No, since the associativity property does not hold.
    \item No, since there is no identity element in this group.
    \item No, since there is no inverse element in this group.
\end{enumerate}

\textbf{Exercise 4.} Consider the Cartesian plane $\mathbb{R}^2$, where two coordinates are real numbers. For two points $A,B$ define the operation $\oplus$
as follows: $A \oplus B$ is the midpoint on segment $AB$. Does $(\mathbb{R}^2, \oplus)$ define a group? 
\begin{enumerate}[a)]
    \item Yes, and this group is abelian.
    \item Yes, but this group is not abelian.
    \item No, since the associativity property does not hold and there is no identity element in this group.
    \item No, since the associativity property does not hold, but we might define an identity element nonetheless.
\end{enumerate}

\textbf{Exercise 5.} Find the inverse of $4$ in $\mathbb{F}_{11}$.
\begin{multicols}{4}
    \begin{enumerate}[a)]
        \item 8
        \item 5
        \item 3
        \item 7
    \end{enumerate}
\end{multicols}

\newpage
\textbf{Exercise 6.} Suppose for three polynomials $p,q,r \in \mathbb{F}[x]$ we have $\deg p = 3, \deg q = 4, \deg r = 5$. Which of the following is true for $n := \deg \{(p-q)r\}$?
\begin{multicols}{2}
    \begin{enumerate}[a)]
        \item $n = 9$.
        \item $n$ might be less than $9$.
        \item $n = 20$.
        \item $n$ is less than $\deg \{qr\}$. 
    \end{enumerate}
\end{multicols}

\textbf{Exercise 7.} Define the polynomial over $\mathbb{F}_5$: $f(x) := 4x^2 + 7$. Which of the following is the root of $f(x)$?
\begin{multicols}{4}
    \begin{enumerate}[a)]
        \item $2$
        \item $3$
        \item $4$
        \item No roots.
    \end{enumerate}
\end{multicols}

\textbf{Exercise 8.} Quadratic polynomial $p(x) = ax^2+bx+c \in \mathbb{R}[x]$ has zeros at $1$ and $2$ and $p(0) = 2$. Find the value of $a+b+c$.
\begin{multicols}{2}
    \begin{enumerate}[a)]
        \item $0$
        \item $-1$
        \item $1$
        \item Not enough information.
    \end{enumerate}
\end{multicols}

\textbf{Exercise 9.} Which of the following is a \textbf{valid} endomorphism $f: X \to X$? 
\begin{multicols}{2}
    \begin{enumerate}[a)]
        \item $X = [0,1]$, $f: x \mapsto x^2$.
        \item $X = [0,1]$, $f: x \mapsto x + 1$.
        \item $X = \mathbb{R}_{>0}$, $f: x \mapsto (x-1)^3$.
        \item $X = \mathbb{Q}_{>0}$, $f: x \mapsto \sqrt{x}$.
    \end{enumerate}
\end{multicols}

\textbf{Exercise 10*.} Denote by $\text{GL}(2,\mathbb{R})$ a set of $2\times 2$ invertable matrices with real entries. Define two functions $\varphi: \text{GL}(2,\mathbb{R}) \to \mathbb{R}$:
\begin{equation*}
    \varphi_1 \left(\begin{bmatrix}
        a & b \\ c & d
    \end{bmatrix}\right) = ad - bc, \; \varphi_2 \left(\begin{bmatrix}
        a & b \\ c & d
    \end{bmatrix}\right) = a + d
\end{equation*}

Den claims the following:
\begin{enumerate}
    \item $\varphi_1$ is a group homomorphism between multiplicative groups $(\text{GL}(2,\mathbb{R}), \times)$ and $(\mathbb{R}, \times)$.
    \item $\varphi_2$ is a group homomorphism between additive groups $(\text{GL}(2, \mathbb{R}), +)$ and $(\mathbb{R}, +)$.
\end{enumerate}

Which of the following is \textbf{true}?

\begin{enumerate}[a)]
    \item Only statement 1 is correct.
    \item Only statement 2 is correct.
    \item Both statements 1 and 2 are correct.
    \item None of the statements is correct.
\end{enumerate}

\subsection{Exercises 2}

\textbf{Exercise 1.} Suppose that for the given cipher with a security parameter $\lambda$, the adversary $\mathcal{A}$ can deduce the least significant bit of the plaintext from the ciphertext. Recall that the advantage 
of a bit-guessing game is defined as $\text{SS}\mathsf{Adv}[\mathcal{A}] = \left|\Pr[b=\hat{b}] - \frac{1}{2}\right|$, where $b$ is the randomly chosen bit of a challenger, while 
$\hat{b}$ is the adversary's guess. What is the maximal advantage of $\mathcal{A}$ in this case?

\textbf{Hint:} The adversary can choose which messages to send to challenger to further distinguish the plaintexts.
\begin{multicols}{3}
    \begin{enumerate}[a)]
        \item $1$
        \item $\frac{1}{2}$
        \item $\frac{1}{4}$
        \item $0$
        \item Negligible value ($\text{negl}(\lambda)$).
    \end{enumerate}
\end{multicols}

\textbf{Exercise 2.} Consider the cipher $\mathcal{E} = (E,D)$ with encryption function $E: \mathcal{K} \times \mathcal{M} \to \mathcal{C}$ over the message space $\mathcal{M}$, ciphertext space $\mathcal{C}$, and key space $\mathcal{K}$. We want to define the security
that, based on the cipher, the adversary $\mathcal{A}$ cannot restore the message (\textit{security against message recovery}). For that reason, we define the following game:
\begin{enumerate}
    \item Challenger chooses random $m \xleftarrow{R} \mathcal{M}, k \xleftarrow{R} \mathcal{K}$.
    \item Challenger computes the ciphertext $c \gets E(k,m)$ and sends to $\mathcal{A}$.
    \item Adversary outputs $\hat{m}$, and wins if $\hat{m} = m$.
\end{enumerate}

We say that the cipher $\mathcal{E}$ is secure against message recovery if the \textbf{message recovery advantage}, denoted as $\text{MR}\textsf{adv}[\mathcal{A}, \mathcal{E}]$ is negligible. Which of the following statements is a valid interpretation of the message recovery advantage?
\begin{enumerate}[a)]
    \item $\text{MR}\textsf{adv}[\mathcal{A},\mathcal{E}] := \left|\text{Pr}[m=\hat{m}] - \frac{1}{2}\right|$
    \item $\text{MR}\textsf{adv}[\mathcal{A},\mathcal{E}] := \left|\text{Pr}[m=\hat{m}] - 1\right|$.
    \item $\text{MR}\textsf{adv}[\mathcal{A},\mathcal{E}] := \text{Pr}[m=\hat{m}]$
    \item $\text{MR}\textsf{adv}[\mathcal{A},\mathcal{E}] := \left|\text{Pr}[m=\hat{m}] - \frac{1}{|\mathcal{M}|}\right|$
\end{enumerate}

\textbf{Exercise 3.} Suppose that $f$ and $g$ are negligible functions. Which of the following functions is not neccessarily negligible?
\begin{enumerate}[a)]
    \item $f + g$
    \item $f \times g$
    \item $f - g$
    \item $f/g$
    \item $h(\lambda) := \begin{cases}
        1/f(\lambda) & \text{if } 0 < \lambda < 100000 \\
        g(\lambda) & \text{if } \lambda \geq 100000
    \end{cases}$
\end{enumerate}

\newpage
\textbf{Exercise 4.} Suppose that $f \in \mathbb{F}_p[x]$ is a $d$-degree polynomial with $d$ \textbf{distinct} roots in $\mathbb{F}_p$. What is the probability that, when evaluating $f$ at $n$ random points, the polynomial will be zero at all of them?
\begin{multicols}{2}
    \begin{enumerate}[a)]
        \item Exactly $(d/p)^n$.
        \item Strictly less that $(d/p)^n$.
        \item Exactly $nd/p$.
        \item Exactly $d/np$.
    \end{enumerate}
\end{multicols}

\textbf{Exercise 5-6.} To demonstrate the idea of Reed-Solomon codes, consider the toy construction. Suppose that our message is a tuple of two elements $a,b \in \mathbb{F}_{13}$. Consider function $f: \mathbb{F}_{13} \to \mathbb{F}_{13}$, defined as $f(x) = ax+b$, and define the encoding of the message $(a,b)$ as $(a,b) \mapsto (f(0),f(1),f(2),f(3))$. 

\textbf{Question 5.} Suppose that you received the encoded message $(3,5,6,9)$. Which number from the encoded message is corrupted?
\begin{multicols}{2}
    \begin{enumerate}[a)]
        \item First element ($3$).
        \item Second element ($5$).
        \item Third element ($6$).
        \item Fourth element ($9$).
        \item The message is not corrupted.
    \end{enumerate}
\end{multicols}

\textbf{Question 6.} Consider the previous question. Suppose that the original message was $(a,b)$. Find the value of $a \times b$ (in $\mathbb{F}_{13}$).
\begin{multicols}{5}
    \begin{enumerate}[a)]
        \item $4$
        \item $6$
        \item $12$
        \item $2$
        \item $1$
    \end{enumerate}
\end{multicols}

\end{document}