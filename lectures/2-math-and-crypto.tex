\documentclass[../lecture-notes.tex]{subfiles}

\begin{document}

\subsection{Basics of Cryptoanalysis}

In many cases, technical papers include the analysis on the key question: ``How secure is this cryptographic algorithm?'' or rather ``Why this cryptographic algorithm is secure?''. In this section, we will shortly describe the notation and typical construction for justifying the security of cryptographic algorithms.

Typically, the cryptographic security is defined in a form of a game between the adversary (who we call $\mathcal{A}$) and the challenger (who we call $\mathcal{C}h$). The adversary is trying to break the security of the cryptographic algorithm using arbitrary (but still efficient) protocol, while the challenger is following a simple, fixed protocol. The game is played in a form of a challenge, where the adversary is given some information and is asked to perform some task. The security of the cryptographic algorithm is defined based on the probability of the adversary to win the game.

\subsubsection{Cipher Semantic Security}
Let us get into specifics. Suppose that we want to specify that the encryption scheme is secure. Recall that cipher $\mathcal{E} = (E,D)$ over the space $(\mathcal{K}, \mathcal{M}, \mathcal{C})$ (here, $\mathcal{K}$ is the space containing all possible keys, $\mathcal{M}$ -- all possible messages and $\mathcal{C}$ -- all possible ciphers) consists of two efficiently computable methods:
\begin{itemize}
    \item $E: \mathcal{K} \times \mathcal{M} \to \mathcal{C}$ -- encryption method, that based on the provided message $m \in \mathcal{M}$ and key $k \in \mathcal{K}$ outputs the cipher $c = E(k,m) \in \mathcal{C}$.
    \item $D: \mathcal{K} \times \mathcal{C} \to \mathcal{M}$ -- decryption method, that based on the provided cipher $c \in \mathcal{C}$ and key $k \in \mathcal{K}$ outputs the message $m = D(k,c) \in \mathcal{M}$.
\end{itemize}

Of course, we require the \textbf{correctness}:
\begin{equation}
    (\forall k \in \mathcal{K}) \, (\forall m \in \mathcal{M}): \{D(k,E(k,m)) = m\}
\end{equation}

Now let us play the following game between adversary $\mathcal{A}$ and challenger $\mathcal{C}h$:
\begin{enumerate}
    \item $\mathcal{A}$ picks any two messages $m_0,m_1 \in \mathcal{M}$ on his choice.
    \item $\mathcal{C}h$ picks a random key $k \xleftarrow{R} \mathcal{K}$ and random bit $b \xleftarrow{R} \{0,1\}$ and sends the cipher $c = E(k,m_b)$ to $\mathcal{A}$.
    \item $\mathcal{A}$ is trying to guess the bit $b$ by using the cipher $c$.
    \item $\mathcal{A}$ outputs the guess $\hat{b}$.
\end{enumerate}

\begin{figure}
    \centering
    \begin{tikzpicture}
        \node[inner sep=0pt, align=center] (challenger) {\includegraphics[width=0.75cm]{lectures/images/lecture_2/challenger.png}\\Challenger $\mathcal{C}h$};
        \node[inner sep=0pt, align=center, right=5cm of challenger] (adversary) {\includegraphics[width=0.75cm]{lectures/images/lecture_2/demon.png}\\Adversary $\mathcal{A}$};

        \draw [dashed,line width=0.3mm] ([yshift=-0.5cm]challenger.south) -- ([yshift=-9cm]challenger.south);
        \draw [dashed,line width=0.3mm] ([yshift=-0.5cm]adversary.south) -- ([yshift=-9cm]adversary.south);

        \draw[-{Stealth[length=3mm]},line width=0.4mm] ([yshift=-1.5cm]adversary.south) coordinate (l2)--(l2-|challenger) node[midway, above=2mm, fill=white]{Send $m_1, m_2 \in \mathcal{M}, |m_1| = |m_2|$};

        \node[align=center,fill=white!5,thick,below=2.5cm of challenger](challenger-actions){
        \noindent\rule{3.5cm}{0.8pt}\\
        $b \xleftarrow{R} \{0,1\}$ \\
        $k \xleftarrow{R} \mathcal{K}$ \\
        $c \gets E(k,m_b)$ \\
        \noindent\rule{3.5cm}{0.8pt}};

        \draw[-{Stealth[length=3mm]},line width=0.4mm] ([yshift=-6cm]challenger.south) coordinate (l2)--(l2-|adversary) node[midway, above=2mm, fill=white]{Send cipher $c$};

        \node[align=center,fill=white!5,thick,below=7cm of adversary](adversary-guess){
        \noindent\rule{3.5cm}{0.8pt}\\
        Guess bit $\hat{b} \in \{0,1\}$\\
        \noindent\rule{3.5cm}{0.8pt}};
    \end{tikzpicture}

    \caption{The game between the adversary $\mathcal{A}$ and the challenger $\mathcal{C}h$ for defining the semantic security.}
\end{figure}

Now, what should happen if our encryption scheme is secure? The adversary should not be able to guess the bit $b$ with a probability significantly higher than $1/2$ (a random guess). Formally, define the 
\textbf{advantage} of the adversary $\mathcal{A}$ as:
\begin{equation}
    \text{SSAdv}[\mathcal{E}, \mathcal{A}] := \left| \Pr[\hat{b} = b] - \frac{1}{2} \right|
\end{equation}

We say that the encryption scheme is \textbf{semantically secure}\footnote{This version of definition is called a \textbf{bit-guessing} version.} if for any efficient adversary $\mathcal{A}$ the advantage $\text{SSAdv}[\mathcal{A}]$ is negligible. In other words, the adversary cannot guess the bit $b$ with a probability significantly higher than $1/2$.

Now, what negligible means? Let us give the formal definition!

\begin{definition}
    A function $f: \mathbb{N} \to \mathbb{R}$ is called \textbf{negligible} if for all $c \in \mathbb{R}_{>0}$ there exists $n_c \in \mathbb{N}$ such that for any $n \geq n_c$ we have $|f(n)| < 1/n^c$.
\end{definition}

The alternative definition, which is problably easier to interpret, is the following.

\begin{theorem}
    A function $f: \mathbb{N} \to \mathbb{R}$ is \textbf{negligible} if and only if for any $c \in \mathbb{R}_{>0}$, we have
    \begin{equation}
        \lim_{n \to \infty} f(n)n^c = 0
    \end{equation}
\end{theorem}

\begin{example}
    The function $f(n) = 2^{-n}$ is negligible since for any $c \in \mathbb{R}_{>0}$ we have
    \begin{equation}
        \lim_{n \to \infty} 2^{-n}n^c = 0
    \end{equation}

    The function $g(n) = \frac{1}{n!}$ is also negligible for similar reasons.
\end{example}

\begin{example}
    The function $h(n) = \frac{1}{n}$ is not negligible since for $c = 1$ we have
    \begin{equation}
        \lim_{n \to \infty} \frac{1}{n} \times n = 1 \neq 0
    \end{equation}
\end{example}

Well, that is weird. For some reason we are considering a function the depends on some natural number $n$, but what is this number?

Typically, when defining the security of the cryptographic algorithm, we are considering the security parameter $\lambda$ (e.g., the length of the key). The function is negligible if the probability of the adversary to break the security of the cryptographic algorithm is decreasing with the increasing of the security parameter $\lambda$. Moreover, we require that the probability of the adversary to break the security of the cryptographic algorithm is decreasing faster than any polynomial function of the security parameter $\lambda$.

So all in all, we can define the semantic security as follows.

\begin{definition}
    The encryption scheme $\mathcal{E}$ with a security paramter $\lambda \in \mathbb{N}$ is \textbf{semantically secure} if for any efficient adversary $\mathcal{A}$ we have:
    \begin{equation}
        \left|\text{Pr}\begin{bmatrix}[c|c]
            & m_1, m_2 \gets \mathcal{A}, \; k \xleftarrow{R} \mathcal{K}, \; b \xleftarrow{R} \{0,1\} \\
            b = \hat{b} & c \gets E(k,m_b) \\
            &\hat{b} \gets \mathcal{A}(c)
        \end{bmatrix} - \frac{1}{2}\right| < \text{negl}(\lambda)
    \end{equation}
\end{definition}

Let us see some more examples of how to define the security of certain crypographic objects.

\subsubsection{Discrete Logarithm Assumption}

Now, let us define the fundamental assumption used in cryptography formally: the \textbf{Discrete Logarithm Assumption}.

\begin{definition}
    Assume that $\mathbb{G}$ is a cyclic group of prime order $r$ generated by $g \in \mathbb{G}$. Define the following game:
    \begin{enumerate}
        \item Both challenger $\mathcal{C}h$ and adversary $\mathcal{A}$ take a description $\mathbb{G}$ as an input: order $r$ and generator $g \in \mathbb{G}$.
        \item $\mathcal{C}h$ computes $\alpha \xleftarrow{R} \mathbb{Z}_r, u \gets g^{\alpha}$ and sends $u \in \mathbb{G}$ to $\mathcal{A}$.
        \item The adversary $\mathcal{A}$ outputs $\hat{\alpha} \in \mathbb{Z}_r$.
    \end{enumerate}

    We define $\mathcal{A}$'s \textbf{advantage in solving the discrete logarithm problem in $\mathbb{G}$}, denoted as $\text{DL}\mathsf{adv}[\mathcal{A},\mathbb{G}]$, as the probability that $\hat{\alpha} = \alpha$.
\end{definition}

\begin{definition}
    The \textbf{Discrete Logarithm Assumption} holds in the group $\mathbb{G}$ if for any efficient adversary $\mathcal{A}$ the advantage $\text{DL}\mathsf{adv}[\mathcal{A},\mathbb{G}]$ is negligible.
\end{definition}

Informally, this assumption means that given $u$, it is very hard to find $\alpha$ such that $u = g^{\alpha}$. But now we can write down this formally!

\subsubsection{Computational Diffie-Hellman}

Another fundamental problem in cryptography is the \textbf{Computational Diffie-Hellman} problem. It states that given $g^{\alpha},g^{\beta}$ it is hard to find $g^{\alpha\beta}$. This property is frequently used in the construction of cryptographic protocols such as the Diffie-Hellman key exchange.

Let us define this problem formally.

\begin{definition}
    Let $\mathbb{G}$ be a cyclic group of prime order $r$ generated by $g \in \mathbb{G}$. Define the following game:
    \begin{enumerate}
        \item Both challenger $\mathcal{C}h$ and adversary $\mathcal{A}$ take a description $\mathbb{G}$ as an input: order $r$ and generator $g \in \mathbb{G}$.
        \item $\mathcal{C}h$ computes $\alpha, \beta \xleftarrow{R} \mathbb{Z}_r, u \gets g^{\alpha}, v \gets g^{\beta}, w \gets g^{\alpha\beta}$ and sends $u,v \in \mathbb{G}$ to $\mathcal{A}$.
        \item The adversary $\mathcal{A}$ outputs $\hat{w} \in \mathbb{G}$.
    \end{enumerate}

    We define $\mathcal{A}$'s \textbf{advantage in solving the computational Diffie-Hellman problem in $\mathbb{G}$}, denoted as $\text{CDH}\mathsf{adv}[\mathcal{A},\mathbb{G}]$, as the probability that $\hat{w} = w$.
\end{definition}

\begin{definition}
    The \textbf{Computational Diffie-Hellman Assumption} holds in the group $\mathbb{G}$ if for any efficient adversary $\mathcal{A}$ the advantage $\text{CDH}\mathsf{adv}[\mathcal{A},\mathbb{G}]$ is negligible.
\end{definition}

\subsubsection{Why this is needed?}

Typically, it is impossible to prove the predicate ``for every efficient adversary $\mathcal{A}$ this probability is negligible'' and therefore we need to make assumptions, such as the Discrete Logarithm Assumption or the Computational Diffie-Hellman Assumption. In turn, proving the statement ``if $X$ is secure then $Y$ is also secure'' is manageable and does not require solving any fundamental problems. So, for example,
knowing that the probability of the adversary to break the Diffie-Hellman assumption is negligible, we can prove that the Diffie-Hellman key exchange is secure. 

\subsection{Basic Number Theory}

\subsubsection{Primes}

Primes are often used when doing almost any cryptographic computation. A prime number is a natural number ($\mathbb{Z}^{+}$) that is not a product
of two smaller natural number. In other words, the prime number is divisible only by itself and 1. First primes look like this: 2, 3, 5, 7, 11...

\subsubsection{Deterministic prime tests}

A primality test is deterministic if it outputs $\mathsf{True}$ when the number is a prime and $\mathsf{False}$ when the input is composite with probability 1.
An example of a deterministic prime test is $\mathsf{Trial\_Division\_Test}$. Here is an example implementation in Rust: 

\begin{lstlisting}[language=Rust]
    fn is_prime(n: u32) -> bool {
        let square_root = (n as f64).sqrt() as u32;

        for i in 2.. = square_root {
            if n % i == 0 {
                return false;
            }
        }

        true
    }
\end{lstlisting}

Deterministic tests often lack efficiency. For instance, even with square root optimization, the asymptotic complexity is $O(\sqrt{N})$. 
While further optimizations are possible, they do not change the overall asymptotic complexity.

In cryptography, $N$ can be extremely large — $256$ bits, $512$ bits, or even $6144$ bits. 
An algorithm is impractical when dealing with such large numbers.

\subsubsection{Probabilistic prime tests}

A primality test is probabilistic if it outputs $\mathsf{True}$ when the number is a prime and $\mathsf{False}$ when the input is composite with probability less than 1. Such test is often called a pseudoprimality test.
Fermat Primality and Miller-Rabin Primality Tests are examples of probabilistic primality test. Both of them use the idea of \textbf{Fermat's Little Theorem}:

\begin{theorem}
    Let $p$ be a prime number and $a$ be an integer not divisible by $p$. Then $a ^ {p-1} - 1$ is always divisible by $p$:  $a^{p-1} \equiv 1 \pmod{p}$
\end{theorem}

The key idea behind the Fermat Primality Test is that if for some $a$ not divisible by $n$ we have $a^{n-1} \not\equiv 1 \pmod{n}$ then $n$ is definitely NOT prime.
Athough, with such an approach, we might get a false positive, as you cannot state for sure that $n$ is prime. For example, consider $n = 15$ and $a = 4$.
$4^{15-1} \equiv 1 \pmod{15}$, but $n = 15 = 3 \cdot 5$ is composite. To solve this issue, $a$ is picked many times, decreasing the chances of a false positive.
The probability that a composite number is mistakenly called prime for $k$ iterations is $2^{-k}$ = $\dfrac{1}{2^k}$.

There exists a problem with such an algorithm in the form of \textbf{Carmichael numbers}, which are numbers that are Fermat pseudoprime to all bases. To put it simply,
no matter how many times you check whether the number is prime using this type of primality test, it will always stay positive, even though the number is
composite. The good thing is that Carmichael numbers are pretty rare. The bad thing is that there are infinitely many of them. 

Even though this algorithm is probabilistic (which does not guarantee the correctness of the output) and has a vulnerability in the form of \textit{Carmichael numbers},
it runs with an asymptotic complexity $O(\log^3{n})$. This is much better for large numbers and is often used in cryptography. Here is a pseudocode implementation of this algorithm:

\begin{lstlisting}[language=Python]
# n = number to be tested for primality
# k = number of times the test will be repeated
def is_prime(n, k):
    i = 1
    while i <= k:
        a = rand(2, n - 1)

        if a^(n - 1) != 1 (mod n):
            return False

        i++

    return True
\end{lstlisting}

Miller-Rabin primality test, is a more advanced form of Fermat primality test. The main difference is it is not vulnerable to \textit{Carmichael numbers},
which makes it much better to use in practice.

\end{document}