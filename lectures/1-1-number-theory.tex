\documentclass[../lecture-notes-148x210.tex]{subfiles}

\begin{document}

As you can recall from high school math, typically real-world processes are described
using real numbers, denoted by $\mathbb{R}$. For example, to describe the position or 
the velocity of an object, you would rather use real numbers. 

However, when it comes to working with computers, real numbers become very inconvenient 
to work with. For instance, different programming languages might output different values 
for quite a straightforward operation, such as an addition \textsf{2.00001 + 2.00000}. This becomes a huge problem 
when dealing with cryptography, which must check \textit{precisely} whether two quantities 
are equal.

This is why the cryptography must work with integer-based formats. One of the earliest and 
most fundamental branches of mathematics is number theory, which deals with the properties 
of the integer set $\mathbb{Z}$. Although, as we will see later, we will primarily need 
the notion of \emph{prime/finite field} further, the basic concepts of number theory will 
still be used extensively nonetheless.

\subsection{The Basics}

We start with the most basic definition of number theory --- \emph{divisibility}.

\begin{definition}
    \label{def:divisibility}
    An integer $a$ is divisible by a non-zero integer $b$, denoted $b \mid a$ (or \emph{b is a divisor of a}), if and only if there exists an integer $k \in \mathbb{Z}$ such that $a = k \cdot b$.
\end{definition}

\begin{example} 
    \label{example:divisibility_1}
    Let us consider the following examples,
    \begin{enumerate}
        \item $41 \mid 123$, since $123 = 3 \cdot 41$
        \item $5 \nmid 29$, since there is no integer $k$ such that $29 = k \cdot 5$
    \end{enumerate}
\end{example}

Let us now explore some more basic properties of divisibility. 

\begin{lemma}[Divisibility properties]
    For all $a, b, c \in \mathbb{Z}:$
    \hfill
    \begin{enumerate}
        \item $1 \mid a$ (any number is divisible by $1$)
        \item If $a \neq 0$, then $a \mid a$ (any non-zero integer divides itself).
        \item If $a \neq 0$, then $a \mid 0$ (any non-zero integer divides $0$).
        \item If $b \mid a$ and $c \mid b$, then $c \mid a$ (formally called \emph{transitivity}).
        \item $b \mid a \iff b \cdot c \mid a \cdot c$ for any $c \neq 0$.
        \item If $c \mid a$ and $c \mid b$, then $c \mid (\alpha \cdot a \pm \beta \cdot b)$, $\text{for any } \alpha, \beta \in \mathbb{Z}$
    \end{enumerate}
\end{lemma}

The question is, what happens if a number $a$ is not divisible by the given integer $b$, 
meaning there is no integer $k$ that satisfies the condition $a = b \cdot k$? 
In such cases, the \emph{Division theorem} comes into play.

\begin{theorem}[Division theorem]\label{th:division}
    \begin{equation*}
        \forall a, b \in \mathbb{Z} \; \exists! q, r \in \mathbb{Z} \; \text{such that} \; a = b \cdot q + r \; \text{with} \; 0 \leq r < |b|
    \end{equation*}
\end{theorem}

To prove this theorem, all we need to prove is the existence and uniqueness of
such a decomposition. To not make the book too long, we will not prove this
theorem here, but you can find the proof in any number theory textbook.

This theorem allows us to define two new operations. Suppose $a,b,q,r$ are given
as in the \Cref{th:division}. Then,
\begin{itemize}
    \item \textbf{Floor Operation} ($\lfloor a/b \rfloor$ or $a \; \text{div} \; b$) is defined
    as $q$. This operation is a standard \texttt{div} opeartion commonly used in programming languages.
    \item \textbf{Mod Operation} ($a \; \text{mod} \; b$) is defined as $r$. This operation is a standard
\texttt{mod} operation commonly used in programming languages. 
\end{itemize}

\begin{example} \label{example:divisibility_2}
    Continuing \Cref{example:divisibility_1}, \hfill
    \begin{enumerate}
        \item We know that $5 \nmid 29$. However, by \Cref{th:division} we can express 29 
        as~$29 = 5 \cdot 5 + 4$.
        \item Similarly, it is evident that $34 \nmid 123$, but $123 = 3 \cdot 34 + 21$.
    \end{enumerate}
\end{example}

In division operations, it is common to check for any shared factors between two
numbers. This is where the concept of the \textbf{greatest common divisor} (or
\textbf{GCD} for short) comes into play.

\begin{definition}[GCD]
    For any $a, b \in \mathbb{Z}$, the \textbf{greatest common divisor} $\gcd(a, b)$ is defined as an integer $d \in \mathbb{N}$ such that:
    \begin{enumerate}
        \item $d \mid a$ and $d \mid b$.
        \item $d$ is a maximal integer that satisfies the first condition.
    \end{enumerate}

    One might right the above definition more concisely:
    \begin{equation*}
        \gcd(a,b) = \max\{d \in \mathbb{N}: d \mid a \; \text{and} \; d \mid b\}.
    \end{equation*}
\end{definition}

\begin{example}
    Just as an example, let us find the $\gcd(a, b)$ for few pairs of integers.
    
    \begin{multicols}{2}
        \begin{enumerate}
            \item $\gcd(42, 28) = 14$
            \item $\gcd(36, 14) = 2$
            \item $\gcd(13, 17) = 1$
            \item $\gcd(13, 18) = 1$
        \end{enumerate}
    \end{multicols}
\end{example}

\begin{definition}
    Integers $a, b \in \mathbb{Z}$ are \textbf{coprime} if and only if $\gcd(a, b) = 1$.
\end{definition}

As with any fundamental concept, let us explore some basic properties of the GCD operation.

\begin{lemma} [Greatest common divisor properties]
    \hfil
    \begin{enumerate}
        \item $\gcd(a, b) = b \iff b \mid a$
        \item If $a \neq 0$, then $\gcd(a, 0) = a$
        \item If there exists $\delta \in \mathbb{Z}$ such that $\delta \mid a$ and $\delta \mid b$, then $\delta \mid \gcd(a, b)$
        \item If $c > 0$, then $\gcd(ac, bc) = c \cdot \gcd(a, b)$
        \item Suppose $d = \gcd(a, b)$. Then, $\gcd(a/d, b/d) = 1$
    \end{enumerate}
\end{lemma}

\begin{lemma} \label{lemma:gcd_basic_calculation}
    For any integers $a,b \in \mathbb{Z}$, we have $\gcd(a, b) = \gcd(b, a - b)$. 
\end{lemma}

When considering how to find the greatest common divisor (GCD) or exploring algorithms 
for this purpose, \Cref{lemma:gcd_basic_calculation} serves as a useful starting point. 
However, on its own, it is not sufficient. The following lemma will be extensively employed
in the Euclidean algorithm, and its generalization is both natural and essential for more 
advanced applications.

\begin{corollary} \label{cor:euclidean}
    For any $a,b \in \mathbb{Z}$, we have 
    $\gcd(a,b) =\gcd(b, a \; \texttt{mod} \; b)$.
\end{corollary}

These properties are interesting and useful from the theory standpoint, but
you may wonder how to practically find $\gcd(a, b)$ (say, if you were to
implement this function in the programming language). \emph{Euclidean algorithm} is an
efficient method for computing the greatest common divisor. The main idea of 
Euclidean algorithm is to recursively apply the \Cref{cor:euclidean}. The concrete implementation 
in Python is specified below. 
\begin{lstlisting}[language=Python, numbers=none]
def gcd(a: int, b: int) -> int:
    return gcd(b, a % b) if b != 0 else a
\end{lstlisting}

Notice that the algorithm can be easily implemented in a single line.

\begin{example}
    Let us find the $\gcd(535, 230)$. We will use the Euclidean algorithm for this purpose.
    \hfill
    \begin{equation*}
        \begin{aligned}
            535 &= 2 \cdot 230 + 75 \\
            230 &= 3 \cdot 75 + 5 \\
            75 &= 15 \cdot 5 + 0 \\            
        \end{aligned}
    \end{equation*}
    Therefore, $\gcd(535, 230) = 5$.
\end{example}

While the greatest common divisor (GCD) focuses on finding the largest shared factor between two numbers, the least common multiple (LCM) deals with finding the smallest multiple that both numbers have in common. The LCM is particularly useful when we need to synchronize cycles or work with fractions.

\begin{definition}[LCM]
    For any $a, b \in \mathbb{Z}$, the \textbf{least common multiple} $\lcm(a,
    b)$ is defined as an integer $m \in \mathbb{N}$ such that:
    \begin{enumerate}
        \item $a \mid m$ and $b \mid m$
        \item $m$ is a minimal integer that satisfies the first condition 
    \end{enumerate}

    One might right the above definition more concisely:
    \begin{equation*}
        \lcm(a,b) = \min\{m \in \mathbb{N}: a \mid m \; \text{and} \; b \mid m\}.
    \end{equation*}
\end{definition}

\begin{example}
    Let us again find, but now $\lcm$ for few pairs of numbers.
    \hfill
    \begin{multicols}{2}
        \begin{enumerate}
            \item $\lcm(12, 4) = 12$
            \item $\lcm(13, 17) = 221$
            \item $\lcm(13, 18) = 234$
            \item $\lcm(42, 28) = 84$
        \end{enumerate}
    \end{multicols}
\end{example}

\begin{lemma} [Least Common Multiple Properties]
    \hfil
    \begin{enumerate}
        \item We assume that $\lcm(a, 0)$ is undefined.
        \item $\lcm(a, b) = a \iff b \mid a$.
        \item If $a$ and $b$ are coprime, then $\lcm(a, b) = a \cdot b$.
        \item Any common divisor $\delta$ of $a$ and $b$ satisfies $\delta \mid \lcm(a, b)$.
        \item For any $c > 0$, we have $\lcm(a \cdot c, b \cdot b) =  c \cdot \lcm(a, b)$.
        \item Integers $\lcm(a, b)/a$ and $\lcm(a, b) / b$ are coprime.
    \end{enumerate}
\end{lemma}

\begin{theorem}
    For any $a, b \in \mathbb{N}$, we have $\gcd(a, b) \cdot \lcm(a, b) = ab$.
\end{theorem}

One interpretation of the above theorem is that no additional algorithm is required for
$\text{lcm}(a, b)$ if we already have an algorithm for $\gcd(a, b)$. Indeed, we
can simply use the formula $\text{lcm}(a, b) = ab / \gcd(a, b)$ with the
previously computed $\gcd(a, b)$.

The reasonable question is how to generalize the $\gcd$ and $\lcm$ operations to more than two arguments. 
For that reason, we provide the following algorithm:
\begin{definition}[GCD and LCM for multiple arguments]
    We define the \textbf{greatest common divisor} $\gcd(a_1, a_2, \dots, a_n)$ and the \textbf{least common multiple} $\lcm(a_1, a_2, \dots, a_n)$ for any set of integers $a_1, a_2, \dots, a_n \in \mathbb{Z}$ as follows:
    \begin{align*}
        \gcd(a_1, a_2, \dots, a_n) &= \max\{ d \in \mathbb{N}: d \mid a_1, d \mid a_2, \dots, d \mid a_n \}. \\
        \lcm(a_1, a_2, \dots, a_n) &= \min\{ m \in \mathbb{N}: a_1 \mid m, a_2 \mid m, \dots, a_n \mid m \}.
    \end{align*}
\end{definition}

\begin{theorem}[Computational Properties Of GCD and LCM For Multiple Arguments.]
    The following two statements are true:
    \begin{itemize}
        \item $\forall a, b \in \mathbb{N}: \gcd(a, b, c) = \gcd(\gcd(a, b), c) = \gcd(a, \gcd(a, b))$.
        \item $\forall a, b \in \mathbb{N}: \lcm(a, b, c) = \lcm(\lcm(a, b), c) = \lcm(a, \lcm(a, b))$.
    \end{itemize}
\end{theorem}

In conclusion, from these two theorems there is no necessity for specific
algorithms for $\gcd(\cdot)$ and $\lcm(\cdot)$ when dealing with many arguments. There are
more specialized algorithms for each when considering a specific number of
arguments, but unfortunately, such topics are beyond the scope of this book.

\subsection{Extended Euclidean algorithm}

In this section, we will introduce the Extended Euclidean Algorithm and an
important lemma related to the GCD. You might have reasonable question: why do
we even need an extended version? One of the primary reasons is that this 
algorithm will help in finding inverse modular elements, introduced in the 
subsequent sections.

\begin{lemma} [Bezout identity] \label{lemma:bezout_identity}
    For any two given integers $a, b \in \mathbb{N}$ with $d = \gcd(a, b)$ there exists such $u, v \in \mathbb{Z}$ that $d = au + bv$.
\end{lemma}

\begin{example}
    For example $\gcd(125, 93) = 1$, and we can find 
    such $u, v \in \mathbb{Z}$ that $1 = 125 \cdot u + 93 \cdot v$. 
    The Bezout coefficients are $u = 32$ and $v = -43$.
    Therefore, $1 = 125 \cdot 32 + 93 \cdot (-43)$.
    We will explain and show how to find these coefficients a little later.
\end{example}

\begin{corollary} [From Bezout identity]
    \hfill
    \begin{enumerate}
        \item Integers $u$ and $v$ are of different signs (excluding the case when either $u=0$ or $v=0$).
        \item \textit{Generalization for multiple integers:} Suppose $d = \gcd(a_1, a_2, \dots, a_n)$, then there exist such integers $u_1, u_2, \dots, u_n \in \mathbb{Z}$ that $d = u_1 a_1 + u_2 a_2 + \cdots + u_n a_n$.
    \end{enumerate}    
\end{corollary}

The integers $u$ and $v$ are called \textbf{Bezout coefficients}. The first
corollary can be understood intuitively: if all coefficients are non-negative,
the result will be much larger than necessary. Similarly, if they are
non-positive, the result must be negative, but the GCD was defined to be
positive. The second consequence follows from the fact that we can decompose
GCD, thus sequentially derive this sequence. Also note that we can find Bezout
coefficients on each Euclidean algorithm step.

Now we introduce the \textbf{extended Euclidean algorithm.} The extended
Euclidean algorithm finds the Bezout coefficients together with the GCD
efficiently.

\begin{algorithm}[H]
    \caption{Extended Euclidean algorithm} \label{alg:extended_euclidean}
    \Input{$a, b \in \mathbb{N}$. Without loss of generality, $a \geq b$}
    \Output{$(\gcd(a, b), u, v)$}
        
    $r_{0} \gets a; \, r_{1} \gets b$ \\
    $u_{0} \gets 1; \, u_{1} \gets 0$ \\
    $v_{0} \gets 0; \, v_{1} \gets 1$ \\

    \While{$r_{i+1} \neq 0, \; i = 1, 2, \dots$}{
        $q_i \gets  r_{i-1} \; \text{div} \; r_{i}$ \\
        $u_{i+1} \gets u_{i-1} - u_{i} q_i$ \\
        $v_{i+1} \gets v_{i-1} - v_{i} q_i$ \\
        $r_{i+1} \gets  a u_{i+1} + b v_{i}$ \\
    }

    \textbf{return} $(r_i, u_i, v_i)$
\end{algorithm}

The time complexity of the Extended Euclidean algorithm is $O(\log{a} \log{b})$ bit
operations, which is very efficient. Although we already know how the
Extended Euclidean Algorithm works, this method is still not very
human-friendly. For that reason, let us consider an example of an easy way to
find the GCD.

\begin{example} [Extended Euclidean algorithm example]
    \hfill

    First, we you need to find $d = \gcd(a,b)$. Then, knowing the sequence of
    expansions, find the Bezout coefficients. Note that this can be done
    simultaneously.

    \hfill

    \begin{minipage}{0.35\textwidth}
        \raggedright
        \vspace*{\fill}
            \begin{enumerate}
                \item $125 = 93 \cdot 1 + 32$
                \item $93 = 32 \cdot 2 + 29$
                \item $32 = 29 \cdot 1 + 3$
                \item $29 = 3 \cdot 9 + 2$
                \item $3 = 2 \cdot 1 + 1$
                \item $2 = 1 \cdot 2 + 0$        
            \end{enumerate}
        \vspace*{\fill}
    \end{minipage}
    \begin{minipage}{0.75\textwidth}
        \resizebox{0.8\textwidth}{!}{
            \begin{tabular}{|c|c|c|c|c|c|c|c|}
                \hline
                
                \rowcolor{gray!30} $i$   & $0$      & $1$ & $2$ & $3$ & $4$ & $5$ & $6$ \\
                
                \hline
                \hline

                $q_i$ & $\times$ & $1$ & $2$ & $1$ & $9$ & $1$ & $2$ \\
                
                \hline
                
                $u_i$ & $1$      & $0$ & $1$ & $-2$ & $3$ & $-29$ & $32$ \\
                \hline
                $v_i$ & $0$      & $1$ & $-1$ & $3$ & $-4$ & \textcolor{green!60!black}{\textbf{?}} & \textcolor{green!60!black}{\textbf{?}} \\
                
                \hline
            \end{tabular}
        }
    \end{minipage}

    \hfill
    
    Here, each corresponding cell is calculated with the following formula: 
    \begin{align*}
        u_{i-1} - q_i u_i &= u_{i+1} \\
        v_{i-1} - q_i v_i &= v_{i+1} 
    \end{align*}
    Knowing that, try to finish this example by filling in the missed cells
    marked by \textcolor{green!60!black}{\textbf{?}}. After finding $v_6$, be
    sure to check yourself.
\end{example}

\subsection{Prime numbers}

Prime numbers are fundamental in mathematics due to their role of the building
blocks of all natural numbers. Every integer greater than 1 can be uniquely
factored into primes, a concept known as the \emph{Fundamental Theorem of
Arithmetic}, which we introduce in the next section. This property makes primes
central to number theory, and they play a crucial role in various mathematical
proofs and structures.

\begin{definition}
    Number $n \in \mathbb{N}$ is \textbf{prime} if and only if its only two divisors are $1$
    and $n$. 
\end{definition}

\begin{definition}
    Number $n \in \mathbb{N}$ is \textbf{composite} iff there exists an integer
    $a \in \mathbb{N}$, which is not $1$ or $n$, for which $a \mid n$. In other words,
    it is not prime.
\end{definition}

\begin{example}
    For example 3, 17, 19, 29, 71, 997, 7817, 104729 are all examples of prime numbers.
    In contrast, 8, 1001, 10000, 100000 are all examples of composite numbers.
\end{example}

One might ask: what to do with the number $1$? We consider it to be neither prime 
nor composite.

\begin{theorem} [Euclidean theorem]
    If $P = \{p_1, p_2, p_3, \dots p_k\} \subset \mathbb{N}$ is a finite set,
    consisting of prime numbers, then there exists a prime number $p$ such that
    $p \notin P$.
\end{theorem}

\begin{corollary}
    There are exist an infinite number of primes.
\end{corollary}

\begin{lemma}
    For all $n \in \mathbb{N}$, where $n$ is composite, if there exists a 
    minimal divisor $d>1$ of $n$, then $d$ is a prime number.
\end{lemma}

Let's say we have a large number, and we need to find out if it is prime or not, 
how do we do it? In other words, are there any methods for checking a number for prime?
Yes, there are, starting with logical reasoning, you can check numbers with brute force, although
this method is not practically applicable. There are probabilistic prime tests that will fail
with some error rate. In 2003, an effective deterministic test of simplicity was found, which moved
the problem to the $P$ class (complexity).

It is worth mentioning, that different forms of primes find their application in some 
fields. For example, Mersenne primes, factorial primes, Euclidean primes, Fibonacci primes,
and many other types of primes. We will consider one of the most famous types of 
primes --- Mersenne primes.

\begin{definition} [Mersenne primes]
    The prime number of the form $2^p - 1$ is called \textbf{Mersenne prime}, where $p$ is a
    prime number, usually notated as following $M_p = 2^p - 1$.
\end{definition}

\begin{example}
    For example, $2^7 - 1 = 127$ is a Mersenne prime, since $127$ is a prime number.
    However, $2^{11} - 1 = 2047$ is not a Mersenne prime, since $2047 = 23 \cdot 89$. 
    As of 2025, only 52 such numbers were found, the largest of which is $2^{136279841} - 1$.
\end{example}

The following lemma about the form of prime numbers 
is also worth mentioning. 

\begin{lemma}
    If the number $M_p$ is prime, then $p$ is also prime.
\end{lemma}

Mersenne primes, are important in both number theory and cryptography.
They have unique mathematical properties that make them useful for testing primality and generating large prime numbers.
Mersenne primes are also crucial in the construction of efficient algorithms for error correction in coding theory and for generating random numbers in cryptographic applications, etc. 
Beside, next theorem importan says that we can know the form of a prime numbers.

\begin{theorem} [Dirichlet's theorem]
    For any $a, b \in \mathbb{N}$ if $\gcd(a, b) = 1$, then infinite primes number form of $am + b$ exists, where $m \in \mathbb{N}$.
\end{theorem}

In other words, every infinite arithmetic progression whose first term and difference are positive integers contains an infinite number of primes.

\subsection{Fundamental Theorem of Arithmetic}
The Fundamental Theorem of Arithmetic states that every integer greater than 1 can be 
uniquely factored into primes, which is crucial for understanding the structure of numbers. 
It plays a key role in areas like number theory, cryptography, and simplifying 
calculations involving divisibility. 

\begin{example}
    For example let us decompose the number $n = 123456$ into a product of prime numbers.
    \begin{equation*}
        123456 = 2^6 \cdot 3^1 \cdot 643^1.
    \end{equation*}
\end{example}

But before formal description of the theorem, for better understanding it is good to know the following Lemma \ref{euclidean_lemma}.

\begin{lemma}[Euclidean] \label{euclidean_lemma}
    If $p$ is a prime and $p \mid ab$, then $p \mid a$ or $p \mid b$.
\end{lemma}

\textbf{Proof.} 
$\blacktriangleright$
Let $p \mid ab$, but $ p \nmid a$, then $\gcd(a, p) = 1$ because there are no common divisors.
In which case, by Bezout identity \Cref{lemma:bezout_identity}, there exist such $u, v \in \mathbb{Z}$ that $au + pv = 1$.
Let's multiply the left and right sides by $b$: $abu + pbv = b$, but $p \mid ab$ and $p \mid pb$, therefore their sum is also divisible by $p \mid abu + pbv$. 
$\blacktriangleleft$

Now, we are ready to introduce the central Number Theory theorem --- Fundamental
Theorem of Arithmetic.

\begin{theorem}[Fundamental theorem of arithmetic]\label{th:fundamental_arithmetic}
    Any integer $n>1$ can be decomposed in the unique way into a product of prime numbers:
    \begin{equation*}
        n = p_1^{\alpha_1}p_2^{\alpha_2}\cdots p_t^{\alpha_t} = \prod_{j=1}^t p_j^{\alpha_j},
    \end{equation*}
    where $p_1,\dots,p_t$ are prime numbers and $\alpha_1,\dots,\alpha_t \in \mathbb{N}$.
\end{theorem}

\textbf{Proof.} 
$\blacktriangleright$  
The theorem state equation then need to proof equation existence and uniqueness. Since the existence property is 
easy to prove, let us make the small exception and show it. For other proves (in particular, for the uniqueness case, see literature)

\textcolor{green!60!black}{\textbf{Existence}}. Suppose the theorem statement is false and there exist $n$ that do not have such a representation.
Let $n_0$ be the smallest of them.
If $n_0$ is prime, it can be represented as $p_1=n_0$, $\alpha_1=1$, which satisfies the representation. Therefore, $n_0$ must be composite, which means $\exists a, b \in \mathbb{N}, 1 < a, b < n_0$ such that $n = ab$.
Since $n_0$ is the smallest non-decomposable number and $a, b < n_0$, then $a$ and $b$ can be represented as 
\begin{align*}
    a &= p_{1}^{\alpha_{1}} \dots p_{t}^{\alpha_{t}} \\
    b &= q_{1}^{\beta_{1}} \dots q_{k}^{\beta_{k}}.
\end{align*}
Therefore, $n_0 = ab=p_{1}^{\alpha_{1}} \dots p_{t}^{\alpha_{t}} q_{1}^{\beta_{1}} \dots q_{k}^{\beta_{k}}$. Thus, the contradiction. $\blacktriangleleft$  

\begin{corollary}
    Suppose $n$ is decomposed as in \Cref{th:fundamental_arithmetic}. Then,

    \begin{enumerate}
        \item If $d \mid n$, then $d = p_{1}^{\beta_1}p_{2}^{\beta_2} \dots p_{t}^{\beta_t}$, where $0 \leq \beta_i \leq \alpha_i$.
        \item Suppose $a = p_{1}^{\alpha_1} \dots p_{t}^{\alpha_t}$ and $b = p_{1}^{\beta_1} \dots p_{t}^{\beta_t}$. Then, the GCD can be evaluated as $\gcd(a, b) = \prod_{i=0}^{t}p^{\min\{\alpha_i, \beta_i\}}$.
        \item Suppose $a = p_{1}^{\alpha_1} \dots p_{t}^{\alpha_t}$ and $b = p_{1}^{\beta_1} \dots p_{t}^{\beta_t}$. Then, the LCM can be evaluated as $\lcm(a, b) = \prod_{i=0}^{t}p^{\max\{\alpha_i, \beta_i\}}$.
        \item If $b \mid a$ and $c \mid a$ and $\gcd(b, c) = 1$, then $bc \mid a$.
    \end{enumerate}
\end{corollary}

The implications and applications of the Fundamental Theorem of Arithmetic are very large and important, and a number of “obvious” ones are listed above.
However, you should also be aware that the problem of factorization, i.e., knowing the decomposition of a number into prime factors, is in a NP class (complexity) and is the basis of some cryptosystems, although it is gradually being abandoned, including due to the potential of quantum computers.

\subsection{Modular arithmetic}

Now we are ready for practical applications of number theory, starting with modulo congruence.
In this section, we will review the idea of congruence, learn how to use it, and explore
its key properties. Understanding these properties will help simplify calculations and solve 
problems more efficiently in fields such as cryptography, algorithms, and number theory.

\begin{definition}
    Two integers $a, b \in \mathbb{Z}$ are said to be \textbf{congruent} modulo $n \in \mathbb{N}$ (congruence modulo $n$ is denoted as $a \equiv b \pmod{n}$), if one of the following conditions is met:
    \begin{enumerate}
        \item There exists such $t \in \mathbb{Z}$ that $a = b + nt$
        \item $a \; \texttt{mod} \; n = b \; \texttt{mod} \; n$
        \item $n \mid (a - b)$
    \end{enumerate}
\end{definition}

\begin{example}
    It easy to see that $26 \equiv 5 \pmod{7}$, since $26 = 5 + 7 \cdot 3$ 
    or $26 \; \texttt{mod} \; 7 = 5 \; \texttt{mod} \; 7 = 5$, etc.
\end{example}

It is fairly easy to see that all conditions are equivalent to each other. Next, as always, it is necessary to formally describe the properties, although some of them may seem intuitive, they need to be formally defined.

\begin{lemma}[Reflexivity, Symmetry, and Transitivity] 
    \hfill
    \begin{enumerate}
        \item $a \equiv a \pmod{n}$
        \item If $a \equiv b \pmod{n}$, then $b \equiv a \pmod{n}$
        \item If $a \equiv b \pmod{n}$ and $b \equiv c \pmod{n}$, then $a \equiv c \pmod{n}$
    \end{enumerate}
\end{lemma}

The lemma states three basic properties of congruence modulo $n$: reflexivity, symmetry, and transitivity. 
These properties confirm that congruence modulo $n$ is an equivalence relation (for general equivalence relation definition, see \Cref{section:relations}).

\begin{lemma}
    Suppose we have $a \equiv b \pmod{n}$ and $c \equiv d \pmod{n}$, then 
        \begin{enumerate}
            \item $a \pm c \equiv b \pm d \pmod{n}$
            \item $ac \equiv bd \pmod{n}$
        \end{enumerate}
\end{lemma}

In turn, this lemma states that we can perform addition, subtraction, and multiplication on the congruent numbers modulo $n$ similarly to the usual arithmetic.

\begin{example}
    Let use see how the lemma works in practice.
    \begin{enumerate}
        \item $17 \times 26 \pmod{7} \equiv 3 \times 5 \equiv 15 \equiv 1 \pmod{7}$
        \item $5 + 26 \pmod{7} \equiv 5 + 5 \equiv 10 \equiv 3 \pmod{7}$
        \item $(-5) \times 13 \pmod{7} \equiv (7-5) \times 6 \equiv 2 \times 6 \equiv 12 \equiv 5 \pmod{7}$
    \end{enumerate}
\end{example}

\begin{lemma}
    If $ca \equiv cb \pmod{n}$, $\gcd(c, n) = 1$, then $ a \equiv b \pmod{n}$.
\end{lemma}

This means that we can cancel out the same left and right parts respectively, but with a 
certain requirement. You may ask why this is necessary, the answer is in the following example.

\begin{example}
    Let us try to simplify the following equation: $6 \equiv 2 \pmod{4}$. Suppose we 
    divide both sides by $2$, then we have the false statement $3 \equiv 1 \pmod{4}$. 
    That being said, the requirement $\gcd(c, n) = 1$ is mandatory. 
\end{example}

\begin{remark}
    In particular, the example above shows why the division operation is fundamentally 
    different from the regular real/rational numbers. We will see that, in fact, the 
    arithmetic modulo $n$ is not always what mathematicians call a \textbf{field}. 
    You will see more details in \Cref{section:field_extensions}.
\end{remark}

\begin{lemma} \label{lemma:congruence_scale}
    Modulo congruence can be scaled in both directions,
    
    \begin{enumerate}
        \item If $k \neq 0, a \equiv b \pmod{n}$, then $ak \equiv bk \pmod{nk}$
        \item If $d = \gcd(a, b, n) \text{ and } a = a_1d, b =b_1d, n = n_1d, a \equiv b \, (\text{mod } n)$, then $a_1 = b_1 \, (\text{mod } n_1)$
    \end{enumerate}
\end{lemma}

\begin{lemma}
    If $d \mid n$ and $a \equiv b \, (\text{mod } n)$, then $a \equiv b \, (\text{mod } d)$.
\end{lemma}

This property is very convenient when it comes to large calculations, if you know its decomposition.
It allows you to significantly reduce them, and then restore the result by a large modulus.

\begin{lemma}
    Suppose $a \equiv b \pmod{n_1}$, $a \equiv b \pmod{n_2}$, \ldots, $a \equiv b \pmod{n_k}$. Then, the following statement is true: 
    \begin{equation*}
        a \equiv b \pmod{\lcm(n_1, n_2, \dots, n_k)}.
    \end{equation*}
\end{lemma}

\begin{lemma}
    If $a \equiv b \, (\text{mod } n)$, then $\gcd(a, n) = \gcd(b, n)$
\end{lemma}

\begin{definition}
    The congruence class or residues of $k$ modulo $n$ is defined as a set 
    $k + n\mathbb{Z} = \{ k + nt \mid t \in \mathbb{Z}\}$, sometimes denoted as $[k]$ or $[k]_n$.
\end{definition}

\begin{example}
    As we can see congruence module 2 we have, 
    \begin{equation*}
        \begin{aligned}
        &[0]_2 = \{0, \pm 2, \pm 4, \pm 6, \pm 8, \ldots \}, \\
        &[1]_2 = \{1, \pm 3, \pm 5, \pm 7, \pm 9, \ldots \}.
        \end{aligned}
    \end{equation*}
    Thus, the congruence classes of $[0]_2, [1]_2$ are respectively even and odd numbers.
    However, with modulo 5, we have the following,
    \begin{equation*}
        [3]_5 = \{3, 3 \pm 5, 3 \pm 10, 3 \pm 15, \ldots \}.
    \end{equation*}
\end{example}

\begin{definition}
    The \textbf{complete residue system modulo} $n$ (or residue ring) is a set of integers, 
    whereevery integer is congruent to a unique member of the set modulo $n$, usually 
    denoted as $\mathbb{Z}_n = \{0, 1, 2, \dots, n-1\}$. 
    In more formal literature, the set is often denoted as $\mathbb{Z}_n = \{ [0], [1], \ldots, [n-1] \}$.
\end{definition}

\begin{example}
    $\mathbb{Z}_{5}$ is a complete residue system modulo $5$ and consists of the following elements: $\mathbb{Z}_5 = \{0, 1, 2, 3, 4\}$.
    Also, as another example $\mathbb{Z}_{7} = \{0, 1, 2, 3, 4, 5, 6\}$ and others.
\end{example}

\begin{remark}
    Sometimes, one might also encounter the notation $\mathbb{Z}/n\mathbb{Z}$, which is more frequently used in Abstract Algebra.
\end{remark}

The aforementioned lemmas and definitions describe the properties of addition, subtraction, and multiplication. But what about division? To define the division (if such operation is valid at all), we need to consider the so-called \emph{modular multiplicative inverse}, which is usually denoted as $a^{-1} \, (\text{mod } n)$, similarly to real/rational numbers.

\begin{definition}
    A modular multiplicative inverse of an integer $a \in \mathbb{Z}$ modulo $n \in \mathbb{N}$ is such an integer $a^{-1}$ that satisfies $a \cdot a^{-1} \equiv a^{-1}a \equiv 1 \pmod{n}$.
\end{definition}

\begin{example}
    Let us find the modular multiplicative inverse of $a = 3$ modulo $n = 7$.
    We need to find such $a^{-1}$ that satisfies $3 \cdot a^{-1} \equiv 1 \pmod{7}$.
    By simple brute force, we can find that $3 \cdot 5 \equiv 1 \pmod{7}$, thus~$3^{-1} = 5$.
\end{example}

\begin{remark}
    The inverse (if exists) behaves similarly to the usual inverse over rations/reals. For example, $(a^{2})^{-1} = (a^{-1})^{2}$, which means, we can first find the inverse, then the squared value, and vice versa.
\end{remark}

The question then is when does the number have the inverse? Consider the following theorem.

\begin{theorem}\label{th:inverse_existence}
    The modular multiplicative inverse $a^{-1} \pmod{n}$ exists if and only if $\gcd(a, n) = 1$.
\end{theorem}

Based on Extended Euclidean \Cref{alg:extended_euclidean} and \Cref{th:inverse_existence}, we build the algorithm that can verify an existence and find the inverse value.

\begin{algorithm}[H]
    \caption{Modular multiplicative inverse algorithm} \label{alg:modular_inverse}
    \Input{$a, n$}
    \Output{$a^{-1}$}
        
    $(d, u, v) \gets \text{ExtendedEuclideanAlgorithm}(a, n)$ \Comment{See \Cref{alg:extended_euclidean}}

    \If{$d \neq 1$}{
        \textbf{return} inverse does not exist.
    } 

    \textbf{return} $u$
\end{algorithm}

\begin{example}
    Let us find the modular multiplicative inverse of $a = 12$ modulo $n = 17$.
    First of all, we need to verify that the inverse exists, in other words if $\gcd(12, 17) = 1$.

    \begin{equation*}   
        \begin{aligned}
            17 &= 12 \cdot 1 + 5 \\
            12 &= 5 \cdot 2 + 2 \\
            5 &= 2 \cdot 2 + 1 \\
            2 &= 1 \cdot 2 + 0
        \end{aligned}
    \end{equation*}

    Thus, we verify inverse existence. In the previous example, we used 
    the Extended Euclidean \Cref{alg:extended_euclidean} to find Bezout coefficients,
    but in this case, we will show another method.

    \begin{equation*}
        \begin{aligned}
            1 &= 5 - 2 \times 2 = 5 - 2(12 - 2 \times 5) = 5 \times 5 - 2 \times 12 \\
              &= 5(17 - 12) - 2 \times 12 = 5 \times 17 - 7 \times 12.
        \end{aligned}
    \end{equation*}

    Intuitively, we go backward by substituting the corresponding composition.
    As you can see, we have found Bezout coefficients, $\gcd(12, 17) = 5 \times 17 - 7 \times 12$.
    Therefore, the inverse is $-7$, but as we already know, the inverse must be positive, thus $10$.
    In the end, it is highly desirable to check the result obtained,~$12 \times 10 \equiv 1 \pmod{17}$.

\end{example}

\begin{remark}
    Note that the complexity of this algorithm is the same as for \Cref{alg:extended_euclidean}, which is $O(\log a \log n)$ bit operations.
\end{remark}

\begin{definition}
    The \textbf{multiplicative group of integers} modulo $n$, denoted as $\mathbb{Z}_n^{\times}$, is a set of natural numbers that are coprime to $n$ and less than $n$. In other words, $\mathbb{Z}_n^{\times} = \{a \in \mathbb{N}: \gcd(a, n) = 1\}$.
\end{definition}

\begin{example}
    $\mathbb{Z}_{11}^{\times} = \{1, 2, 3, 4, 5, 6, 7, 8, 9, 10\}$ as you can see, 
    all numbers are coprime to $11$.
    However, $\mathbb{Z}_{12}^{\times} = \{1, 5, 7, 11\}$, since 
    integers $2, 3, 4, 6, 8, 9, 10$ are not coprime to $12$. 
\end{example}

The $\mathbb{Z}_{n}^{\times}$ is a fundamental object in number theory and 
plays a crucial role in cryptography, forming the basis almost for every cryptographic primitive.

The structure of the group $\mathbb{Z}_{n}^{\times}$ has crucial meaning and has deep connections 
with algebraic structures. We will consider this in more detail in the next sections, but for now, 
let us consider a function that helps to understand the size (order) of $\mathbb{Z}_{n}^{\times}$. 

\begin{definition} \label{def:euler_totient_function}
    \textbf{Euler's Totient Function} $\varphi(n)$ is the cardinality of the multiplication 
    group of integers $\mathbb{Z}_n^{\times}$. In other words, $\varphi(n) = |\mathbb{Z}_n^{\times}|$.
\end{definition}

\begin{remark}
    The alternative Euler's totient function intuition is following: $\varphi(n)$ counts all coprimes with $n$ in range $[1, n]$. 
\end{remark}

\begin{lemma} [Euler's totient function properties]
    \hfill
    \begin{enumerate}
        \item $\varphi(1) = 1$.
        \item $\varphi(p) = p - 1$, where $p$ is prime.
        \item $\varphi(pq) = \varphi(p) \cdot \varphi(q)$, where $p, q$ are primes.
        \item $\varphi(p^{\alpha}) = p^{\alpha} - p^{\alpha - 1}$, where $p$ is prime.        
    \end{enumerate}    
\end{lemma}

With these properties, we can derive a general formula for Euler's totient function. 

\begin{corollary}
    For any number $n = p_{1}^{\alpha_1}p_{2}^{\alpha_2} \dots p_{t}^{\alpha_t}$ general formula for Euler's totient function is: 
    \begin{equation*}
        \varphi(n) = \prod_{i = 1}^{t} \left( p_{i}^{\alpha_i} - p_{i}^{\alpha_i - 1} \right) = n \prod_{i = 1}^{t} \left( 1 - \frac{1}{p_i} \right).
    \end{equation*}
\end{corollary}

\begin{example}
    \hfill

    \begin{itemize}
        \item $\varphi(107) = 106$, because $107$ is prime.
        \item $\varphi(123) = \varphi(3 \cdot 41) = \varphi(3)\varphi(41) = 2 \cdot 40 = 80$
        \item $\varphi(729) = \varphi(3^6) = 3^6 - 3^5 = 486$
    \end{itemize}
\end{example}

\subsection{Chinese Remainder Theorem}
The Chinese Remainder Theorem is a fundamental result in number theory
that provides an efficient method for solving congruence systems. It allows you to
decompose a complex problem with large integers into several simpler 
problems with smaller modules. This decomposition reduces the required computational 
complexity, especially when working with large numbers, and is widely used 
in areas such as modular arithmetic, cryptography, and algorithmic number theory. 

\begin{theorem} [Chinese Remainder Theorem] \label{th:chinese_remainder_theorem}
    Suppose there is the following system
    \begin{equation*}    
        \begin{cases}
            x \equiv x_1 \pmod{n_1} \\
            x \equiv x_2 \pmod{n_2} \\
            x \equiv x_3 \pmod{n_3} \\
            \cdots \\
            x \equiv x_t \pmod{n_t} \\
        \end{cases}
    \end{equation*}
    where the $n_i$ are pairwise coprime and $x_1, x_2, \cdots x_t$ are fixed numbers. 
    Then there is a unique solution in modulo $N = n_1n_2 \cdots n_t$.
\end{theorem}

In fact, we can easily find such a solution. Let $N_i = N/n_i$ since $n_i$ are pairwise coprime; 
for all $i$ have $\gcd(N_i, n_i) = 1$, and by the Extended Euclidean Algorithm, there exists an integer
$a_i$ such that $a_iN_i \equiv 1 \pmod{n_i}$. Thus, the solution is given by 
\begin{equation*}
    x_0 = a_1N_1x_1 + a_2N_2x_2 + \cdots + a_tN_tx_t \pmod{N}.
\end{equation*}
This expression provides the unique solution modulo $N$, where $N = n_1 n_2 \cdots n_t$. 
Besides, using this idea, we can write an efficient algorithm to compute $x_0$ iteratively.

\begin{example}
    Let us solve the following system of congruences.
    \begin{equation*}    
        \begin{cases}
            x \equiv 1 \pmod{5} \\
            x \equiv 2 \pmod{7} \\
            x \equiv 3 \pmod{11}
        \end{cases}
    \end{equation*}

    First of all find the module $N = 5 \times 7 \times 11 = 385$, then each $N_i$. \\
    \vspace{-7mm}
    \begin{multicols*}{2}    
        \begin{itemize}
            \item $N_1 = 7 \times 11$ = 77
            \item $N_2 = 5 \times 11$ = 55
            \item $N_3 = 5 \times 7$ = 35
        \end{itemize}
    \end{multicols*}
    \vspace{-3mm}

    To simplify the calculations, we are not going to find the each inverse, suppose 
    we somehow know them, $a_1 = 3$, $a_2 = 6$, $a_3 = 6$.
    Finally, the solution is $x_0 \equiv 3 \times 77 \times 1 + 6 \times 55 \times 2 + 6 \times 35 \times 3 \equiv 1521 \equiv 81 \pmod{385}$.
    \\
    
    However, as you can notice, the solution is not unique. We can increase the value with $N = 385$.
    Thus, the final solutions are $x = 81 + 385k \in \mathbb{Z}$.

\end{example}

\subsection{Euler's Theorem}
Now we are ready to introduce one of the fundamental results in Number Theory.
Euler's Theorem was discovered by the mathematician Leonhard Euler 
as a generalization of Fermat's Little Theorem.

Why is it so important? Euler's Theorem is a key result in number theory and 
cryptographic applications, such as RSA encryption, and so on. 
Essentially, it helps in understanding the structure of the multiplicative group of 
integers modulo $n$ and gives a foundation for other important results in number theory. 
Besides it provides the simplification of large exponents in modular arithmetic. 

\begin{theorem} [Euler] \label{th:euler_theorem}
    For all $n \in \mathbb{N}$, $a \in \mathbb{Z}_n^{\times}$ 
    holds $a^{\varphi(n)} \equiv 1 \pmod{n}$.
\end{theorem}

$\blacktriangleright$ To prove this fact, we prove the following auxiliary lemma.

\begin{lemma}
    Suppose $a \in \mathbb{Z}_n^{\times}$. Denote by $a\mathbb{Z}_n^{\times} =
    \{ax: x \in \mathbb{Z}_n^{\times}\}$. Then, $\mathbb{Z}_n^{\times} =
    a\mathbb{Z}_n^{\times}$. In other words, elementwise multiplication by $a$
    permutes the elements of $\mathbb{Z}_n^{\times}$.
\end{lemma}

\textbf{Proof.} $\blacktriangleright$ Our statement is equivalent to claiming that the 
function $f: \mathbb{Z}_n^{\times} \to a\mathbb{Z}_n^{\times}$ defined 
as $f(x) = ax \pmod{n}$ is a \textbf{bijection}. To show this, we need to prove the following 
three staments:
\begin{enumerate}
    \item \textbf{Correctness} is obvious since $ax$ is in $a\mathbb{Z}_n^{\times}$ if $x$ is in $\mathbb{Z}_n^{\times}$.
    \item \textbf{Injectivity}: We need to prove that if $f(x_1)=f(x_2)$, 
    then $x_1=x_2 \in \mathbb{Z}_n^{\times}$. 
    From the function definition, we have $ax_1 \equiv ax_2 \pmod{n}$. By the following 
    \Cref{lemma:congruence_scale}, we can cancel $a$ modulo $n$ as $a$ is coprime to $n$. 
    Thus, $x_1 \equiv x_2 \pmod{n}$.
    \item \textbf{Surjectivity}: We need to prove that every element $y \in a\mathbb{Z}_n^{\times}$ has at least 
    one pre-image $f^{-1}(y)$. Indeed, in this case we have $ax \equiv y \pmod{n}$. Then, set $x := a^{-1}y \pmod {n}$. This 
    expression is well-defined as $a$ is coprime to $n$ and has an inverse.
\end{enumerate}

Therefore, $f$ is bijective and thus $\mathbb{Z}_n^{\times} = a\mathbb{Z}_n^{\times}$.
$\blacktriangleleft$

So we know, $a\mathbb{Z}_n^{\times} = \mathbb{Z}_n^{\times}$, how do we proceed? Notice, from the 
set equality follows the equality of the products of all elements. Let us write this down:
\begin{equation*}
    \prod_{i = 1}^{\varphi(n)} x_i \equiv \prod_{i = 1}^{\varphi(n)} a x_i \pmod{n} \implies \prod_{i = 1}^{\varphi(n)} x_i \equiv a^{\varphi(n)} \prod_{i = 1}^{\varphi(n)} x_i \pmod{n}
\end{equation*}    
    
Since all $x_i$ are coprime to $n$ by definition, we can cancel out the product
of all $x_i$ from both sides, leading to the conclusion of Euler's theorem.
$\blacktriangleleft$

\begin{corollary} \label{cor:fermat_little_theorem}
    If $p$ is a prime number and $a$ is not divisible by $p$, then $a^{p - 1} \equiv 1 \pmod{p}$.
\end{corollary}
This result is commonly known as Fermat's Little Theorem. However, to maintain generality, 
we state the theorem in a more general form below.

\begin{theorem} [Fermat's Little Theorem.] \label{th:fermat_little_theorem}
    Let $p$ is prime number. For any integer $a$, the following holds $a^{p} \equiv a \pmod{p}$.
\end{theorem}

The proof of Fermat's Little Theorem is a direct consequence of Euler's theorem.
We can consider Euler's Theorem as a generalization of Fermat's Little Theorem.

\begin{example}
    How to find an integer $29^{202} \pmod{13}$?
    By \Cref{th:euler_theorem} we have $29^{\varphi(n)} \equiv 1 \pmod{13}$, where $\varphi(13) = 12$. 
    Using this, let us try to simplify the expression, 
 
    \begin{equation*}
        \begin{aligned}
            29^{202} \pmod{13}  &\equiv \\
                                &\equiv 29^{12 \times 16 + 10} \pmod{13} \\
                                &\equiv (29^{12})^{16} 29^{10} \pmod{13} \\
                                &\equiv 29^{10} \pmod{13} \\
                                &\equiv 3^{10} \pmod{13} \\
                                &\equiv 3^{3 \times 3} 3 \pmod{13} \\
                                % &\equiv 27^{3}3 \pmod{13} \quad \text{/*} 27^{3} \equiv 1 \pmod{13} \text{*/} \\
                                &\equiv 27^{3}3 \pmod{13} \\
                                &\equiv 3 \pmod{13}.
        \end{aligned}
    \end{equation*}
\end{example}

In other words, we significantly reduce computational complexity by using Euler's theorem,
in practical applications.

\subsection{Exercises}

\begin{xexercise}
    {Exercise 1.}
    {Find GCD for $(1236, 5682)$.}
    {4}
    {
        \item 3 
        \item 6
        \item 1
        \item Not exist
    }
\end{xexercise}

\begin{xexercise}
    {Exercise 2.}
    {Find inverse element $37^{-1} \pmod{113}$.}
    {4}
    {
        \item -18
        \item 55
        \item 22
        \item Not exist
    }
\end{xexercise}

\begin{xexercise}
    {Exercise 3.}
    {Find inverse element $74^{-1} \pmod{138}$.}
    {4}
    {
        \item -15
        \item 28
        \item 6
        \item Not exist
    }
\end{xexercise}

\begin{xexercise}
    {Exercise 4.}
    {Find Euler's totient function $\varphi(9000)$.}
    {4}
    {
        \item 24
        \item 9999
        \item 1125
        \item 2400
    }
\end{xexercise}

\begin{xexercise}
    {Exercise 5.}
    {
        Find the solution of the following congruences system.
        \begin{xequation}    
            \begin{cases}
                x \equiv 1 \pmod{3} \\
                x \equiv 2 \pmod{5} \\
                x \equiv 4 \pmod{7} \\
                x \equiv 5 \pmod{11} \\
                x \equiv 9 \pmod{13}
            \end{cases}
        \end{xequation}
        \vspace{-3mm}
    }
    {2}
    {
        \item $8992 \pmod{15015}$
        \item $360 \pmod{15015}$
        \item $907 \pmod{15015}$
        \item $1445 \pmod{15015}$
    }
\end{xexercise}

% \exerciseTitle{Exercise 6.}{Prove that for arbitrary natural numbers n the following holds $\gcd(n, n + 1) = 1$.}

\end{document}
