\documentclass[../lecture-notes-148x210.tex]{subfiles}

\begin{document}

We will now define some of the notations and conventions used throughout this book, such as sets, function, etc.

\subsection*{Set Theory}

A \textbf{set} is a well-defined collection of objects \cite[section 1]{Judson_2012}. The objects that belong to a set are called 
its \textbf{elements}. We will denote sets by capital letters, if $a$ is an element of the set $A$, 
we write $a \in A$.

Let us enumerate some fundamental sets:

\begin{itemize}
    \item $\mathbb{N} = \{1, 2, 3, \dots\}$ -- a set of natural numbers.
    \item $\mathbb{Z} = \{0, \pm 1, \pm 2, \pm 3, \dots\}$ -- a set of integers.
    \item $\mathbb{Q} = \{\frac{n}{m}: n \in \mathbb{Z}, m \in \mathbb{N}\}$ -- a set of rational numbers.
    \item $\mathbb{R}$ -- a set of real numbers. Examples: $2.2, -3.9 0.12, 1.4, -6.7, \dots$
    \item $\mathbb{R}_{>0}$ -- a set of positive real numbers. Examples: $2.6, 10.4, 100.2$.
    \item $\mathbb{C}$ -- a set of complex numbers\footnote{Complex number is an expression in a form $x+iy$ for $i^2=-1$}. 
    Examples: $1+2i, 5i, -7-5.7i, \dots$.
\end{itemize}

To represent the number of elements in a set $A$, we write $|A|$. If the set is finite, $|A| \in \mathbb{N}$, 
otherwise $|A| = \infty$. $A \subset B$ denotes ``$A$ is a subset of $B$'' (meaning that all elements of $A$ 
are also in $B$, e.g., $\mathbb{Q} \subset \mathbb{R}$).

$A \cap B$ means the intersection of $A$ and $B$ (a set of elements belonging to both $A$ and $B$), while 
$A \cup B$ -- the union of $A$ and $B$ (the set of elements belonging to either $A$ or $B$). $A \setminus B$ 
denotes the set difference (the set of elements belonging to $A$, but not $B$). $\overline{A}$ denotes the 
complement of $A$ (the set of elements not belonging to $A$). All operations are illustrated in 
\Cref{fig:venn_diagrams} (this picture is typically called the \textit{Venn Diagram}).

% --- Writing diagrams ---
\def\firstcircle{(0,0) circle (1.5cm)}
\def\secondcircle{(0:2cm) circle (1.5cm)}

\colorlet{circle edge}{blue!50}
\colorlet{circle area}{blue!20}

\tikzset{filled/.style={fill=circle area, draw=circle edge, ultra thick},
    outline/.style={draw=circle edge, ultra thick}}

\begin{figure}[H]
    \begin{center}

    \scalebox{0.5}{
    \begin{tabular}{cc}
    % Set A and B
    \begin{tikzpicture}
        \begin{scope}
            \clip \firstcircle;
            \fill[filled] \secondcircle;
        \end{scope}
        \draw[outline] \firstcircle node {$A$};
        \draw[outline] \secondcircle node {$B$};
        \node[anchor=south] at (current bounding box.north) {$A \cap B$};
    \end{tikzpicture} &   
    %Set A or B but not (A and B) also known a A xor B
    \begin{tikzpicture}
        \draw[filled, even odd rule] \firstcircle node {$A$}
                                    \secondcircle node{$B$};
        \node[anchor=south] at (current bounding box.north) {$\overline{A \cap B}$};
    \end{tikzpicture}
    \\
    % Set A or B
    \begin{tikzpicture}
        \draw[filled] \firstcircle node {$A$}
                    \secondcircle node {$B$};
        \node[anchor=south] at (current bounding box.north) {$A \cup B$};
    \end{tikzpicture} &  
    % Set A but not B
    \begin{tikzpicture}
        \begin{scope}
            \clip \firstcircle;
            \draw[filled, even odd rule] \firstcircle node {$A$}
                                        \secondcircle;
        \end{scope}
        \draw[outline] \firstcircle
                    \secondcircle node {$B$};
        \node[anchor=south] at (current bounding box.north) {$A \setminus B$};
    \end{tikzpicture}
\end{tabular}
}
\end{center}
    \caption{Set operations illustrated with Venn diagrams.}
\label{fig:venn_diagrams}
\end{figure}

As you may have noticed, to define the set we typically write $\{f(a): \phi(a)\}$, where $f(a)$ is some 
function and $\phi(a)$ is a predicate (function, inputting $a$ and returning true/false if a certain condition 
on $a$ is met). For example, $\{x^3: x \in \mathbb{R}, x^2 = 4\}$ is ``a set of values $x^3$ which are the real 
solutions to equation $x^2 = 4$''. It is quite easy to see that this set is simply $\{2^3, (-2)^3\} = \{8,-8\}$. 

\subsection*{Cartesian Products and Functions}

The notation $A \times B$ means a set of pairs $(a,b)$ where $a \in A$ and $b \in B$ (or, written shortly, 
$A \times B = \{(a,b): a \in A, b \in B\}$), called a Cartesian product. We additionally introduce notation 
$A^n := A \times A \times \dots \times A$ -- Cartesian product $n$ times. For example, $\mathbb{Q}^3$ is a set of triplets 
$(a,b,c)$ where $a,b,c \in \mathbb{Q}$, while $\mathbb{Q}^2 \times \mathbb{R}$ is a set of triplets $(a,b,c)$ where 
$a,b \in \mathbb{Q}$ and $c \in \mathbb{R}$.

Subsets of $ A \times B$ are called relations \cite[section 1]{Judson_2012}. We
define a function $f \subset A \times B$ from a set $A$ to a set $B$, as the
special type of relation in which for each element $a \in A$ there is a unique
element $b \in B$ such that $(a, b) \in f$. In other words, for every element in
$A$ we assigns a unique element in $B$. We usually write $f : A \to B$ to denote
such a relation and write $f(a) = b$ instead of writing down $(a,b) \in f$.

We say the function $f : A \to B$ is \textbf{injective} if condition $a_1 = a_2$ implies
$f(a_1) = f(a_2)$. Function $f$ is said to be \textbf{surjective} if for each 
$b \in B$ there exists $a \in A$ such that $f(a) = b$. A function that is both
is injective and surjective is called a \textbf{bijection}.

% todo: add injective, surjective and bijective illustrations

\subsection*{Logic}

Statement beginning with $\forall$ means ``for all...''. For instance, $(\forall a \in A \subset \mathbb{R}): \{a < 1\}$ is 
read as: ``For any $a$ in set $A$ (which is a subset of real numbers), it is true that $a<1$''. Or, more shortly, ``Any (real) 
$a$ from $A$ is less than $1$''.

Statement beginning from $\exists$ means ``there exists such...''. Let us consider the following example: 
$(\exists \varepsilon > 0) (\forall a \in A): \{a > \varepsilon\}$ is read as ``there exists such a positive $\varepsilon$ such 
that for any element $a$ from $A$, $a$ is greater than $\varepsilon$'', or, more concisely, ``there exists a positive constant 
$\varepsilon$ such that any element from $A$ is greater than $\varepsilon$''.

Statement beginning from $\exists !$ means ``there exists a unique...''. For example, $(\exists! x \in \mathbb{R}_{>0}): \{x^2 = 4\}$ 
is read as ``there exists a unique positive real $x$ such that $x^2 = 4$''.

Symbol $\wedge$ means ``and''. For example, $\{x \in \mathbb{R}: x^2 = 4 \wedge x > 0\}$ is read as ``a set of real $x$ such that 
$x^2=4$ and $x$ is positive''. Of course, $\{x \in \mathbb{R}: x^2 = 4 \wedge x > 0\} = \{2\}$.

Symbol $\vee$ means ``or''. For example, $\{x \in \mathbb{R}: x^2 = 4 \vee x^2 = 9\}$ is read as ``a set of real $x$ such that 
either $x^2=4$ or $x^2=9$'', this set is equal to~$\{\pm 2, \pm 3\}$.

\subsection*{Sequences, Matrices, and Vectors}

A \textbf{sequence} is a list of elements, typically indexed by natural numbers.
To denote the regular sequence we write $\{x_n\}_{n \in \mathbb{N}}$. If the
sequence iterates through some specific range $\mathcal{I} \subset \mathbb{N}$,
we write $\{x_n\}_{n \in \mathcal{I}}$. For example, $\{x_n\}_{n \in \{1,2,5\}}
= \{x_1, x_2, x_5\}$. We write vectors in bold font, e.g., $\mathbf{x} = (x_1,
x_2, x_3) \in \mathbb{F}^3$, while matrices are denoted by capital letters in
regular font. Here is an example matrix: $A = \begin{bmatrix} 1 & 2 \\ 3 & 4
\end{bmatrix} \in \mathbb{F}^{2 \times 2}$. For additional linear algebra notation,
refer to \Cref{section:linear-algebra}, where we introduce the basics of linear algebra.

Finally, we denote the set $\{0,\dots,n-1\}$ for $n \in \mathbb{N}$ by $[n]$.
This notation is used very extensively in the book, so make sure to remember it.
For example, instead of writing $\sum_{k=0}^{n-1}k$ we also write $\sum_{k \in
[n]}k = \frac{n(n-1)}{2}$. Or, instead of specifying $n$ conditions
$z_1=f(x_0)$, $z_2=f(x_1)$, \dots, $z_n=f(x_{n-1})$ explicitly, we simply say
$z_{i+1} = f(x_i)$ for all $i \in [n]$.

\subsection*{Algorithms and Probabilities}

Suppose we specified the algorithm $A$ that takes some input $x$ and returns
some output $y$. We write $y \gets A(x)$ to denote that the algorithm $A$ is
executed on input $x$ and returns output $y$. To denote some party in the 
protocol we use calligraphic font, e.g., $\mathcal{P}$ typically denotes 
the prover while $\mathcal{V}$ denotes the verifier. Such parties are algorithms 
themselves, so we can easily write $y \gets \mathcal{P}(x)$ to denote that 
the prover, based on $x$, has outputted $y$.

To denote the probability of an event $E$ hapenning, we write $\Pr[E]$. For
example, if $E$ is an event of rolling a dice and getting a number $4$, then
$\Pr[E] = \frac{1}{6}$. To denote the random sampling from the finite set $X$,
we write $x \xleftarrow{R} X$. For example, $x \xleftarrow{R} \{1,2,3,4,5,6\}$
denotes the random sampling of $x$ from the set $\{1,2,3,4,5,6\}$.

To denote the probability of both $E$ and $F$ happening, we write $\Pr[E,F]$ (as
opposed to $E \cap F$). To denote the conditional probability, we write
$\Pr[E|F]$, which means the probability of $E$ happening given that $F$ has
already happened. In practice, we place the protocol execution in $F$, while $E$
is the event of the protocol outputting a certain result. For example,
\begin{equation*}
    \Pr\left[x \equiv 1 \; (\text{mod}\,3) \;|\; x \gets \mathsf{RollDice}(\cdot)\right] = \frac{1}{3}.
\end{equation*}

When event $F$ can be written concisely, we write it in the subindex of $\Pr$.
For instance, $\underset{x \xleftarrow{R} [6]}{\Pr}[x \equiv 0 \; (\text{mod}\,2)]=\frac{1}{2}$.

\end{document}
