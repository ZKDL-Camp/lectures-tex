\documentclass[../lecture-notes.tex]{subfiles}

\begin{document}

\subsection{Sigma Protocols}

\subsection{Schnorr's Identification Protocol}

One very useful protocol for demonstration purposes is Schnorr's identification protocol. It is a simple and elegant protocol that allows one party to prove to another party that it knows a discrete logarithm of a given element. 

Let us formalize it using theory above. Introduce the language 

Suppose $\mathbb{G}$ is a cyclic group of prime order $q$ with generator $g \in \mathbb{G}$. Suppose prover $\mathcal{P}$ has a secret key $\alpha \in \mathbb{Z}_q$ and the corresponding public key $u = g^{\alpha} \in \mathbb{G}$ and he wants to convince the verifier $\mathcal{V}$ that he knows $\alpha$ corresponding to the public key $u$. 

Well, the easiest way how to proceed is simply giving $\alpha$ to $\mathcal{V}$, but this is obviously not what we want. Instead, the Schnorr protocol allows $\mathcal{P}$ to prove the knowledge of $\alpha$ without revealing it. 

Let us finally describe the protocol. The schnorr identification protocol $\Pi_{\text{Schnorr}} = (\mathsf{Gen}, \mathcal{P}, \mathcal{V})$ with a generation function $\mathsf{Gen}$ and prover $\mathcal{P}$ and verifier $\mathcal{V}$ is defined as follows:
\begin{itemize}
    \item $\mathsf{Gen}(1^{\lambda})$: As with most public-key cryptosystems, we take $\alpha \xleftarrow{R} \mathbb{Z}_q$ and $u \gets g^{\alpha}$. We output the \textit{verification key} as $\mathsf{vk} := u$, and the \textit{secret key} as $\mathsf{sk} := \alpha$.
    \item The protocol between $(\mathcal{P},\mathcal{V})$ is run as follows:
    \begin{itemize}
        \item $\mathcal{P}$ computes $\alpha_T \gets \mathbb{Z}_q, u_T \gets g^{\alpha_T}$ and sends $u_T$ to $\mathcal{V}$.
        \item $\mathcal{V}$ sends a random challenge $c \xleftarrow{R} \mathbb{Z}_q$ to $\mathcal{P}$.
        \item $\mathcal{P}$ computes $\alpha_C \gets \alpha_T + \alpha c \in \mathbb{Z}_q$ and sends $\alpha_C$ to $\mathcal{V}$.
        \item $\mathcal{V}$ accepts if $g^{\alpha_C} = u_T \cdot u^c$, otherwise it rejects.
    \end{itemize}
\end{itemize}



\end{document}