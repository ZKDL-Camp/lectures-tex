\documentclass{zkdl-presentation-template}

\title[Plonk Arithmetization]{\textbf{Plonk Arithmetization}}
\author{Distributed Lab}
\date{January 09, 2025}
\homepage{zkdl-camp.github.io}
\github{ZKDL-Camp}

\begin{document}
    \frame {
        \tikz [remember picture,overlay]
        \node at
        ([yshift=1.5cm,xshift=-1.5cm]current page.south east)
    %or: (current page.center)
            {\includegraphics[width=60pt]{images/logo.png}};

        \titlepage
    }

    \begin{frame}{Plan}
        \tableofcontents
    \end{frame}

    \section{Introduction. PlonK: Five Ws}

    \begin{frame}{What is PlonK?}
        \textbf{PlonK is a type of zkSNARK:}
        \begin{itemize}
            \item Groth16
            \item Halo2
            \item Marlin
            \item \textbf{PlonK}
            \item $\dots$
        \end{itemize}
    \end{frame}

    \begin{frame}{Who and When invented Plonk?}
        Ariel Gabizon, Zachary Williamson, Oana Ciobotaru introduced paper ``PLONK: Permutations over Lagrange-bases for Oecumenical Noninteractive arguments of Knowledge'' in 2019

        \begin{figure}
            \centering
            \includegraphics[width=\textwidth]{images/lecture_12/plonk.png}
            \caption{PlonK Paper. Date in Paper reflects the last update :)}
        \end{figure}
    \end{frame}

    \begin{frame}{Why use Plonk?}
        \textbf{Focus on what you want:}
        \begin{itemize}
            \item ZKP for different tasks?
            \item Efficient proving times?
            \item Small-medium proof sizes?
            \item Flexibility?
        \end{itemize}
        \begin{figure}
            \centering
            \includegraphics[width=0.4\textwidth]{images/lecture_12/goals.jpg} 
        \end{figure}
    \end{frame}

    \begin{frame}{Where Plonk is used?}
        \centering
        \textbf{zkVMs love Plonk!}
        \begin{itemize}
            \centering
            \item Aztek Protocol (Noir)
            \item zkSync
            \item Dusk Network
            \item Mina Protocol
        \end{itemize}
        \begin{figure}
            \centering
            \includegraphics[width=0.8\textwidth]{images/lecture_12/plonk-projects.jpg} 
        \end{figure}
    \end{frame}

    \section{PlonK Arithmetization}

    \begin{frame}{Arithmetization}
        \textbf{Goal:} Write some program (computation) into math processing-prone form.
        \begin{example}
            \textbf{Public Input}: $x \in \mathbb{F}$ \\
            \textbf{Private Input}: $e \in \mathbb{F}$ \\
            \textbf{Output}: $e \times x + x - 1$ \\
        \end{example}
        \pause Let's split our program into the sequence of gates with left, right operands and output - circuit.
        \begin{example}
            We need \textbf{three gates} to encode our program:
            \\
            \begin{enumerate}
                \item \textbf{Gate \#1}: left $e$, right $x$, output \(u = e \times x\)\pause
                \item \textbf{Gate \#2}: left $u$, right $x$, output \(\upsilon = u + x\)\pause
                \item \textbf{Gate \#3}: left $\upsilon$, right $x$, output \(w = \upsilon + (-1)\)
            \end{enumerate}
        \end{example}
    \end{frame}

    \begin{frame}{Execution Trace}
        Then, form  \textit{execution trace table} --- a matrix $T$ with columns $L$, $R$ and $O$.
        \begin{example}
            \begin{center}
                \begin{tabular}{|c|c|c|}
                    \hline
                    A & B & C \\ \hline
                    2 & 3 & 6 \\ \hline
                    6 & 3 & 9 \\ \hline
                    9 & \xmark & 8 \\ \hline
                \end{tabular}
            \end{center}
        \end{example}
        \pause \begin{block}{Remark}
            Notice how the last row has no value in $B$ column (marked by \xmark) --- this is reasoned by the fact it is not a variable, but rather a constant, meaning it doesn't depend on execution.
        \end{block}
    \end{frame}

    \begin{frame}{Encode the Program}
        Suppose you were given random matrix $T$. How could you tell if it is suitable for your circuit?
        \begin{block}{Solution}
            Encode the circuit. Check $T$ using encoding:
            \begin{enumerate}
                \item \textbf{Gates} (gate constraints) --- using matrix $Q$.
                \item \textbf{Wires} (copy constraints) --- using matrix $V$.
            \end{enumerate}
        \end{block}

        \pause \begin{definition}[Gate Matrix]
            $Q$ matrix has one row per each gate with columns $Q_L$, $Q_R$, $Q_O$, $Q_M$, $Q_C$. If columns $A$, $B$ and $C$ of the execution trace table form valid evaluation of the circuit, 
            \[A_i Q_{Li} + B_i Q_{Ri} + A_i B_i Q_{Mi} + C_i Q_{Oi} + Q_{ci} = 0\]
        \end{definition}
    \end{frame}

    \begin{frame}{$Q$ Matrix}
        \begin{example}
            For our program, we would have a following $Q$ table:
            \begin{center}
                \begin{tabular}{|c|c|c|c|c|}
                    \hline
                    $Q_L$ & $Q_R$ & $Q_M$ & $Q_O$ & $Q_c$ \\ 
                    \hline
                    \textcolor{blue}{0} & \textcolor{blue}{0} & \textcolor{blue}{1} & \textcolor{blue}{$-1$} & \textcolor{blue}{0} \\ 
                    \hline
                    \textcolor{blue}{1} & \textcolor{blue}{1} & \textcolor{blue}{0} & \textcolor{blue}{$-1$} & \textcolor{blue}{0} \\ 
                    \hline
                    \textcolor{blue}{1} & \textcolor{blue}{0} & \textcolor{blue}{0} & \textcolor{blue}{$-1$} & \textcolor{blue}{$-1$} \\ 
                    \hline
                \end{tabular}
            \end{center}

            You can verify that our claim holds for aforementioned trace matrix:
            \begin{align*}
              2 \times \textcolor{blue}{0} + 3 \times \textcolor{blue}{0} + 2 \times 3 \times \textcolor{blue}{1} + 6 \times \textcolor{blue}{(-1)} + \textcolor{blue}{0} &= 0 \\
              6 \times \textcolor{blue}{1} + 3 \times \textcolor{blue}{1} + 6 \times 3 \times \textcolor{blue}{0} + 9 \times \textcolor{blue}{(-1)} + \textcolor{blue}{0} &= 0 \\
              9 \times \textcolor{blue}{1} + 0 \times \textcolor{blue}{0} + 9 \times 0 \times \textcolor{blue}{0} + 8 \times \textcolor{blue}{(-1)} + \textcolor{blue}{(-1)} &= 0 
            \end{align*}
        \end{example}
    \end{frame}

    \begin{frame} {$V$ Matrix}
        \begin{definition}
            $V$ consists of indices of all inputs and intermediate values, so that if $T$ is a valid trace,
            \[\forall i, j, k, l: (V_{i,j} = V_{k,l}) \Rightarrow (T_{i,j} = T_{k,l})\]
        \end{definition}
        \begin{example}
            For our program, $V$ would look like following:
            \begin{center}
                \begin{tabular}{|c|c|c|}
                    \hline
                    L & R & O \\
                    \hline
                    0 & 1 & 2 \\
                    \hline
                    2 & 1 & 3 \\
                    \hline
                    3 &  & 4 \\
                    \hline
                \end{tabular}
            \end{center}
            Here $0$ is an index of $e$, $1$ is an index of $x$, 2 --- $u$, 3 --- $\upsilon$ and $4$ --- output $w$.
        \end{example}
    \end{frame}

    \begin{frame} {Custom Gates}
        Default Plonk: \textbf{addition} and \textbf{multiplication} gates.
        How to make it more \textit{interesting}?
        \begin{block}{Solution}
            $Q$ with it's 5 columns already allows for custom gates, however it is possible to include out own columns.
        \end{block}

        \begin{example}
            Our entire program may be encoded as one custom gate.
            \begin{center}
            \scriptsize
            $Q$: 
                \begin{tabular}{|c|c|c|c|c|}
                    \hline
                    $Q_L$ & $Q_R$ & $Q_M$ & $Q_o$ & $Q_c$ \\ 
                    \hline
                    \textcolor{blue}{0} & \textcolor{blue}{1} & \textcolor{blue}{1} & \textcolor{blue}{$-1$} & \textcolor{blue}{$-1$} \\ 
                    \hline
                \end{tabular}
            \quad $V$:
                \begin{tabular}{|c|c|c|}
                    \hline
                    L & R & O \\
                    \hline
                    0 & 1 & 2 \\
                    \hline
                \end{tabular}
            \quad $T$:
                \begin{tabular}{|c|c|c|}
                    \hline
                    A & B & C \\
                    \hline
                    \textcolor{purple}{2} & \textcolor{purple}{3} & \textcolor{purple}{8} \\
                    \hline
                \end{tabular}
            \end{center}

            \begin{equation*}
                \textcolor{purple}{2} \times \textcolor{blue}{0} + \textcolor{purple}{3} \times \textcolor{blue}{1} + \textcolor{purple}{2} \times \textcolor{purple}{3} \times \textcolor{blue}{1} + \textcolor{purple}{8} \times \textcolor{blue}{(-1)} + \textcolor{blue}{(-1)} = 0
            \end{equation*}
        \end{example}
    \end{frame}

    \begin{frame}{Public Inputs}
        Also need to encode public inputs.
        \begin{block}{Idea}
            Consider, as if we added new \textit{selector} rows to $Q$ and tied them in $V$ and $T$.
        \end{block}

        \begin{example}
            \scriptsize
            \begin{center}
            $Q$: 
                \begin{tabular}{|c|c|c|c|c|}
                    \hline
                    $Q_L$ & $Q_R$ & $Q_M$ & $Q_o$ & $Q_c$ \\ 
                    \hline
                    -1 & 0 & 0 & 0 & 3 \\ 
                    \hline
                    -1 & 0 & 0 & 0 & 8 \\ 
                    \hline
                    1 & 1 & 1 & -1 & 1 \\ 
                    \hline
                    1 & -1 & 0 & 0 & 0 \\ 
                    \hline
                \end{tabular}
            \quad $V$:
                \begin{tabular}{|c|c|c|}
                    \hline
                    L & R & O \\
                    \hline
                    0 &  &  \\
                    \hline
                    1 &  &  \\
                    \hline
                    2 & 0 & 3 \\
                    \hline
                    1 & 3 &  \\
                    \hline
                \end{tabular}
             \quad $T$:
                \begin{tabular}{|c|c|c|}
                    \hline
                    A & B & C \\
                    \hline
                    3 &  &  \\
                    \hline
                    8 &  &  \\
                    \hline
                    2 & 3 & 8 \\
                    \hline
                    8 & 8 &  \\
                    \hline
                \end{tabular}
            \end{center}
        \end{example}

        \pause Now $Q$ and $V$ are not independent of evaluations. 
        \begin{block}{Solution}
            We introduce another one-column matrix named $\Pi$ (public inputs).
        \end{block}
    \end{frame}

    \begin{frame}{Wrap-Up}
        \begin{example}
            With only $Q$ modified, we have:
            \centering 
            \begin{center}
                \begin{tabular}{|c|}
                    \hline
                    $\Pi$ \\ 
                    \hline
                    3 \\ 
                    \hline
                    8 \\ 
                    \hline
                    0 \\ 
                    \hline
                    0 \\ 
                    \hline
                \end{tabular} \hspace{1cm}
            $Q$: 
                \begin{tabular}{|c|c|c|c|c|}
                    \hline
                    $Q_L$ & $Q_R$ & $Q_M$ & $Q_o$ & $Q_c$ \\ 
                    \hline
                    -1 & 0 & 0 & 0 & 0 \\ 
                    \hline
                    -1 & 0 & 0 & 0 & 0 \\ 
                    \hline
                    1 & 1 & 1 & -1 & 1 \\ 
                    \hline
                    1 & -1 & 0 & 0 & 0 \\ 
                    \hline
                \end{tabular}
            \end{center}
        \end{example}

        \pause\begin{definition}[Interim Summary]
            Matrix $T$ with columns $A$, $B$ and $C$ encodes correct execution of the program, if the following two conditions hold:
            \begin{enumerate}
                \item \(\forall i: A_i Q_{Li} + B_i Q_{Ri} + A_i B_i Q_{Mi} + C_i Q_{Oi} + Q_{ci} + \Pi_i = 0\)
                \item \(\forall i, j, k, l: (V_{i,j} = V_{k,l}) \Rightarrow (T_{i,j} = T_{k,l})\)
            \end{enumerate}
        \end{definition}
    \end{frame}

    \section{Polynomial Form}
    \subsection{Formulating Conditions}
    \begin{frame}{Matrices to Polynomials}
        Encode matrices into a few equations on polynomials. 

        Let $\omega$ be a primitive $N$-th root of unity and let $\Omega = \{\omega^j: 0 \le j < N\}$. 
        Let \(a, b, c, q_L, q_R, q_M, q_O, q_C, \pi\) be polynomials of degree at most $N$ that
        interpolate corresponding columns from matrices at the domain $\Omega$. This means,
        that \(\forall j: a(\omega^j) = A_j\) and the same for other columns.

        \pause\begin{block}{Proposition}
            Now we can reduce down our first condition to checking valid execution trace
            into the following claim over polynomials:
            \[\exists t \in \mathbb{F}[X]: \; aq_L + bq_R + abq_M + cq_O + q_C + \pi = z_{\Omega} t,\]
            where $z_{\Omega}(X)$ is the vanishing polynomial $X^N - 1$.
        \end{block}
    \end{frame}

    \begin{frame} {Copy constraints in polynomial form.}
        Spoiler: we can use the concept of permutation to encode $V$ wirings.
    
        A permutation is a rearrangement of the set: 
        \[\mathcal{I} = \{(i, j) : \text{such that } 0 \leq i < N, \text{ and } 0 \leq j < 3\}\]

        Permutation of the set is commonly denoted as $\sigma$.
    \end{frame}

    \begin{frame} {Copy constraints in polynomial form.}
        \begin{example}
            The matrix $V$ induces a permutation of this set where $\sigma((i,j))$ is equal to the indices of the next occurrence of the value at position $(i,j)$. So, for our example:
            
            \begin{center}
                \quad $V$:
                \begin{tabular}{|c|c|c|}
                    \hline
                    L & R & O \\
                    \hline
                    0 &  &  \\
                    \hline
                    1 &  &  \\
                    \hline
                    2 & 0 & 3 \\
                    \hline
                    1 & 3 &  \\
                    \hline
                \end{tabular}
            \end{center}

            \[\sigma((0,0)) = (2,1), \sigma((0,1)) = (0,3), \sigma((0,2)) = (0,2)\]
            \[\sigma((0,3)) = (0,1), \sigma((2,1)) = (0,0), \sigma((3,1)) = (2,2)\]

        \end{example}
    \end{frame}

    \subsection{Permutation Check}
    \begin{frame}
        \textcolor{blue!80!black}{\textbf{Permutation Check.}} Having defined permutation, we can now reduce a condition $2$ of valid execution trace matrix into the following check:
        \[\forall (i, j) \in \mathcal{I}: T_{i,j} = T_{\sigma(i,j)}\]
        
        You may have noticed how this can be reformulated as equality of $A$ and $B$:
        \[A = \{((i, j), T_{i,j}) : (i, j) \in \mathcal{I}\}\]
        \[B = \{(\sigma((i, j)), T_{i,j}) : (i, j) \in \mathcal{I}\}\]
        
        We can reduce this check down to polynomial equations.        
    \end{frame}

    \begin{frame}
        Suppose we have sets \(A = \{a_0, a_1\}\) and \(B = \{b_0, b_1\}\). We can consider polynomials \(A' = \{a_0 + X, a_1 + X\}, B' = \{b_0 + X, b_1 + X\}\). 
        \\
        So, \(A' = B'\), only if \((a_0 + X)(a_1 + X) = (b_0 + X)(b_1 + X)\). This is
        true because of linear polynomial unique factorization property, working as
        prime factors. Now, we can utilize Schwartz-Zippel lemma to replace the latter
        formula with \((a_0 + \gamma)(a_1 + \gamma) = (b_0 + \gamma)(b_1 + \gamma)\) for
        some random $\gamma$ with overwhelming probability. If we wish to generalize this
        for arbitrary sets \(A = \{a_0, \ldots, a_{k-1}\}\) and \(B = \{b_0, \ldots, b_{k-1}\}\), apply the following check:

        \[\prod_{i=0}^{k-1} (a_i + \gamma) = \prod_{i=0}^{k-1} (b_i + \gamma)\]
    \end{frame}

    \begin{frame}
        Let $\Omega$ be a domain of the form \(\{1, \omega, \dots, \omega^{k-1}\}\) for some $k$-th root of unity $\omega$. Let $f$ and $g$ be polynomials that interpolate the following values at $\Omega$:

        \[f: (a_0 + \gamma, \ldots, a_{k-1} + \gamma)\]
        \[g: (b_0 + \gamma, \ldots, b_{k-1} + \gamma)\]

        Then \(\prod_{i=0}^{k-1} (a_i + \gamma) = \prod_{i=0}^{k-1} (b_i + \gamma)\) if and only if exists a polynomial $Z \in \mathbb{F}[X]$ such that for all $h \in \Omega$ we have $Z(\omega^{0}) = 1$ and $Z(h)f(h) = g(h)Z(\omega h)$.
    \end{frame}

    \begin{frame}
        Now that we can encode equality of sets of field elements, let's expand this to sets of tuples of field elements. Let \(A = \{(a_0, a_1), (a_2, a_3)\}\) and \(B = \{(b_0, b_1), (b_2, b_3)\}\), then, similarly:
        \[A' = \{a_0 + a_1Y + X, a_2 + a_3Y + X\}\]
        \[B' = \{b_0 + b_1Y + X, b_2 + b_3Y + X\}\]
        \[A = B \leftrightarrow A' = B'\]

        As before, we can leverage Schwartz-Zippel lemma to reduce this down into sampling random $\beta$ and $\gamma$ and checking equality of:
        \[(a_0 + \beta a_1 + \gamma)(a_2 + \beta a_3 + \gamma) = (b_0 + \beta b_1 + \gamma)(b_2 + \beta b_3 + \gamma)\]
    \end{frame}

    \begin{frame}
        Let's make $(i, j)$ into one field element, so that we can use statement above for encoding.
        
        Recall that $i \in [0; N-1]$ and $j \in [0; 2]$; we can take $3N$-th primitive root of unity $\eta$ and define our field element as \(a_0 = \eta^{3i + j}\):
        \[A = \{(\eta^{3i+j}, T_{i,j}) : (i, j) \in \mathcal{I}\}\]
        \[B = \{(\eta^{3k+l}, T_{i,j}) : (i, j) \in \mathcal{I}, \sigma((i, j)) = (k, l)\}\]        
    \end{frame}

    \begin{frame}
        Let $\eta$ be a $3N$-th primitive root of unity, $\beta$ and $\gamma$ random field elements. Let $\mathcal{D} = \{1, \eta, \eta^2, \ldots, \eta^{3N-1}\}$. Then let $f$ and $g$ interpolate at $\mathcal{D}$:
        \begin{align*}
            f: & \{T_{i,j} + \eta^{3i+j}\beta + \gamma : (i, j) \in \mathcal{I}\} \\
            g: & \{T_{i,j} + \eta^{3k+l}\beta + \gamma : (i, j) \in \mathcal{I}, \sigma((i, j)) = (k, l)\}
        \end{align*}
        
        So, $\exists Z \in \mathbb{F}[X]$ s.t. $\forall h \in \Omega$ we have $Z(\eta^{0}) = 1$ and $Z(h)f(h) = g(h)Z(\eta h) \leftrightarrow A = B$ w.h.p.        
    \end{frame}

    \begin{frame}
        Notice, that $\omega = \eta^3$ is a primitive $N$-th root of unity. Let \(\Omega = \{1, \omega, \omega^2, \ldots, \omega^{N-1}\}\). 
        We will define three polynomials, which interpolate following sets:
        \[S_{\sigma 1}: \{\eta^{3k+l} : (i, 0) \in \mathcal{I}, \sigma((i, 0)) = (k, l)\}\]
        \[S_{\sigma 2}: \{\eta^{3k+l} : (i, 1) \in \mathcal{I}, \sigma((i, 0)) = (k, l)\}\]
        \[S_{\sigma 3}: \{\eta^{3k+l} : (i, 2) \in \mathcal{I}, \sigma((i, 0)) = (k, l)\}\]
    \end{frame}
    
    \begin{frame} {Copy constraints via polynomials}
        Let $\omega$ be an $N$-th root of unity. Let $\Omega = \{1, \omega, \omega^2, \ldots, \omega^{N-1}\}$. Let $k_1$ and $k_2$ be two field elements such that $\omega^i \neq \omega^j k_1 \neq \omega^l k_2$ for all $i, j, l$. Let $\beta$ and $\gamma$ be random field elements. Let $f$ and $g$ be the polynomials that interpolate, respectively, the following values at $\Omega$:
        \[f: \left\{\left(T_{0,j} + \omega^i \beta + \gamma\right)\left(T_{1,j} + \omega^i k_1 \beta + \gamma\right)\left(T_{2,j} + \omega^i k_2 \beta + \gamma\right) : 0 \leq i < N\right\}\]
        \[g: \left\{\left(T_{0,j} + S_{0,1}(\omega^i) \beta + \gamma\right)\left(T_{0,j} + S_{0,2}(\omega^i) \beta + \gamma\right)\left(T_{0,j} + S_{0,3}(\omega^i) \beta + \gamma\right) : 0 \leq i < N\right\}\]
        
        So, $\exists Z \in \mathbb{F}[X]$ such that $\forall d \in \mathcal{D}$ we have $Z(\omega^{0}) = 1$ and $Z(d)f(d) = g(d)Z(\omega d) \leftrightarrow A = B$ w.h.p.
    \end{frame}

    \begin{frame} {Summary | Matrices}
        \begin{definition}
            Let $T$ be a $N \times 3$ matrix with columns $A, B, C$ and $\Pi$ a $N \times 1$
            matrix where $N$ is the number of gates. They correspond to a valid execution
            instance with public input given by $\Pi$ if and only if:
            
            \begin{enumerate}
                \item \(\forall i: A_iQ_{Li} + B_iQ_{Ri} + A_iB_iQ_{Mi} + C_iQ_{Oi} + Q_{Ci} + \Pi_i = 0\)
                \item \(\forall i, j, k, l: V_{i,j} = V_{k,l} \implies T_{i,j} = T_{k,l}\)
                \item \(\forall i > n: \Pi_i = 0\)
            \end{enumerate}
        \end{definition}
    \end{frame}

    \begin{frame} {Summary | Polynomials}
        \begin{definition}
            Let $z_{\Omega} = X^N - 1$. Let $T$ be a $N \times 3$ matrix with columns $A, B,
            C$ and $\Pi$ a $N \times 1$ matrix. They correspond to a valid execution
            instance with public input given by $\Pi$ if and only if:
            
            \begin{enumerate}
                \item \(\exists t_1 \in \mathbb{F}[X]: aq_L + bq_R + abq_M + cq_O + q_C + \pi = z_{\Omega}t_1\)
                \item \(\exists t_2, t_3, z \in \mathbb{F}[X]: zf - gz' = z_{\Omega}t_2\) and \((z-1)L_1 = z_{\Omega}t_3\), where $z'(X) = z(X\omega)$.
            \end{enumerate}
        \end{definition}
    \end{frame}

    \begin{frame}[plain, standout]
        \centering
        \LARGE
        \textbf{Thank you for your attention} \\

        \vspace{0.2cm} \Huge \ding{170} \large \\

        \vspace{1cm}

        \href{https://zkdl-camp.github.io/}{\raisebox{-.1em}{\hspace{.025em}\faIcon{globe}}\hspace{.325em}zkdl-camp.github.io} \\

        \href{https://github.com/ZKDL-Camp}{\raisebox{-.1em}{\hspace{.025em}\faIcon{github}}\hspace{.325em}github.com/ZKDL-Camp}

        \begin{center}
            \includegraphics[width=0.15\textwidth]{images/logo.png}
        \end{center}
    \end{frame}
\end{document}