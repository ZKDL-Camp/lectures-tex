\documentclass{zkdl-presentation-template}

\title[PlonK Proving and Verifying]{\textbf{PlonK Proving and Verifying}}
\author{Distributed Lab}
\date{January 21, 2025}
\homepage{zkdl-camp.github.io}
\github{ZKDL-Camp}

\begin{document}
    \frame {
        \tikz [remember picture,overlay]
        \node at
        ([yshift=1.5cm,xshift=-1.5cm]current page.south east)
    %or: (current page.center)
            {\includegraphics[width=60pt]{images/logo.png}};

        \titlepage
    }

    \begin{frame}{Plan}
        \tableofcontents
    \end{frame}

    \section{Gadgets}

    \begin{frame}{Gadgets - Commitments}
        \begin{enumerate}
            \item Commit($f$) $\rightarrow$ $com(f) \in G$
            \item Open($f, \zeta$) $\rightarrow$ $\pi \in G$
            \item Verify($com(f), \pi, \zeta, y$) $\rightarrow$ $Accept$ if $f(\zeta) = y$ w.h.p.
        \end{enumerate}
    \end{frame}

    \begin{frame}{Gadgets - Blindings}
        \textbf{Hide the polynomial} \\
        Let $a \in \mathbb{F}[X]$ be a polynomial of degree $N$. Select $M \geq N$.
        Then polynomial $a_{\text{blinded}}$ will be of degree $M$ and is defined as:

        \begin{align*}
            z_H &= X^N - 1 \\
            k &= M - N \\
            b_0, \dots, b_k &\xleftarrow{R} \mathbb{F} \\
            a_{\text{blinded}} &= (b_0 + b_1 X + \dots + b_k X^k) z_H + a
        \end{align*}
    \end{frame}

    \section{Setup}

    \begin{frame}{Transcript Setup}
        Arithmetization $\rightarrow$ 8 polynomials. \\ 
        \textbf{Add their commitments to transcript.}
        \vspace{1em}
        
        \textbf{1. Encoding of the copy constraints:}
        \begin{equation*}
            \mathsf{com}(S_{\sigma,1}), \quad \mathsf{com}(S_{\sigma,2}), \quad \mathsf{com}(S_{\sigma,3})
        \end{equation*}
        
        \textbf{2. Encoding of the gate constraints:}
        \begin{equation*}
            \mathsf{com}(q_L), \quad \mathsf{com}(q_R), \quad \mathsf{com}(q_M), \quad 
            \mathsf{com}(q_O), \quad \mathsf{com}(q_C)
        \end{equation*}
    \end{frame}    

    \section{Proving}

    \begin{frame} {Proving: Round 1}
        Interpolate polynomials $\textcolor{red}{a', b', c'}$ over corresponding columns of $T$.
        Sample random $\textcolor{blue}{b_1, b_2, b_3, b_4, b_5, b_6} \xleftarrow{R} \mathbb{F}$. \\
        \vspace{1em}
        Let the blinded polynomials be:
        \begin{align*}
        \centering
            \textbf{a} &:= (\textcolor{blue}{b_1} X + \textcolor{blue}{b_2})Z_{\Omega}(X) + \textcolor{red}{a'(X)} \\
            \textbf{b} &:= (\textcolor{blue}{b_3} X + \textcolor{blue}{b_4})Z_{\Omega}(X) + \textcolor{red}{b'(X)} \\
            \textbf{c} &:= (\textcolor{blue}{b_5} X + \textcolor{blue}{b_6})Z_{\Omega}(X) + \textcolor{red}{c'(X)}
        \end{align*}
        Add to the transcript commitments of computed above polynomials:
        \begin{equation*}
            \mathsf{com}(a), \quad \mathsf{com}(b), \quad \mathsf{com}(c)
        \end{equation*}
    \end{frame}

    \begin{frame} {Proving: Round 2}
        Sample random $\textcolor{blue}{\beta, \gamma} \xleftarrow{R} \mathbb{F}$. Let $z_0 = 1$. 
        $\forall k = 0, \dots, N:$\\
        \begin{equation*}
            z_{k+1} = z_{k} \times \textstyle\frac{(a_k + \textcolor{blue}{\beta} \omega^{k} + \textcolor{blue}{\gamma}) (b_k + \textcolor{blue}{\beta} \omega^{k}k_1 + \textcolor{blue}{\gamma}) (c_k + \textcolor{blue}{\beta} \omega^{k}k_2 + \textcolor{blue}{\gamma})}{(a_k + \textcolor{blue}{\beta} S_{\sigma,1}(\omega^{k}) + \textcolor{blue}{\gamma}) (b_k + \textcolor{blue}{\beta} S_{\sigma,2}(\omega^{k}) + \textcolor{blue}{\gamma}) (c_k + \textcolor{blue}{\beta} S_{\sigma,3}(\omega^{k}) + \textcolor{blue}{\gamma})}
        \end{equation*}

        Interpolate polynomial $\textcolor{red}{z'}$ over evaluations $(z_0, \dots, z_{N-1})$.

        Sample random $\textcolor{blue}{b_7, b_8, b_9} \xleftarrow{R} \mathbb{F}$. Compute:
        \begin{equation*}
            \textbf{z} := (\textcolor{blue}{b_7} X^2 + \textcolor{blue}{b_8} X + \textcolor{blue}{b_9})Z_H + z'
        \end{equation*}
        
        Add to the transcript $\mathsf{com}(z)$.
    \end{frame}

    \begin{frame} {Proving: Round 3}
        Sample random $\textcolor{blue}{\alpha} \xleftarrow{R} \mathbb{F}$.
        Interpolate $\textcolor{red}{\pi}$ over $\Pi$.
    
        \begin{align*}
        p_1 &= \textcolor{orange}{a}q_L + \textcolor{orange}{b}q_R + \textcolor{orange}{ab}q_M + \textcolor{orange}{c}q_o + q_C + \textcolor{red}{\pi} \\
        p_2 &= (\textcolor{orange}{a} + \textcolor{blue}{\beta} X + \textcolor{blue}{\gamma})(\textcolor{orange}{b} + \textcolor{blue}{\beta} k_1 X + \textcolor{blue}{\gamma})(\textcolor{orange}{ac} + \textcolor{blue}{\beta} k_2 X + \textcolor{blue}{\gamma})\textcolor{orange}{z} - \\
            &- (\textcolor{orange}{a} + \textcolor{blue}{\beta} S_{\sigma,1} + \textcolor{blue}{\gamma})(\textcolor{orange}{b} + \textcolor{blue}{\beta} S_{\sigma,2} + \textcolor{blue}{\gamma})(\textcolor{orange}{ac} + \textcolor{blue}{\beta} S_{\sigma,3} + \textcolor{blue}{\gamma})\textcolor{orange}{az}(\omega X) \\
        p_3 &= (\textcolor{orange}{az} - 1)L_1
        \end{align*}

        Define the composite polynomial:
        \begin{equation*}
            p = p_1 + \textcolor{blue}{\alpha} p_2 + \textcolor{blue}{\alpha}^2 p_3
        \end{equation*}
    \end{frame}

    \begin{frame} {Proving: Round 3}
        For $t'_{\text{lo}}, t'_{\text{mid}}, t'_{\text{hi}} \in \mathbb{F}^{\leq (N+1)}[X]$ polynomials of degree at most $N+1$: \\
        \vspace{1em}
        $t = t'_{\text{lo}} + X^{N+2}t'_{\text{mid}} + X^{2(N+2)}t'_{\text{hi}}$.
        Compute $t$ such that $p = tZ_{\Omega}$.
        
        Sample random $\textcolor{blue}{b_{10}, b_{11}} \xleftarrow{R} \mathbb{F}$. Define:
        \begin{align*}
            t_{\text{lo}} &= t'_{\text{lo}} + \textcolor{blue}{b_{10}}X^{N+2} \\
            t_{\text{mid}} &= t'_{\text{mid}} - \textcolor{blue}{b_{10}} + \textcolor{blue}{b_{11}}X^{N+2} \\
            t_{\text{hi}} &= t'_{\text{hi}} - \textcolor{blue}{b_{11}}
        \end{align*}

        Add to the transcript commitments: 
        \begin{equation*}
            \mathsf{com}(t_{\text{lo}}), \quad \mathsf{com}(t_{\text{mid}}), \quad \mathsf{com}(t_{\text{hi}}).
        \end{equation*}
    \end{frame}

    \begin{frame} {Proving: Round 4}
        Sample random $\textcolor{blue}{\zeta} \xleftarrow{R} \mathbb{F}$. \\
        \vspace{1em}
        Add to the transcript following evaluations:

        \begin{equation*}
            \textcolor{orange}{\bar{a}} = a(\zeta), \quad \textcolor{orange}{\bar{b}} = b(\zeta), \quad \textcolor{orange}{\bar{c}} = c(\zeta)
        \end{equation*}

        \begin{equation*}
            \textcolor{orange}{\bar{S}_{\sigma,1}} = S_{\sigma,1}(\zeta), \quad \textcolor{orange}{\bar{S}_{\sigma,2}} = S_{\sigma,2}(\zeta), \quad \textcolor{orange}{\bar{z}_{\omega}} = z(\zeta \omega)
        \end{equation*}
    \end{frame}

    \begin{frame} {Proving: Round 5}
        Sample random $v \xleftarrow{R}\mathbb{F}$. Let:

        \begin{align*}
            \hat{p}_{nc1} &= \textcolor{orange}{\bar{a}}q_L + \textcolor{orange}{\bar{b}}q_R + \textcolor{orange}{\bar{a}}\textcolor{orange}{\bar{b}}q_M + \textcolor{orange}{\bar{c}}q_o + q_C \\
            \hat{p}_{nc2} &= (\textcolor{orange}{\bar{a}} + \beta \textcolor{blue}{\zeta_1} + \textcolor{blue}{\gamma})(\textcolor{orange}{\bar{b}} + \beta k_1 \textcolor{blue}{\zeta_1} + \textcolor{blue}{\gamma})(\textcolor{orange}{\bar{c}} + \beta k_2 \textcolor{blue}{\zeta_1} + \textcolor{blue}{\gamma}) \textcolor{orange}{z} - \\
            &\quad - (\textcolor{orange}{\bar{a}} + \beta \textcolor{orange}{\bar{S}_{\sigma_1}} + \textcolor{blue}{\gamma})(\textcolor{orange}{\bar{b}} + \beta \textcolor{orange}{\bar{S}_{\sigma_2}} + \textcolor{blue}{\gamma})(\textcolor{orange}{\bar{c}} + \beta \textcolor{orange}{\bar{S}_{\sigma_3}} + \textcolor{blue}{\gamma}) \textcolor{orange}{z}(\omega \textcolor{blue}{\zeta_1}) \\
            \hat{p}_{nc3} &= L_1(\textcolor{blue}{\zeta_1})\textcolor{orange}{z}
        \end{align*}
     
        Define:
        \begin{align*}
            p_{nc} &= p_{nc1} + \alpha p_{nc2} + \alpha^2 p_{nc3} \\
            t_{partial} &= t_{lo} + \zeta^{N+2}t_{mid} + \zeta^{2(N+2)}t_{hi}
        \end{align*}
    \end{frame}

    \begin{frame} {Proving: Round 5}
        Define:
        \begin{equation*}
            f_{batch} = t_{partial} + \upsilon p_{nc} + \upsilon^2 a + \upsilon^3 b + \upsilon^4 c + \upsilon^5 S_{o1} + \upsilon^6 S_{o2}
        \end{equation*}

        \begin{definition}
            $\pi_{batch}$ - opening proof at $\zeta$ of $f_{batch}$. \\
            $\pi_{single}$ - opening proof at $\zeta\omega$ of $z$.
        \end{definition}

        Compute:
        \begin{align*}
            \hat{p_{nc}} &:= p_{nc}{\zeta} \\
            \hat{t} &:= t{\zeta}
        \end{align*}
    \end{frame}

    \begin{frame} {Proof}
        \begin{definition}
            \textbf{PlonK} proof consists of the following values:
            \begin{align*}
                com(a), com(b), com(c), com(z), \\ 
                com(t_{lo}), com(t_{mid}), com(t_{hi}), \\ 
                \bar{a}, \bar{b}, \bar{c}, \bar{S}_{o1}, \bar{S}_{o2}, \bar{z}_w, \\
                \pi_{batch}, \pi_{single}, \bar{p}_{nc}, \bar{t}
            \end{align*}
        \end{definition}
    \end{frame}

    \section{Verifying}

    \begin{frame} {Verifier computes all challenges:}
        \begin{enumerate}
            \item Add com($a$), com($b$), com($c$) to the transcript.
            \item Sample two challenges \textcolor{blue}{$\beta$, $\gamma$}.
            \item Add com($z$) to the transcript.
            \item Sample a challenge \textcolor{blue}{$\alpha$}.
            \item Add com($t_{lo}$), com($t_{mid}$), com($t_{hi}$) to the transcript.
            \item Sample a challenge \textcolor{blue}{$\zeta$}.
            \item Add $\bar{a}$, $\bar{b}$, $\bar{c}$, $\bar{S}_{o1}$, $\bar{S}_{o2}$, $\bar{z}_w$ to the transcript.
            \item Sample a challenge \textcolor{blue}{$\upsilon$}.
        \end{enumerate}
    \end{frame}

    \begin{frame} {Compute $pi(\zeta)$}
        \begin{block}{Observation}
            We don't need to compute whole $pi$, only one evaluation.
        \end{block}

        \begin{equation*}
            pi(\zeta) = \sum_{i=0}^{n} L_{i}(\zeta) \Pi_i
        \end{equation*}
        
        Where $n$ is the number of public inputs and $L_i$ is the Lagrange basis.
    \end{frame}

    \begin{frame} {Compute claimed $p(\zeta)$}
        Compute: 
        \[\bar{p}_c = p_1(\zeta) + \alpha z_{w} \left( \bar{c} + \gamma \right) \left( \bar{a} + \beta \bar{S}_{\sigma_1} + \gamma \right) \left( \bar{b} + \beta \bar{S}_{\sigma_2} + \gamma \right) - \alpha^2 L_1(\zeta)\]
        This is the constant part of the linearization of $p$. So, adding it to what the prover claims to be $\bar{p}_{nc}$, he obtains $p(\zeta) = \bar{p}_c + \bar{p}_{nc}$.
    \end{frame}

    \begin{frame} {Compute com($t_{partial}$) and com($p_{nc}$)}
        He computes these of the commitments in the proof as follows:
        \[com(t_{partial}) = com(t_{lo}) + \zeta^{N+2}com(t_{mid}) + \zeta^{2(N+2)}com(t_{hi})\]
        For the second one, compute those:
        \begin{align*}
            \hat{p}_{nc1} &= \bar{a} * com(q_L) + \bar{b} * com(q_R) + (\bar{a}\bar{b}) * com(q_M) +  \\
                        &+ \bar{c} * com(q_o) + com(q_C) \\
            \hat{p}_{nc2} &= (\bar{a} + \beta \zeta_1 + \gamma)(\bar{b} + \beta k_1 \zeta_1 + \gamma)(\bar{c} + \beta k_2 \zeta_1 + \gamma) * com(z) - \\ 
            &- (\bar{a} + \beta \bar{S}_{\sigma_1} + \gamma)(\bar{b} + \beta \bar{S}_{\sigma_2} + \gamma)(\bar{c} + \beta \bar{S}_{\sigma_3} + \gamma) * com(z)(\omega \zeta_1) \\
            \hat{p}_{nc3} &= L_1(\zeta_1) * com(z)
        \end{align*}
        Then: \[com(p_{nc}) = com(p_{nc1}) + com(p_{nc2}) + com(p_{nc3})\].
    \end{frame}

    \begin{frame} {Compute claimed value $f_{batch}(\zeta)$ and $com(f_{batch})$}
        \begin{align*}
            f_{batch}(\zeta) &= \bar{t} + \upsilon \bar{p}_{nc} + \upsilon^2 \bar{a} + \upsilon^3 \bar{b} + \upsilon^4 \bar{c} + \upsilon^5 \bar{S}_{o1} + \upsilon^6 \bar{S}_{o2} \\
            com(f_{batch}) &= com(t_{partial}) + \upsilon * com(p_{nc}) + \upsilon^2 * com(a) + \\
            &+ \upsilon^3 * com(b) + \upsilon^4 * com(c) + \\
            &+ \upsilon^5 * com(S_{o1}) + \upsilon^6 * com(S_{o2})
        \end{align*}
    \end{frame}

    \begin{frame} {Proof check}
        Now the verifier has all the necessary values to proceed with the checks.
        \begin{itemize}
            \item Check that $p(\zeta)$ equals $(\zeta^N - 1)t(\zeta)$.
            \item Verify the opening of $f_{batch}$ at $\zeta$. That is, check that 
                \begin{equation*}
                    \text{Verify}([f_{batch}], \pi_{batch}, \zeta, f_{batch}(\zeta)) \text{ outputs Accept.}
                \end{equation*}
            \item Verify the opening of $z$ at $\zeta_w$. That is, check the validity of the proof $\pi_{single}$ using the commitment $com(z)$ and the value $\bar{z}_w$. 
                \begin{equation*}
                    \text{That is, check that } \text{Verify}(com(z), \pi_{single}, \zeta_w, \bar{z}_w) \text{ outputs Accept.}
                \end{equation*}
        \end{itemize}
    \end{frame}

\end{document}