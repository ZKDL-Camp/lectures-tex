\documentclass{zkdl-presentation-template}

\title[Plonk Arithmetization]{\textbf{Plonk Arithmetization}}
\author{Distributed Lab}
\date{January 09, 2025}
\homepage{zkdl-camp.github.io}
\github{ZKDL-Camp}

\begin{document}
    \frame {
        \tikz [remember picture,overlay]
        \node at
        ([yshift=1.5cm,xshift=-1.5cm]current page.south east)
    %or: (current page.center)
            {\includegraphics[width=60pt]{images/logo.png}};

        \titlepage
    }

    \begin{frame}{Plan}
        \tableofcontents
    \end{frame}

    \section{Multiplicative Subgroup. Primitive Roots}

    \begin{frame}{Motivation}
        In the Groth16, recall that we needed to interpolate expressions in the
        following form
        \begin{equation*}
            P(i) = a_i, \quad a_i \in \mathbb{F}, \quad i = 1,\dots,N
        \end{equation*}

        \begin{block}{Recall}
            The interpolation formula in given by:
            \begin{equation*}
                P(x) = \sum_{i=1}^{N} a_i \cdot \ell_i(x), \quad \ell_i(x) = \prod_{j=1, j \neq i}^{N} \frac{x - j}{i - j}
            \end{equation*}

            The complexity of this formula is $\mathcal{O}(N^2)$. But can we do better?
        \end{block}
    \end{frame}

    \begin{frame}{Multiplicative Subgroup.}
        We know that $\mathbb{F}_p$ is a \textbf{field}: we have a usual arithmetic $+,\times$.

        \begin{alertblock}{Question}
            Does $(\mathbb{F}_p, \times)$ form a group?
        \end{alertblock}

        No, since $0$ does not have an inverse. But, if we consider
        $(\mathbb{F}_p \setminus \{0\}, \times)$, we do have a group structure!

        \begin{definition}
            A \textbf{multiplicative group} of a finite field $\mathbb{F}$, denoted as $\mathbb{F}^{\times}$, is a multiplicative group $(\mathbb{F} \setminus \{0\}, \times)$.
        \end{definition}

        \begin{block}{Number of Elements}
            The number of elements in $\mathbb{F}^{\times}_p$ is $p - 1$.
        \end{block}
    \end{frame}

    \begin{frame}{Primitive Root}
        \begin{theorem}
            Multiplicative group of a finite field $\mathbb{F}^{\times}$ is cyclic. The generators $\omega$ of 
            this group are called \textbf{primitive roots}.
        \end{theorem}

        \begin{example}
            $\omega=3$ is the primitive root of $\mathbb{F}_7$. Indeed,
            \begin{align*}
                3^1 = 3, \quad 3^2 = 2, \quad 3^3 = 6, \quad 3^4 = 4, \quad 3^5 = 5, \quad 3^6 = 1.
            \end{align*} 

            Clearly, $\langle \omega \rangle = \mathbb{F}_7^{\times}$.
        \end{example}
    \end{frame}
\end{document}