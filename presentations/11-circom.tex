\documentclass{zkdl-presentation-template}

\title[Circom]{\textbf{Circom}}
\author{Distributed Lab}
\date{December 05, 2024}
\homepage{zkdl-camp.github.io}
\github{ZKDL-Camp}

\begin{document}
    \frame {
        \tikz [remember picture,overlay]
        \node at
        ([yshift=1.5cm,xshift=-1.5cm]current page.south east)
    %or: (current page.center)
            {\includegraphics[width=60pt]{images/logo.png}};

        \titlepage
    }

    \begin{frame}{Plan}
        \tableofcontents
    \end{frame}


    \section{Introduction}

    \begin{frame}{Why do we need ZK?}
        \pause

        \begin{block}{Option}
            Solution to privacy
        \end{block}

        \pause

        \begin{example}
            \begin{enumerate}
                \item \textit{I know the private key that corresponds to this public key}
                \item \textit{I know a private key that corresponds to a public key from this list}
            \end{enumerate}
        \end{example}

        \pause

        \begin{block}{Option}
            Solution to scalability
        \end{block}

        \pause

        \begin{example}
            \textit{This is the hash of a blockchain block that does not produce negative balances}
        \end{example}
    \end{frame}

    \begin{frame}{Using ZKP}
        \only<1>{\includegraphics[width=\linewidth]{images/lecture_11/process_start}}
        \only<2>{\includegraphics[width=\linewidth]{images/lecture_11/process_1}}
        \only<3>{\includegraphics[width=\linewidth]{images/lecture_11/process_2}}
        \only<4>{\includegraphics[width=\linewidth]{images/lecture_11/process_3}}
        \only<5>{\includegraphics[width=\linewidth]{images/lecture_11/process_4}}
        \only<6>{\includegraphics[width=\linewidth]{images/lecture_11/process_5}}
        \only<7>{\includegraphics[width=\linewidth]{images/lecture_11/process_6}}
        \only<8>{\includegraphics[width=\linewidth]{images/lecture_11/process_7}}
        \only<9>{\includegraphics[width=\linewidth]{images/lecture_11/process_8}}
        \only<10>{\includegraphics[width=\linewidth]{images/lecture_11/process_9}}
        \only<11>{\includegraphics[width=\linewidth]{images/lecture_11/process_10}}
        \only<12>{\includegraphics[width=\linewidth]{images/lecture_11/process_end}}
    \end{frame}

    \begin{frame}{Toolchain}
        \only<1>{\includegraphics[width=\linewidth]{images/lecture_11/flow_start}}
        \only<2>{\includegraphics[width=\linewidth]{images/lecture_11/flow_1}}
        \only<3>{\includegraphics[width=\linewidth]{images/lecture_11/flow_2}}
        \only<4>{\includegraphics[width=\linewidth]{images/lecture_11/flow_3}}
        \only<5>{\includegraphics[width=\linewidth]{images/lecture_11/flow_end}}
    \end{frame}

    \section{Circom}

    \begin{frame}[fragile]{Previously on ZKDL Camp}
        Probably you can recall the function
        \begin{lstlisting}[language=Python,numbers=none]
        def r(x1: F, x2: F, x3: F) -> F:
            return x2 * x3 if x1 else x2 + x3
        \end{lstlisting}
        \vspace{-10pt}
        \pause That can be expressed as:
        \begin{equation*}
            r = x_1 \times (x_2 \times x_3) + (1 - x_1) \times (x_2 + x_3)
        \end{equation*}

        \pause
        \begin{alertblock}{}
            We need a boolean restriction for $x_1$:
            \vspace{-8pt}
            \begin{equation*}
                x_1 \times (1 - x_1) = 0
            \end{equation*}
        \end{alertblock}

        \pause
        Thus, the next constraints can be build:
        \vspace{-5pt}
        \begin{align*}
            x_1 \times x_1 &= x_1 \quad \text{(binary check)} \tag{1} \\
            x_2 \times x_3 &= \mathsf{mult} \tag{2} \\
            x_1 \times \mathsf{mult} &= \mathsf{selectMult} \tag{3} \\
            (1 - x_1) \times (x_2 + x_3) &= r - \mathsf{selectMult} \tag{4}
        \end{align*}
    \end{frame}

    \begin{frame}[fragile]{Previously on ZKDL Camp}
        The witness vector: $\boldsymbol{w} = (1, r, x_1, x_2, x_3, \mathsf{mult}, \mathsf{selectMult})$.
        \pause The coefficients vectors:

        \vspace{-20pt}\small

        \begin{align*}
            \boldsymbol{a}_1 &= (0, 0, 1, 0, 0, 0, 0), & \boldsymbol{b}_1 &= (0, 0, 1, 0, 0, 0, 0), & \boldsymbol{c}_1 &= (0, 0, 1, 0, 0, 0, 0) \\
            \boldsymbol{a}_2 &= (0, 0, 0, 1, 0, 0, 0), & \boldsymbol{b}_2 &= (0, 0, 0, 0, 1, 0, 0), & \boldsymbol{c}_2 &= (0, 0, 0, 0, 0, 1, 0) \\
            \boldsymbol{a}_3 &= (0, 0, 1, 0, 0, 0, 0), & \boldsymbol{b}_3 &= (0, 0, 0, 0, 0, 1, 0), & \boldsymbol{c}_3 &= (0, 0, 0, 0, 0, 0, 1) \\
            \boldsymbol{a}_4 &= (1, 0, -1, 0, 0, 0, 0), & \boldsymbol{b}_4 &= (0, 0, 0, 1, 1, 0, 0), & \boldsymbol{c}_4 &= (0, 1, 0, 0, 0, 0, -1)
        \end{align*}\normalsize

        \pause \vspace{-10pt}

        Using the arithmetic in a large $\mathbb{F}_p$, consider the following values:

        \vspace{-15pt}
        \begin{equation*}
            x_1 = 1, \quad x_2 = 3, \quad x_3 = 4
            \vspace{-5pt}
        \end{equation*}

        Verifying the constraints:
        \begin{enumerate}
            \item \( x_1 \times x_1 = x_1 \quad (1 \times 1 = 1) \)
            \item \( x_2 \times x_3 = \mathsf{mult} \quad (3 \times 4 = 12) \)
            \item \( x_1 \times \mathsf{mult} = \mathsf{selectMult} \quad (1 \times 12 = 12) \)
            \item \( (1 - x_1) \times (x_2 + x_3) = r - \mathsf{selectMult} \quad (0 \times 7 = 12 - 12) \)
        \end{enumerate}
    \end{frame}

    \begin{frame}{Previously on ZKDL Camp}
        By Groth16 Protocol the verifier \only<1>{...}\pause should check the following condition:
        \begin{equation*}
            e(\pi_L, \pi_R) = e(g_1^{\alpha}, g_2^{\beta})e(\pi_{\text{io}},g_2^{\gamma})e(\pi_O,g_2^{\delta})
        \end{equation*}

        \pause

        \begin{alertblock}{Recall}
            For BN254 (BN128), we have:
            \begin{itemize}
                \item Left inputs to $e$ is of form $(x,y) \in \mathbb{G}_1$ --- regular curve.
                \item Right inputs to $e$ is of form $((x_1,y_1),(x_2,y_2)) \in \mathbb{G}_2$ --- ``complex'' curve, consisting of two $\mathbb{F}_{p^2}$ coordinates.
                \item $e(g_1^{\alpha}, g_2^{\beta})$ is of form $(x_1,\dots,x_{12}) \in \mathbb{F}_{p^{12}}$
            \end{itemize}
        \end{alertblock}
    \end{frame}

    \begin{frame}[plain, standout]
        \centering
        \LARGE
        \textbf{Thank you for your attention} \\

        \vspace{0.2cm} \Huge \ding{170} \large \\

        \vspace{1cm}

        \href{https://zkdl-camp.github.io/}{\raisebox{-.1em}{\hspace{.025em}\faIcon{globe}}\hspace{.325em}zkdl-camp.github.io} \\

        \href{https://github.com/ZKDL-Camp}{\raisebox{-.1em}{\hspace{.025em}\faIcon{github}}\hspace{.325em}github.com/ZKDL-Camp}

        \begin{center}
            \includegraphics[width=0.15\textwidth]{images/logo.png}
        \end{center}
    \end{frame}
\end{document}