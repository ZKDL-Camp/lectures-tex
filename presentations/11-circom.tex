\documentclass{zkdl-presentation-template}
\usepackage{blkarray}
\usepackage{blkarray}

\title[Circom]{\textbf{Circom}}
\author{Distributed Lab}
\date{November 28, 2024}
\homepage{zkdl-camp.github.io}
\github{ZKDL-Camp}

\begin{document}
    \frame {
        \tikz [remember picture,overlay]
        \node at
        ([yshift=1.5cm,xshift=-1.5cm]current page.south east)
    %or: (current page.center)
            {\includegraphics[width=60pt]{images/logo.png}};

        \titlepage
    }

    \begin{frame}{Plan}
        \tableofcontents
    \end{frame}


    \section{Introduction}

    \begin{frame}{Why do we need ZK?}
        \pause

        \begin{block}{Option}
            Solution to privacy
        \end{block}

        \pause

        \begin{example}
            \begin{enumerate}
                \item \textit{I know the private key that corresponds to this public key}
                \item \textit{I know a private key that corresponds to a public key from this list}
            \end{enumerate}
        \end{example}

        \pause

        \begin{block}{Option}
            Solution to scalability
        \end{block}

        \pause

        \begin{example}
            \textit{This is the hash of a blockchain block that does not produce negative balances}
        \end{example}
    \end{frame}

    \begin{frame}{Using ZKP}
        \only<1>{\includegraphics[width=\linewidth]{images/lecture_11/process_start}}
        \only<2>{\includegraphics[width=\linewidth]{images/lecture_11/process_1}}
        \only<3>{\includegraphics[width=\linewidth]{images/lecture_11/process_2}}
        \only<4>{\includegraphics[width=\linewidth]{images/lecture_11/process_3}}
        \only<5>{\includegraphics[width=\linewidth]{images/lecture_11/process_4}}
        \only<6>{\includegraphics[width=\linewidth]{images/lecture_11/process_5}}
        \only<7>{\includegraphics[width=\linewidth]{images/lecture_11/process_6}}
        \only<8>{\includegraphics[width=\linewidth]{images/lecture_11/process_7}}
        \only<9>{\includegraphics[width=\linewidth]{images/lecture_11/process_8}}
        \only<10>{\includegraphics[width=\linewidth]{images/lecture_11/process_9}}
        \only<11>{\includegraphics[width=\linewidth]{images/lecture_11/process_10}}
        \only<12>{\includegraphics[width=\linewidth]{images/lecture_11/process_end}}
    \end{frame}

    \begin{frame}{Toolchain}
        \only<1>{\includegraphics[width=\linewidth]{images/lecture_11/flow_start}}
        \only<2>{\includegraphics[width=\linewidth]{images/lecture_11/flow_1}}
        \only<3>{\includegraphics[width=\linewidth]{images/lecture_11/flow_2}}
        \only<4>{\includegraphics[width=\linewidth]{images/lecture_11/flow_3}}
        \only<5>{\includegraphics[width=\linewidth]{images/lecture_11/flow_end}}
    \end{frame}


    \section{Circom}

    \begin{frame}{It looks good, as long as all I need is to look at it.}
        \scriptsize

        \textbf{Trusted Setup: }
        $\tau, \alpha, \beta_L, \beta_R, \beta_O, \gamma \xleftarrow{R} \mathbb{F} \textbf{,} \quad \{ \{g^{\tau^i}, g^{\alpha \tau^i} \}_{i \in [d]}\textbf{,} \quad \{\textcolor{black!50!black}{g^{\beta_LL_i(\tau)}, g^{\beta_RR_i(\tau)}, g^{\beta_OO_i(\tau)}}\}_{i \in \left[ n \right]} \} \textbf{,}$
        $\{ g^{Z(\tau)}, g^{\alpha}, g^{\beta_L}, g^{\beta_R}, g^{\beta_O}, g^{\beta_L\gamma}, g^{\beta_R\gamma}, g^{\beta_O\gamma}, g^{\gamma}\}\textbf{,} \quad \textbf{delete}(\tau, \alpha, \beta_L, \beta_R, \beta_O, \gamma)$.
        \vspace{10pt}

        \tikzset{
            arrow/.style={-latex, ultra thick}
        }

        \begin{center}
            \begin{tikzpicture}
                \node[inner sep=0pt, align=center] (prover) at (-3.5, -3) {
                    \includegraphics[width=2cm]{images/common/prover.png}\\Prover $\mathcal{P}$
                };

                \node[inner sep=0pt, align=center] (verifier) at (4, -3) {
                    \includegraphics[width=2cm]{images/common/verifier.png}\\Verifier $\mathcal{V}$
                };

                \node[inner sep=0pt, align=center] (pdata) at (-3, 0) {
                    \begin{minipage}{4cm}
                    {\tiny\begin{itemize}[label=\ding{51}]
                              \item $H(x) = \frac{L(x) \times R(x) - O(x)}{Z(x)}$.
                              \item {
                                  Sample $\delta_L,\delta_R,\delta_O \xleftarrow{R} \mathbb{F}$, compute:
                                  \vspace{-8pt}
                                  \begin{align*}
                                      \pi_L \gets g^{L(\tau)}\textcolor{black!50!black}{(g^{Z(\tau)})^{\delta_L}}, \pi_L' \gets g^{\alpha L(\tau)}\textcolor{black!50!black}{(g^{\alpha Z(\tau)})^{\delta_L}},\\[-2pt]
                                      \pi_R \gets g^{R(\tau)}\textcolor{black!50!black}{(g^{Z(\tau)})^{\delta_R}}, \pi_R' \gets g^{\alpha R(\tau)}\textcolor{black!50!black}{(g^{\alpha Z(\tau)})^{\delta_R}},\\[-2pt]
                                      \pi_O \gets g^{O(\tau)}\textcolor{black!50!black}{(g^{Z(\tau)})^{\delta_O}}, \pi_O' \gets g^{\alpha O(\tau)}\textcolor{black!50!black}{(g^{\alpha Z(\tau)})^{\delta_O}},\\[-2pt]
                                      \pi_H \gets g^{H(\tau)}\textcolor{black!50!black}{(g^{\delta_O})(g^{R(\tau)})^{\delta_L}(g^{L(\tau)})^{\delta_R}(g^{Z(\tau)})^{\delta_L\delta_R}}
                                  \end{align*}
                                  \vspace{-16pt}
                              }

                              \vspace{-5pt}
                              $\pi_{\beta} = \dots$

                    \end{itemize}}
                    \end{minipage}
                };

                \draw[arrow] (prover) -- (verifier) node[midway, above=1mm] {$\boldsymbol{\pi} = (\pi_L,\pi_R,\pi_O,\pi_H,\pi_L',\pi_R',\pi_O',\pi_H',\pi_{\beta})$};

                \node[inner sep=0pt, align=center] (vdata) at (5, -0.2) {
                    \begin{minipage}{5cm}
                    {\tiny\begin{itemize}[label=\ding{51}]
                              \item $e(\pi_L, \pi_R) \xlongequal{?}\\ e(\mathsf{com}(Z), \pi_H) \cdot e(\pi_O, g)$.
                              \item {
                                  Proof of Exponent:
                                  \vspace{-8pt}
                                  \begin{align*}
                                      e(\pi_L, g^{\alpha}) &= e(\pi_L', g), \qquad\qquad\qquad\\[-2pt] %quads are to adjust spacing on the slide
                                      e(\pi_R, g^{\alpha}) &= e(\pi_R', g), \\[-2pt]
                                      e(\pi_O, g^{\alpha}) &= e(\pi_O', g), \\[-2pt]
                                      e(\pi_H, g^{\alpha}) &= e(\pi_H', g).
                                  \end{align*}
                                  \vspace{-16pt}
                              }
                              \item $e(\pi_L, g^{\gamma\beta_L}) \cdot e(\pi_R, g^{\gamma\beta_R}) \cdot \\ \cdot e(\pi_O, g^{\gamma\beta_O}) = e(\pi_{\beta}, g^{\gamma})$
                    \end{itemize}}
                    \end{minipage}
                };
            \end{tikzpicture}
        \end{center}
    \end{frame}

    \begin{frame}[fragile]{Something we can touch.}
        \begin{lstlisting}[language=Circom,numbers=none,label={lst:lstlisting1}]
pragma circom 2.1.6;

template Math() {
    signal output res;

    signal input a;
    signal input b;
    signal input c;

    a * (1 - a) === 0;

    signal mul1 <== a * b;
    signal mul2 <== a * c;
    signal res1 <== mul1 * mul2;

    signal res2 <== (1 - a) * (b + c);

    res1 + res2 ==> res;
}
        \end{lstlisting}
    \end{frame}


    \section{Practice}

    \begin{frame}[plain, standout]
        \centering
        \LARGE
        \textbf{Thank you for your attention} \\

        \vspace{0.2cm} \Huge \ding{170} \large \\

        \vspace{1cm}

        \href{https://zkdl-camp.github.io/}{\raisebox{-.1em}{\hspace{.025em}\faIcon{globe}}\hspace{.325em}zkdl-camp.github.io} \\

        \href{https://github.com/ZKDL-Camp}{\raisebox{-.1em}{\hspace{.025em}\faIcon{github}}\hspace{.325em}github.com/ZKDL-Camp}

        \begin{center}
            \includegraphics[width=0.15\textwidth]{images/logo.png}
        \end{center}
    \end{frame}
\end{document}