\documentclass{beamer}
\usepackage[utf8]{inputenc}
\usepackage[english]{babel}

% -- Including some standard packages --
\usepackage{graphicx}
\usepackage{soul}
\usepackage{hyperref}
\usepackage{colortbl}
\usepackage{dsfont}
\usepackage{soul}

% -- Choosing theme --

\usetheme{Boadilla}
\usecolortheme{spruce}
\setbeamercolor{alerted text}{fg=purple} % Making alerted text non-red

% Tikz
\usepackage{tikz,tikz-3dplot,tikz-cd,tkz-tab,tkz-euclide,pgf,pgfplots}
\usetikzlibrary{matrix,positioning,fit,backgrounds,intersections}

% -- Cross signs --
\usepackage{pifont} % http://ctan.org/pkg/pifont
\newcommand{\cmark}{\ding{51}}%
\newcommand{\xmark}{\ding{55}}%
\newcommand{\xopt}{\ding{48}}%

% -- Custom commands --
\DeclareMathOperator*{\argmax}{arg\,max}
\DeclareMathOperator*{\argmin}{arg\,min}

\title[Cryptographic commitments]{\textbf{Cryptographic commitments}}
\author{Distributed Lab}
\date{August 8, 2024}
\titlegraphic{
    \includegraphics[width=\textwidth]{images/banner_wide.png}
}

\expandafter\def\expandafter\insertshorttitle\expandafter{%
  \insertshorttitle\hfill%
  \insertframenumber\,/\,\inserttotalframenumber}

\AtBeginSection[]{
  \begin{frame}
  \vfill
  \centering
  \begin{beamercolorbox}[sep=8pt,center,shadow=true,rounded=true]{title}
    \usebeamerfont{title}\insertsectionhead\par%
  \end{beamercolorbox}
  \vfill
  \end{frame}
}

\begin{document}
    \frame {
      \titlepage
    }
  
    \begin{frame}{Plan}
      \tableofcontents
    \end{frame}

    \section{Field Extensions}

    \subsection{A bit of intuition}

    \begin{frame}{$\mathbb{Q}$ vs $\mathbb{R}$}
        \begin{alertblock}{Question \#1}
            What is the difference between rational numbers $\mathbb{Q}$ and real numbers $\mathbb{R}$?
        \end{alertblock}

        \begin{definition}
            \textbf{Rational numbers} $\mathbb{Q}$ are defined as the set $\{\frac{n}{m}: n \in \mathbb{Z}, m \in \mathbb{N}\}$.
        \end{definition}

        \begin{alertblock}{Question \#2}
            Why cannot we say $m \in \mathbb{Z}$, similarly to $n$?
        \end{alertblock}

        \begin{theorem}
            $\sqrt{2}$ is not a rational number. Neither is $\pi$ and $e$. But they are reals.
        \end{theorem}

        \begin{block}{Conclusion}
            $\mathbb{R}$ is sort of ``an extended version of $\mathbb{Q}$''.
        \end{block}
    \end{frame}

    \begin{frame}{What about $\mathbb{R}$?}
        \begin{alertblock}{Rethorical Question}
           Can we extend $\mathbb{R}$?
        \end{alertblock}

        Yes --- just use complex numbers $\mathbb{C}$!

        \begin{definition}
            Complex numbers $\mathbb{C}$ is defined as the set of $x+iy$ where $i^2=-1$.
        \end{definition}

        \begin{definition}
            Complex numbers $\mathbb{C}$ are defined as the set of pairs $(x,y) \in \mathbb{R}^2$ where addition is defined as $(x_1,y_1)+(x_2,y_2)=(x_1+x_2,y_1+y_2)$, and the multiplication is:
            \begin{equation*}
                (x_1,y_1) \cdot (x_2,y_2) = (x_1x_2-y_1y_2,x_1y_2+x_2y_1).
            \end{equation*}
        \end{definition}
    \end{frame}


    \begin{frame}{}
        \centering \Large
        \emph{Thank you for your attention!}
      \end{frame}
\end{document}