\documentclass{zkdl-presentation-template}

\title[Introduction into ZK-STARK protocol]{\textbf{Introduction into ZK-STARK protocol}}
\author{Distributed Lab}
\date{January 23, 2025}
\homepage{zkdl-camp.github.io}
\github{ZKDL-Camp}

\begin{document}
\frame {
    \tikz [remember picture,overlay]
    \node at
    ([yshift=1.5cm,xshift=-1.5cm]current page.south east)
%or: (current page.center)
        {\includegraphics[width=60pt]{images/logo.png}};

    \titlepage
}

\begin{frame}{Plan}
    \tableofcontents
\end{frame}

\section{Introduction}

\begin{frame}{What is STARK?}
    ZK-STARK -- Zero-Knowledge Scalable Transparent Argument of Knowledge.
    \begin{itemize}
        \item \textit{scalable} implies that the proving time grows at most quasilinearly (linear up to the logarithmic factor) relative to the witness-checking process. Additionally, the verification is limited to a polylogarithmic growth concerning same process.
        \item \textit{transparent} means there is no requirement for a trusted setup.
    \end{itemize}
\end{frame}

\begin{frame}{STARK is a SNARK?}
    Non-interactive STARK = transparent SNARK. All existing protocols in production are non-interactive.
\end{frame}

\section{STARK-friendly fields}

\begin{frame}{Two-adicity fields}
    \begin{definition}
        We call two-adicity fields, the fields where we can select the multiplicative subgroup of order $2^k$.
    \end{definition}

    For the multiplicative group generator $w \in \mathbb{F}_N^{\times}$, the generator of the two-adicity subgroup will be $w^\frac{N - 1}{2^k}$.

    Example fields:
    \begin{itemize}
        \item Goldilocks field: $N=2^{64} - 2^{32} + 1$
        \item Mersenne31 field: $N=2^{31} - 1$
        \item StarkNet field: $N=2^{251} + 17 \cdot 2^{192} + 1$
    \end{itemize}
\end{frame}

\begin{frame}
    $h$ -- two-adicity group $H$ generator. 

    \begin{equation*}
        h = w^{\frac{N - 1}{|H|}}    
    \end{equation*}

    \begin{equation*}
        \forall x \in H, x = h^i = w^{\frac{N - 1}{|H|} \cdot i}
    \end{equation*}

    \begin{equation*}
        -x = h^j = w^{\frac{N - 1}{|H|} \cdot j}
    \end{equation*}

    Then, the $i$ and $j$ values obtain the following property: 
    \begin{equation*}
        j = i + \frac{|H|}{2} \mod |H|   
    \end{equation*}    

\end{frame}

\section{Witness and commitments}

\begin{frame}{Trace}
    \begin{definition}
        We call \textbf{trace} a sequence of elements from $\mathbb{F}$ that represents our witness.
    \end{definition}

    \begin{definition}
        We call \textbf{domain} a two-adicity subgroup $G \in \mathbb{F}$ where we evaluate our polynomials.
    \end{definition}
\end{frame}

\begin{frame}
    \begin{example}
        The Fibonacci square sequence is a sequence of elements defined as follows:
        \begin{equation*}
        a_{i} = a_{i-1}^2 + a_{i-2}^2  
        \end{equation*}

        We gonna evaluate this sequence under the prime modulus $N = 3\cdot 2^{30} + 1$. Then, we can prove for example the following statement:
        \begin{itemize}
            \item \textit{I know a field element $x$ such that the $1023$rd element of the Fibonacci square sequence starting with $1$ and $x$ is $2338775057$.} 
        \end{itemize}

        (The private $x$ in this case equals to $3141592$).
    \end{example}
\end{frame}

\begin{frame}
    \begin{example}
        In our example, we put trace a sequence $a$ of first $1023$ elements of the Fibonacci square sequence over $\mathbb{F}_N$, where $N=3\cdot 2^{30} + 1$.
        \begin{equation*}
        1, 1, 2, 5, 29, ...
        \end{equation*}
        To interpolate our trace polynomial we select as a domain a two-adicity subgroup of $2^{10}$ elements from $\mathbb{F}^\times$ with generator $g = 5^{\frac{3\cdot 2^{30}}{2^{10}}}$ (here $5$ stands for the primitive element in $\mathbb{F}^\times_N$):
        \begin{equation*}
        G = \{g^i \;|\; g = 5^{3\cdot 2^{20}} \land i \in [0;1024) \}
        \end{equation*}
    \end{example}
\end{frame}

\begin{frame}
    Using any interpolation scheme over $(g^i, a_i)_0^{|a| - 1}$ points we compute a trace polynomial $f \in \mathbb{F}[x]$.
\end{frame}

\begin{frame}
    \begin{definition}
        We call \textbf{evaluation domain} a two-adicity coset $E = wH\in \mathbb{F}$, where $H \in \mathbb{F}$ is a two-adicity subgroup, that is larger $\rho$ times (some small constant) then the domain. 
    \end{definition}

    \begin{example}
        In our case we select a two-adicity subgroup of $2^{13}$ elements from $\mathbb{F}^\times$ ($\rho = 8$): 
        \begin{equation*}
        H = \{h^i \;|\; h = 5^{3\cdot 2^{17}} \land i \in [0;8192) \}
        \end{equation*}
        Then, we define the evaluation domain as:
        \begin{equation*}
        E = \{5\cdot h_i \;|\; \forall h_i \in H\}  
        \end{equation*}
        \end{example}
\end{frame}


\begin{frame}{Commitment}
    We build a Merkle tree over the values $f(e_i),\; \forall e_i \in E$ and label it's root as a \textbf{trace polynomial commitment}. This approach will also be used to commit other polynomials during the protocol walkthrough.    
\end{frame}

\begin{frame}{Constraints}
    The \textbf{constraints} in STARK protocol are expressed as polynomials evaluated over the trace cells, which are satisfied if and only if the computations are correct.
    
    \begin{example}
        Obviously, our initial statement consists of the following three requirements:
        \begin{enumerate}
            \item The element $a_0$ is equal to $1$;
            \item The element $a_{1022}$ is equal to $2338775057$;
            \item Each element $a_{i+2}$ is equal to $a_{i+1}^2 + a_{i}^2 \mod N$.
        \end{enumerate}
    \end{example}
\end{frame}

\begin{frame}
    The relation $r(a_i, a_j) = 0$ can be rewritten as $r(f(g^i), f(g^j)) = 0$.

    \begin{example}
        For our Fibonacci trace we have the following constraints to be checked over the interpolated polynomial:
        \begin{enumerate}
            \item \textit{The element $a_0$ is equal to $1$} translated to: $f(x)-1$ has root at $x = g^0 = 1$;
            \item \textit{The element $a_{1022}$ is equal to $2338775057$} translated to: $f(x) - 2338775057$ has root at $x = g^{1022}$;
            \item \textit{Each element $a_{i+2}$ is equal to $a_{i+1}^2 + a_{i}^2$} translated to: $f(g^2x) - f(gx)^2 - f(x)^2$ has roots in $G \setminus \{g^{1021}, g^{1022}, g^{1023}\}$
        \end{enumerate}
    \end{example}


    Note, that the verifier should be able to compute the constraints polynomials $p_i(x)$ using only the given trace polynomial evaluations for the certain $x$.
    
\end{frame}

\begin{frame}{Composition polynomial}
    \begin{equation*}
        CP(x) = \sum \alpha_i\cdot p_i(x)
    \end{equation*}

    \begin{example}
        The Fibonacci composition polynomial looks like as follows:
        \begin{gather*}
            CP(x) = \alpha_0 p_0(x) + \alpha_1 p_1(x) + \alpha_2 p_2(x) =\\
            \alpha_0 \frac{f(x)-1}{x - 1} + \alpha_1 \frac{f(x) - 2338775057}{x - g^{1022}} + \\
            \alpha_2 \frac{(f(g^2x) - f(gx)^2 - f(x)^2)(x - g^{2021})(x - g^{2022})(x - g^{2024})}{x^{1024} - 1}
        \end{gather*}
    \end{example}
\end{frame}

\section{FRI}

\begin{frame}
    FRI - \textbf{Fast Reed-Solomon IOP of Proximity}
    \begin{gather*}
        z_0(x) = \sum a_i\cdot x^i\\
        z_0^o(x^2) = \sum_{i=0}^{n/2} (a_{2i+1}\cdot x^{2i})\\
        z_0^e(x^2) = \sum_{i=0}^{n/2} (a_{2i}\cdot x^{2i})
    \end{gather*}
    Or, in more comfortable form:
    \begin{gather*}
        z_0^e(x^2) = \frac{z_0(x) + z_0(-x)}{2}\\
        z_0^o(x^2) = \frac{z_0(x) - z_0(-x)}{2x}
    \end{gather*}
\end{frame}

\begin{frame}{Next layer}
    \begin{gather*}
        z_1(x^2) = z_0^e(x^2) + \beta z_0^o(x^2)\\
        E_1 = \{(w\cdot h_i)^2 \;|\; i \in [0;\frac{|E_0|}{2})\}
    \end{gather*}
\end{frame}

\section{Protocol definition}

\begin{frame}  
    The prover and the verifier run the interactive version of the ZK-STARK protocol. Both know the statement to be proved, that is defined by the constraint polynomials and the field $\mathbb{F}$ to work over. Prover also knows the witness to be able to generate the trace.

    Preparation:
    \begin{itemize}[label=\ding{51}]
        \item The prover interpolates trace polynomial $f(x)$ and submits it's commitment to the verifier.
        \item The verifier selects challenges random $\alpha_0, \alpha_1, \alpha_2 \in \mathbb{F}$ and sends to the prover.
        \item The prover builds the composition polynomial $CP(x)$ and submits it's commitment to the verifier.
    \end{itemize}
\end{frame}

\begin{frame}
    FRI:
    \begin{itemize}[label=\ding{51}]
        \item The verifier selects random $i \in [0; |E|)$, puts $c = w\cdot h^i$ and sends it to the prover.
        \item The prover responds with the $CP(c), CP(-c)$ and all $f(x)$ required to check $CP$ evaluation with corresponding Merkle proofs to them.
        \item The verifier checks Merkle proofs and the evaluation of $CP(c)$ by evaluating the constraints polynomials $p_i(c)$.
        \item The prover and the verifier go through the FRI protocol for $z_0(x) = CP(x)$ where the prover commits to the layer-$i$ polynomial $z_i(x)$, the verifier selects a challenge $\beta$ and queries from the prover $z_i(c), z_i(-c)$ to compute $z_{i+1}(c)$ until $z_i(x), i \leq log_2(\deg CP(x))$ becomes constant.
    \end{itemize}
\end{frame}

\section{Security}

\begin{frame}
    \begin{itemize}
        \item Blowup factor $\rho$
        \item Proof-of-work bits $\delta$
        \item NUmber of queries $s$
    \end{itemize}

    \begin{equation*}
        \lambda \geq min\{ \delta + log_2(\rho) \cdot s, log_2(|F|) \} - 1
    \end{equation*}

    \begin{example}
        If the protocol is deployed over $256$-bit field and the domain ratio is $\rho = 3$, to achieve the $128$ bit security we can for example execute $33$ FRI query and evaluate $29$ proof-of-work bits:
        \begin{equation*}
            min\{29+3\cdot 33, 256\} = 128
        \end{equation*}
    \end{example}
\end{frame}


\end{document}