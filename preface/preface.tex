\documentclass[../lecture-notes-148x210.tex]{subfiles}

\begin{document}

\textbf{Motivation.} Cryptography is wonderful. If you are holding this book, you are probably 
already know the significance of cryptography in the modern 
informational infrastructure: from classical standard protocols such as 
Transport Layer Security (TLS) to the more recent cryptographic
advancements used in Blockchain technologies. This book is 
exactly about the latter.

In particular, as the book's name suggests, we consider the \emph{zero-knowledge
cryptography}, which is a cornerstone of numerous privacy-preserving protocols.
However, in its current state, the zero-knowledge technology is
\emph{tremendously} hard to work with:
\begin{itemize}
    \item The technology is relatively new, so there are almost 
    no high-level abstractions that can be used to simplify the
    design of zero-knowledge protocols for regular developers.
    \item The technology is based on advanced mathematics, which is hard to
    understand without a proper background.
    \item The amount of available resources on the topic is surprisingly limited.
    What's more, the resources are very diverse: they are either (a) too
    high-level, which makes you understand what the protocol \emph{should} do,
    but you still have no idea how to implement it, (b) or too low-level,
    consisting of numerous security proofs and obscure theorems with little
    practical implications. Note that in the latter case you typically still
    have no clue what was going on, unless you spend a significant amount of
    time studying the topic.
\end{itemize}

Of course, this book cannot solve all the aforementioned problems at once.
However, we do hope that this book will serve as a complete, practical guide to
most state-of-the-art techniques in zero-knowledge cryptography for practicing
engineers. We gathered all the necessary information in one place, and tried to
make it easy-to-follow, with a lot of examples and code snippets. Yet, we still
try to preserve the mathematical rigor where suitable and necessary.

\textbf{Who are we?} We are \href{https://distributedlab.com/}{\emph{Distributed
Lab}} --- a Ukrainian cryptography and engineering team focused on cryptography
related projects (primarily in Bitcoin and EVM ecosystems). We have been working
on numerous zero-knowledge solutions involving optimizing advanced algebraic
system components, ranging from fast Groth16 RSA or 384-bit Brainpool ECDSA
verification to PlonK circuits acceleration of BN254 elliptic curve operations
and pairing. We are significantly contributing to the Bitcoin ecosystem and
developing payment and social applications with the focus on privacy (e.g.
\href{https://rarimo.com/}{\emph{Rarimo}}).

\textbf{Who is this book for?} This book does require basic background in
Mathematics. Of course, we will try to explain all the necessary concepts from
scratch (such as basic number and group theory, security analysis), but this
material is rather supplementary and is not the main focus of the book. That
being said, you do not need Math PhD, but be prepared: the book is challenging.

\textbf{How to use this book?} The book is structured in the following way:
\begin{enumerate}
    \item The first part of the book, consisting of sections from
    \Cref{section:number-theory} to \Cref{section:security-analysis}, is dedicated to
    the theoretical background on Mathematics and Cryptography needed for
    understanding the zero-knowledge protocols. If you feel comfortable with
    polynomials, groups, elliptic curves and commitment schemes, you can skip
    this part.
    \item The main part of the book starts from \Cref{section:intro-zk}, where 
    we first define the zero-knowledge proofs and explain the basic concepts
    behind them. We then proceed to the more advanced topics, such as Sigma
    protocols (\Cref{section:sigma}), SNARKs (from \Cref{section:r1cs}
    to \Cref{section:circom}), and PlonK (\Cref{section:plonk}).
\end{enumerate}

\end{document}
