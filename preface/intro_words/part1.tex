\documentclass[../../lecture-notes-148x210.tex]{subfiles}

\begin{document}

As any cryptography subfield, Cryptography requires a solid knowledge of
Mathematics. While the practical development of cryptographic protocols might
not include the need to understand the whole underlying theory, much of it
typically still arises. For example, one might need to implement the Elliptic
Curve Pairing operation verification in a Zero-Knowledge Proof system, or RSA
encryption and decryption methods in Circom circuits. In case you specifically
develop the zero-knowledge systems from scratch, there is no need to explain why
you need to understand the underlying theory.

This part of the book is dedicated to the mathematical background required for
Cryptography. Namely, we will cover essentials specified in \Cref{tab:essentialls}.

\begin{table}[H]
    \centering
    \begin{tabularx}{\textwidth}{|c|c|X|}
        \hline
        \rowcolor{blue!30}\textbf{Section} & \textbf{Topic} & \textbf{Key Concepts} \\
        \hline
        \rowcolor{blue!10}\ref{section:number-theory} & Number Theory & Modular
        Arithmetic, Ring $\mathbb{Z}_n$ and field $\mathbb{Z}_n^{\times}$,
        Fermat's Little Theorem \\
        \hline
        \rowcolor{blue!20}\ref{section:abstract-algebra} & Abstract Algebra &
        Groups, Subgroups, Fields, Prime field $\mathbb{F}_p$, Isomorphisms,
        Automorphisms \\
        \hline
        \rowcolor{blue!10}\ref{section:polynomial-rings} & Polynomials & Divisibility, Lagrange Interpolation, Shamir's Secret Sharing \\
        \hline
        \rowcolor{blue!20}\ref{section:linear-algebra} & Linear Algebra Basics & Basic Operations over Vectors and Matrices \\
        \hline
        \rowcolor{blue!10}\ref{section:finite-fields} & Fields Extensions & Definition of $\mathbb{F}_{p^m}$, Field Multiplicative Subgroup, Algebraic Closure \\
        \hline
    \end{tabularx}
    \caption{Topics covered in Part I}
    \label{tab:essentialls}
\end{table}

This is not the Mathematics book, so we will not prove every mentioned theorem
or lemma, but we do provide reasoning for some key facts. We
focus on the practical side of the theory, providing examples and exercises for
the reader to solve.

The reader should already be familiar with very basics of the first four topics,
so they will be covered for the sake of completeness and reminding the reader of
the key concepts. The last topic, Fields Extensions, is less frequent in 
the Mathematics curriculum, so we will cover it in more detail.

In case you feel comfortable with the aforementioned topics, feel free to skip 
them and move to the next part of the book. Finally, if you are unsure about 
your knowledge, you should open the corresponding section and check whether 
you can solve exercises in the end without any help.


\end{document}