\documentclass[../../lecture-notes-148x210.tex]{subfiles}

\begin{document}

With all Mathematics preliminaries covered, we can now move to the next part of
the book. This part is dedicated to the mathematical background required for
Cryptography. Namely, we will cover essentials specified in \Cref{tab:essentialls-2}.

\begin{table}[H]
    \centering
    \begin{tabularx}{\textwidth}{|c|c|X|}
        \hline
        \rowcolor{green!30}\textbf{Section} & \textbf{Topic} & \textbf{Key Concepts} \\
        \hline
        \rowcolor{green!10}\ref{section:elliptic-curves} & Elliptic Curves, $\mathsf{ecpairing}$ & Group of points on Elliptic Curve, Projective Coordinates, elliptic curve pairing \\
        \hline
        \rowcolor{green!20}\ref{section:commitment-schemes} & Commitment Schemes &
        Hash-based Commitments, Pedersen Commitments, KZG Commitments \\
        \hline
        \rowcolor{green!10}\ref{section:security-analysis} & Security Analysis & Advantage, negligible function, DL/CDH/DDH Assumptions \\
        \hline
    \end{tabularx}
    \caption{Topics covered in Part II}
    \label{tab:essentialls-2}
\end{table}

In the \Cref{section:elliptic-curves}, we consider the core object in the
majority of modern zk-SNARK systems, the Elliptic Curves. Besides considering
the definition and basics, we also describe the projective coordinates and the
elliptic curve pairing operation. In the \Cref{section:commitment-schemes}, we
consider the commitment schemes: hash-based, Pedersen, and KZG commitments. In
the \Cref{section:security-analysis}, we consider the basics of security
analysis, which, from the practical standpoint, is needed to read 
cryptographic papers and understand which security assumptions are used in the
protocols and how authors typically describe them.


\end{document}